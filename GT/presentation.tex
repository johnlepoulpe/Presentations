\documentclass[11pt,xcolor=dvipsnames,presentation]{beamer}

\usepackage{BeamerColor}
\usepackage{etoolbox}

\usepackage[T1]{fontenc}%
\usepackage[utf8]{inputenc}%
\usepackage[main= english,francais]{babel}%
\usepackage{mathtools}

\usepackage{graphicx}%
\usepackage{courier}%
\usepackage{algorithm2e}

\usepackage{url}%
\urlstyle{sf}%

\usepackage{subcaption}%
\usepackage{caption}
\captionsetup[figure]{labelformat=empty}% redefines the caption setup of the figures environment in the beamer class.
\usepackage{tikz}

\usepackage{listings}%

\usepackage{faktor}%

\usepackage{stmaryrd}%

\usepackage[style=verbose,backend=biber]{biblatex}
\addbibresource{biblio.bib}

\newcommand\blfootnote[1]{%
  \begingroup
  \renewcommand\thefootnote{}\footnote{#1}%
  \addtocounter{footnote}{-1}%
  \endgroup
}

%%%%%%%%% Begin of Beamer Layout %%%%%%%%%%%%%
\ProcessOptionsBeamer
\usecolortheme{whale}
\usecolortheme[named=DarkGreen]{structure}
\useinnertheme{rounded}
\useoutertheme{infolines}
\setbeamertemplate{footline}[frame number]
\setbeamertemplate{headline}[default]
\setbeamertemplate{navigation symbols}{}
\setbeamertemplate{itemize items}[circle]
\defbeamertemplate*{headline}{info theme}{}
\defbeamertemplate*{footline}{info theme}{\leavevmode%
\hbox{%
\begin{beamercolorbox}[wd=.2\paperwidth,ht=2.25ex,dp=1ex,center]{author in head/foot}%
\usebeamerfont{author in head/foot}\insertshortauthor
\end{beamercolorbox}%
\begin{beamercolorbox}[wd=.71\paperwidth,ht=2.25ex,dp=1ex,center]{title in head/foot}%
\usebeamerfont{title in head/foot}\insertsectionhead
\end{beamercolorbox}%
\begin{beamercolorbox}[wd=.09\paperwidth,ht=2.25ex,dp=1ex,right]{section in head/foot}%
\usebeamerfont{section in head/foot}\insertframenumber{}~/~\inserttotalframenumber\hspace*{2ex}
\end{beamercolorbox}
}\vskip0pt}
\setbeamercolor*{block title example}{fg=LightSeaGreen}
\AtBeginEnvironment{exampleblock}{\setbeamercolor{itemize item}{fg=LightSeaGreen}}
\setbeamercolor*{block title alerted}{fg=DarkOrange2}
\AtBeginEnvironment{alertblock}{\setbeamercolor{itemize item}{fg=DarkOrange2}}
\setbeamertemplate{footline}[info theme]
\defbeamertemplate{description item}{align left}{\insertdescriptionitem\hfill}
%%%%%%%%% End of Beamer Layout %%%%%%%%%%%%%
\newcommand{\myorange}{DarkOrange2}
\newcommand{\mygreen}{DarkGreen}
\newcommand{\mycyan}{LightSeaGreen}


\lstset{%
	basicstyle=\sffamily,%
	columns=fullflexible,%
	language=C++,% % À ajuster dans chaque projet!
	frame=lb,%
	frameround=fftf,%
}%

\sloppy

%\setbeamertemplate{headline}{}%

\setbeamertemplate{navigation symbols}{%
	\insertslidenavigationsymbol%
	% \insertframenavigationsymbol%
	% \insertsubsectionnavigationsymbol%
	% \insertsectionnavigationsymbol%
	% \insertdocnavigationsymbol%
	% \insertbackfindforwardnavigationsymbol%
}

\newcommand{\backupbegin}{
   \newcounter{finalframe}
   \setcounter{finalframe}{\value{framenumber}}
}
\newcommand{\backupend}{
   \setcounter{framenumber}{\value{finalframe}}
}

\DeclareMathOperator{\diam}{diam}
\DeclareMathOperator{\tw}{tw}

\begin{document}

\title{Kempe changes on $\Delta$-colorings}
\author[Clément Legrand]{Clément Legrand-Duchesne}
\institute{ENS Rennes}
%\date{\today}
\date{\today}

\begin{frame}
  \titlepage

  \centering Joint work with Marthe Bonamy and Vincent Delecroix at LaBRI in Bordeaux
\end{frame}

\section*{Some context}

\begin{frame}{Recoloring with Kempe changes}
  \uncover<1->{
  \begin{columns}
    \begin{column}{.66\textwidth}
      \begin{exampleblock}{Kempe chain (1879)}
        Maximal bichromatic connected component in $G$
      \end{exampleblock}
    \end{column}
    \begin{column}{.33\textwidth}
      \centering
      \begin{tikzpicture}
        \node[draw=black,fill=\myorange,circle,inner sep=2pt]   (a) at (90:.8cm) {};
        \node[draw=black,fill=\mycyan,circle,inner sep=2pt]  (b) at (162:.8cm) {};
        \node[draw=black,fill=magenta,circle,inner sep=2pt] (c) at (234:.8cm) {};
        \node[draw=black,fill=blue,circle,inner sep=2pt] (d) at (306:.8cm) {};
        \node[draw=black,fill=magenta,circle,inner sep=2pt] (e) at (18:.8cm) {};

        \draw (a) -- (b) -- (c) -- (d) -- (e) -- (a);
      \end{tikzpicture}
      
      %\includegraphics[width=.8\textwidth]{../figures/pdf/knot_example.pdf}
    \end{column}
  \end{columns}
  }

  \vspace{1.5cm}
  
   \begin{columns}
%%     \uncover<2->{
%%       \begin{column}{.33\textwidth}
%%         \centering
%%         \begin{tikzpicture}
%%           \node[draw=black,fill=\myorange,circle,inner sep=2pt]   (a) at (90:.8cm) {};
%%           \node[draw=black,fill=\mycyan,circle,inner sep=2pt]  (b) at (162:.8cm) {};
%%           \node[draw=black,very thick,fill=blue,circle,inner sep=2pt] (c) at (234:.8cm) {};
%%           \node[draw=black,very thick,fill=magenta,circle,inner sep=2pt] (d) at (306:.8cm) {};
%%           \node[draw=black,very thick,fill=blue,circle,inner sep=2pt] (e) at (18:.8cm) {};

%%           \draw (e) -- (a) -- (b) -- (c);
%%           \draw[very thick] (c) -- (d) -- (e);
%%         \end{tikzpicture}      
%%     \end{column}
%%     }
    \uncover<2->{
    \begin{column}{.45\textwidth}
       \centering
      \begin{tikzpicture}
        \node[draw=black,fill=\myorange,circle,inner sep=2pt]   (a) at (90:.8cm) {};
        \node[draw=black,very thick,fill=magenta,circle,inner sep=2pt]  (b) at (162:.8cm) {};
        \node[draw=black,very thick,fill=\mycyan,circle,inner sep=2pt] (c) at (234:.8cm) {};
        \node[draw=black,fill=blue,circle,inner sep=2pt] (d) at (306:.8cm) {};
        \node[draw=black,fill=magenta,circle,inner sep=2pt] (e) at (18:.8cm) {};

        \draw[very thick] (b) -- (c);
        \draw (a) -- (b);
        \draw (c) -- (d) -- (e) -- (a);
      \end{tikzpicture}
    \end{column}
    }
    \uncover<3->{
    \begin{column}{.45\textwidth}
       \centering
      \begin{tikzpicture}
        \node[draw=black,fill=\myorange,circle,inner sep=2pt]   (a) at (90:.8cm) {};
        \node[draw=black,fill=\mycyan,circle,inner sep=2pt]  (b) at (162:.8cm) {};
        \node[draw=black,fill=magenta,circle,inner sep=2pt] (c) at (234:.8cm) {};
        \node[draw=black,very thick,fill=\mycyan,circle,inner sep=2pt] (d) at (306:.8cm) {};
        \node[draw=black,fill=magenta,circle,inner sep=2pt] (e) at (18:.8cm) {};

        \draw (a) -- (b) -- (c) -- (d) -- (e) -- (a);
      \end{tikzpicture}
    \end{column}
    }
  \end{columns}
\end{frame}


%% \begin{frame}{Natural questions}
%%   \begin{exampleblock}{Reconfiguration graph}
%%     \begin{itemize}
%%     \item $V(R^k(G)) =$ $k$-colorings of $G$.
%%     \item $\alpha$ and $\beta$ adjacent  if $\alpha
%%       \xleftrightarrow[\text{Kempe}]{} \beta$
%%     \end{itemize}
%%   \end{exampleblock}

%%   \setbeamertemplate{description item}[align left] 
%%   \begin{description}
%%   \item[Reachability] Are $\alpha$ and $\beta$ Kempe-equivalent ? 
%%   \item[Connectivity] Is $R^k(G)$ connected ?
%%   \item[Diameter] Estimate $\diam(R^k(G))$ ?
%%   \end{description}

%%   \begin{block}{Application}
%%     \begin{itemize}
%%     \item Powerful tool (ex: Vizing theorem)
%%     \item Sampling coloring with a Markov chain
%%     \end{itemize}
%%   \end{block}
%% \end{frame}


\begin{frame}{$d$-degenerate graphs}
  \begin{exampleblock}{$d$-degenerate graph}
    $G$ is $d$-degenerate if for any $H \subset G$, $\delta(H)\le
    d$\\ Equivalently, $G$ admits an elimination ordering $v_1 \prec v_2 \dots
    \prec v_n$ such that $$\forall i, |N^+(v_i)| \le d$$
  \end{exampleblock}

%%   \begin{itemize}
%%   \item $G$ has degree less than $k$ implies $G$ $k$-degenerate
%%   \item $G$ not regular of degree less than $k$ implies $G$ $(k-1)$-degenerate
%%   \end{itemize}

  \vspace{1cm}
  \centering
    \begin{tikzpicture}[auto,node distance=1cm,thick]
      \clip (-.5,.8) rectangle (7.5,-1.5);
       
      \node[draw=black,circle,inner sep=2pt] (1) {};
      \node[draw=black,circle,inner sep=2pt] (2) [right of=1] {};
      \node[draw=black,circle,inner sep=2pt] (3) [right of=2] {};
      \node[draw=black,circle,inner sep=2pt] (4) [right of=3] {};
      \node[draw=black,circle,inner sep=2pt] (5) [right of=4] {};
      \node[draw=black,circle,inner sep=2pt] (6) [right of=5] {};
      \node[draw=black,circle,inner sep=2pt] (7) [right of=6] {};


      \draw (1) -- (2);
      \draw (3) -- (4) -- (5) -- (6) -- (7);
      \path[every node/.style={font=\sffamily\small}]
      (4) edge[bend right] node {} (6)
      (7) edge[bend right] node {} (5)
      (4) edge[bend left] node {} (2)
      (2) edge[bend left] node {} (5)
      (1) edge[bend left] node {} (4)
      (3) edge[bend right] node {} (7);

  \end{tikzpicture}
  
%%   \begin{block}{With $\Delta+ 1$ colors}
%%     \begin{itemize}
%%     \item $\forall G, \chi(G) \le \Delta +1$
%%     \item $\forall G, \forall k \ge \Delta+1, R^k(G)$ is connected
%%     \end{itemize}
%%   \end{block}
\end{frame}

\begin{frame}{Fundamental lemma}
  \begin{block}{Las Vergnas, Meyniel 1981}
  All $k$-colorings of a $d$-degenerate graph are Kempe-equivalent for
    $k \ge d+1$
  \end{block}

  \vspace{.4cm}
  \begin{columns}
    \begin{column}{.6\textwidth}
      Induction: Let $\alpha$ and $\beta$ be two colorings of $G$\\
      Consider $G' = G\setminus\{v_1\}$
      $$\alpha_{|G'} \mathop{\leadsto}_{K}
      \beta_{|G'}$$
      Lift sequence to $G$ then recolor $v_1$
    \end{column}
    \hfill
    \begin{column}{.3\textwidth}
      \centering
      \only<1>{
        \begin{tikzpicture}
          \clip (-.5,.7) rectangle (3,-2);
          \node[draw,fill=magenta, circle,inner sep=2pt] (a) at (0,0) {};
          \node[above] (A) at (0,0) {$v_1$};
          \node[draw,fill=blue, circle, inner sep=2pt] (b) at (0:1.5cm) {};
          \node[draw,very thick,fill=blue, circle, inner sep=2pt] (c) at (320:1.5cm) {};
          \node[draw,fill=\myorange, circle, inner sep=2pt] (d) at (340:1.5cm) {};
          
          \node (e) at (335:2.2cm) {};
          \node (f) at (345:2.2cm) {};

          \node (h) at (315:2.2cm) {};
          \node (g) at (320:2.22cm) {};
          \node (i) at (325:2.18cm) {};

          \node (j) at (0:2.2cm) {};

          \node[above] (G) at (2.6, -1.2) {$G'$};
          \draw[domain=120:180] plot ({2*(cos(\x)+1.3)}, {2*(sin(\x)-.6)});

          \draw (j) -- (b) -- (a) -- (c);
          \draw[very thick] (h) -- (c) -- (g);
          \draw (a) -- (d);
          \draw (f) -- (d) -- (e);
          \draw (i) -- (c);
          
      \end{tikzpicture}}

      \only<2>{
        \begin{tikzpicture}
          \clip (-.5,.7) rectangle (3,-2);
          \node[draw,very thick,fill=magenta, circle,inner sep=2pt] (a) at (0,0) {};
          \node[above] (A) at (0,0) {$v_1$};
          \node[draw,very thick,fill=blue, circle, inner sep=2pt] (b) at (0:1.5cm) {};
          \node[draw,very thick,fill=blue, circle, inner sep=2pt] (c) at (320:1.5cm) {};
          \node[draw,fill=\myorange, circle, inner sep=2pt] (d) at (340:1.5cm) {};
          
          \node (e) at (335:2.2cm) {};
          \node (f) at (345:2.2cm) {};

          \node (h) at (315:2.2cm) {};
          \node (g) at (320:2.22cm) {};
          \node (i) at (325:2.18cm) {};

          \node (j) at (0:2.2cm) {};

          \node[above] (G) at (2.6, -1.2) {$G'$};
          \draw[domain=120:180] plot ({2*(cos(\x)+1.3)}, {2*(sin(\x)-.6)});

          \draw[very thick] (j) -- (b) -- (a) -- (c);
          \draw[very thick] (h) -- (c) -- (g);
          \draw (a) -- (d);
          \draw (f) -- (d) -- (e);
          \draw (i) -- (c);
        \end{tikzpicture}}

      \only<3->{
        \begin{tikzpicture}
          \clip (-.5,.7) rectangle (3,-2);
          \node[draw,fill=\mycyan, circle,inner sep=2pt] (a) at (0,0) {};
          \node[above] (A) at (0,0) {$v_1$};
          \node[draw,fill=blue, circle, inner sep=2pt] (b) at (0:1.5cm) {};
          \node[draw,very thick,fill=blue, circle, inner sep=2pt] (c) at (320:1.5cm) {};
          \node[draw,fill=\myorange, circle, inner sep=2pt] (d) at (340:1.5cm) {};
          
          \node (e) at (335:2.2cm) {};
          \node (f) at (345:2.2cm) {};

          \node (h) at (315:2.2cm) {};
          \node (g) at (320:2.22cm) {};
          \node (i) at (325:2.18cm) {};

          \node (j) at (0:2.2cm) {};

          \node[above] (G) at (2.6, -1.2) {$G'$};
          \draw[domain=120:180] plot ({2*(cos(\x)+1.3)}, {2*(sin(\x)-.6)});

          \draw (j) -- (b) -- (a) -- (c);
          \draw[very thick] (h) -- (c) -- (g);
          \draw (a) -- (d);
          \draw (f) -- (d) -- (e);
          \draw (i) -- (c);
        \end{tikzpicture}}
    \end{column}
  \end{columns}

  \vspace{.4cm}

%%   \uncover<4>{
%%   \begin{block}{Natural question}
%%    What is the diameter of the reconfiguration graph in this setting ?
%%   \end{block}
%%   }
\end{frame}

%% \begin{frame}{Bounded treewidth}
%%   \begin{block}{Treewidth}
%%     Graph parameter that measures how close a graph is from being a tree
%%   \end{block}

%%   $\tw(G) \le k$ implies $G$ $k$-degenerate
%%   \begin{block}{Bonamy, Delecroix, L. '21} If $G$ has treewidth $\tw$ then for all $k \ge \tw+1$,
%%     $\diam(R^k(G)) = O(\tw n^2)$
%%   \end{block}
%% \end{frame}

\section{Recoloring with $\Delta$ colors}
\begin{frame}{Recoloring with $\Delta$ colors}
  \begin{block}{Brook's theorem}
    All graphs but odd cycles and cliques are $\Delta$-colorable
  \end{block}

  \begin{alertblock}{Mohar conjecture '07}
    For all $G$, for all $k \ge \Delta$, $R^k(G)$ is connected\\
    True if $G$ is not regular
  \end{alertblock}
  \begin{description}
  \item[Feghali, Johnson, Paulusma '17] True for all cubic graphs but the 3-prism
  \item[Bonamy, Bousquet, Feghali, Johnson '19] True for $\Delta \ge 4$
  \end{description}

  \begin{columns}
    \begin{column}{.48\textwidth}
      \centering      
      \begin{tikzpicture}
        \node[draw=black,fill=\myorange,circle,inner sep=1.5pt]   (a) at (90:.3cm) {};
        \node[draw=black,fill=\mycyan,circle,inner sep=1.5pt]  (b) at (210:.3cm) {};
        \node[draw=black,fill=magenta,circle,inner sep=1.5pt] (c) at (330:.3cm) {};

        \node[draw=black,fill=magenta,circle,inner sep=1.5pt]   (A) at (90:1cm) {};
        \node[draw=black,fill=\myorange,circle,inner sep=1.5pt]  (B) at (210:1cm) {};
        \node[draw=black,fill=\mycyan,circle,inner sep=1.5pt] (C) at (330:1cm) {};

        \draw (a) -- (A);
        \draw (b) -- (B);
        \draw (c) -- (C);
        \draw (a) -- (b) -- (c) -- (a);
        \draw (A) -- (B) -- (C) -- (A);
      \end{tikzpicture}
    \end{column}
    \hfill
    \begin{column}{.48\textwidth}
      \centering
      \begin{tikzpicture}
        \node[draw=black,fill=\myorange,circle,inner sep=1.5pt]   (a) at (90:.3cm) {};
        \node[draw=black,fill=\mycyan,circle,inner sep=1.5pt]  (b) at (210:.3cm) {};
        \node[draw=black,fill=magenta,circle,inner sep=1.5pt] (c) at (330:.3cm) {};

        \node[draw=black,fill=\mycyan,circle,inner sep=1.5pt]   (A) at (90:1cm) {};
        \node[draw=black,fill=magenta,circle,inner sep=1.5pt]  (B) at (210:1cm) {};
        \node[draw=black,fill=\myorange,circle,inner sep=1.5pt] (C) at (330:1cm) {};

        \draw (a) -- (A);
        \draw (b) -- (B);
        \draw (c) -- (C);
        \draw (a) -- (b) -- (c) -- (a);
        \draw (A) -- (B) -- (C) -- (A);
      \end{tikzpicture}
    \end{column}
  \end{columns}

  \begin{block}{Bonamy, Delecroix, L. '21+} For all graph $G$ but the 3-prism, for $k \ge \Delta(G)$,
    $\diam(R^k(G)) = O(\Delta n^2)$
  \end{block}

\end{frame}

\begin{frame}{Recoloring with $\Delta$ colors}
  \begin{block}{Bonamy, Delecroix, L. '21+} For all graph $G$ but the 3-prism, for $k \ge \Delta(G)$,
    $$\diam(R^k(G)) = O(\Delta n^2)$$
  \end{block}

%%   \begin{block}{Sketch of the proof}
%%     \begin{itemize}
%%     \item Reach a coloring where two fixed distant vertices are colored alike
%%     \item Identify them to decrease the degeneracy of the graph
%%     \end{itemize}
%%   \end{block}

  \vspace{.5cm}
  \begin{columns}
    \begin{column}{.48\textwidth}
      \centering      
      \begin{tikzpicture}
        \node[draw=black,fill=\myorange,circle,inner sep=2pt]   (A) at (180:1.5cm) {};
        \node[left] (u) at (180:1.5cm) {$u$};
        \node[draw=black,fill=magenta,circle,inner sep=2pt]  (B) at (120:1.5cm) {};
        \node[draw=black,fill=\myorange,circle,inner sep=2pt] (C) at (60:1.5cm) {};
        \node[draw=black,fill=\mycyan,circle,inner sep=2pt]   (D) at (0:1.5cm) {};
        \node[draw=black,fill=\myorange,circle,inner sep=2pt]  (E) at (300:1.5cm) {};
        \node[draw=black,fill=blue,circle,inner sep=2pt]  (F) at (240:1.5cm) {};

        \node[draw=black,fill=\mycyan,circle,inner sep=2pt]   (1) at (135:.7cm) {};
        \node[right]  (v) at (135:.7cm) {$v$};
        \node[draw=black,fill=\mycyan,circle,inner sep=2pt]  (4) at (225:.7cm) {};
        \node[right]  (w) at (225:.7cm) {$w$};
        \node[draw=black,fill=magenta,circle,inner sep=2pt] (3) at (315:.7cm) {};
        \node[draw=black,fill=blue,circle,inner sep=2pt]   (2) at (45:.7cm) {};
        \draw (A) -- (B) -- (C) -- (D) -- (E) -- (F) -- (A) -- (1) -- (C) -- (2)
        -- (B) -- (1) -- (3) -- (D) -- (2) -- (4) -- (E) -- (3) -- (F) -- (4) -- (A);
      \end{tikzpicture}

      \centering $G$
      
    \end{column}
    \hfill
    \begin{column}{.48\textwidth}
      \centering
      \begin{tikzpicture}
        \node[draw=black,fill=\myorange,circle,inner sep=2pt]   (A) at (180:1.5cm) {};
        \node[left] (u) at (180:1.5cm) {$u$};
        \node[draw=black,fill=magenta,circle,inner sep=2pt]  (B) at (120:1.5cm) {};
        \node[draw=black,fill=\myorange,circle,inner sep=2pt] (C) at (60:1.5cm) {};
        \node[draw=black,fill=\mycyan,circle,inner sep=2pt]   (D) at (0:1.5cm) {};
        \node[draw=black,fill=\myorange,circle,inner sep=2pt]  (E) at (300:1.5cm) {};
        \node[draw=black,fill=blue,circle,inner sep=2pt]  (F) at (240:1.5cm) {};

        \node[draw=black,fill=\mycyan,circle,inner sep=2pt]   (1) at (180:.7cm) {};
        \node[right]  (v) at (180:.7cm) {${v+w}$};
%        \node[draw=black,fill=\mycyan,circle,inner sep=2pt]  (4) at (225:.7cm) {};
        \node[draw=black,fill=magenta,circle,inner sep=2pt] (3) at (315:.7cm) {};
        \node[draw=black,fill=blue,circle,inner sep=2pt]   (2) at (45:.7cm) {};

        \draw (A) -- (B) -- (C) -- (D) -- (E) -- (F) -- (A) -- (1) -- (C) -- (2)
        -- (B) -- (1) -- (3) -- (D) -- (2) -- (1) -- (E) -- (3) -- (F) -- (1) -- (A);
      \end{tikzpicture}
      
      \centering $G_{v+w}$
    \end{column}
  \end{columns}
  \vspace{.5cm}
  $$V(R^k(G)) \simeq \bigcup_{(v,w) \in N(u)^2\setminus E} V(R^k(G_{v+w}))$$          
\end{frame}

\begin{frame}{Sketch of proof of the lemma}
  \begin{block}{Key lemma}
    If $G'$ is $(d-1)$-degenerate, with $\deg(v_i) \le d$ for all $i <n$, then
    $\diam(R^k(G')) = O(dn^2)$  
  \end{block}

  \begin{block}{Induction}
    If $c \notin \alpha(N^+(u))$, then there exits $\beta$ such that:
    \begin{itemize}
    \item $\forall v \succ u$, $\alpha(u) = \beta(u)$
    \item $\beta(u) = c$
    \item $\alpha \mathop{\leadsto}_{K} \beta$ by recoloring each vertex $w
      \prec u$ at most $p$ times where $p = |\alpha^{-1}(c) \cup N^-(u)|$ 
    \end{itemize}
  \end{block}
%%   \begin{block}{Key lemma}
%%     If $G'$ is $(d-1)$-degenerate, with $\deg(v_i) \le d$ for all $i <n$, then
%%     $\diam(R^k(G')) = O(dn^2)$  
%%   \end{block}
  \vspace{-.3cm}
  \only<1>{
    \centering
    \begin{tikzpicture}
      \clip (-9.5, -1.5) rectangle (.5,1.5);
      \node[draw=black,fill=\mycyan,circle,inner sep=2pt] (A) {};
      \node[draw=black,fill=\myorange,circle,inner sep=2pt] (B) [left of=A] {};
      \node[draw=black,fill=blue,circle,inner sep=2pt] (C) [left of=B] {};
      \node[draw=black,fill=\myorange,circle,inner sep=2pt] (D) [left of=C] {};
      \node[draw=black,fill=\mycyan,circle,inner sep=2pt] (E) [left of=D] {};
      \node[draw=black,fill=magenta,circle,inner sep=2pt] (F) [left of=E] {};
      \node[draw=black,fill=\myorange,circle,inner sep=2pt] (G) [left of=F] {};
      \node[draw=black,fill=\mycyan,circle,inner sep=2pt] (H) [left of=G] {};
      \node[draw=black,fill=magenta,circle,inner sep=2pt] (I) [left of=H] {};
      \node[draw=black,fill=\mycyan,circle,inner sep=2pt] (J) [left of=I] {};

      \node[draw=black,fill=\mycyan,circle,inner sep=2pt] (a) [below of=A] {};
      \node[draw=black,fill=\myorange,circle,inner sep=2pt] (b) [below of=B] {};
      \node[draw=black,fill=blue,circle,inner sep=2pt] (c) [below of=C] {};
      \node[draw=black,fill=\mycyan,circle,inner sep=2pt] (d) [below of=D] {};

      \draw (A) -- (B) -- (C) -- (D) -- (E) -- (F) -- (G) -- (H) -- (I) -- (J);

      \path[every node/.style={font=\sffamily\small}]
      (E) edge[bend left] node {} (B)
      (E) edge[bend left] node {} (C)
      (H) edge[bend left] node {} (D)
      (J) edge[bend left] node {} (F)
      (I) edge[bend left] node {} (A)
      (G) edge[bend left] node {} (A);
  \end{tikzpicture}}

  \only<2>{
  \centering
  \begin{tikzpicture}
    \clip (-9.5, -1.5) rectangle (.5,1.5);
    \node[draw=black,very thick,fill=\mycyan,circle,inner sep=2pt] (A) {};
    \node[draw=black,very thick,fill=\myorange,circle,inner sep=2pt] (B) [left of=A] {};
    \node[draw=black,fill=blue,circle,inner sep=2pt] (C) [left of=B] {};
    \node[draw=black,very thick,fill=\myorange,circle,inner sep=2pt] (D) [left of=C] {};
    \node[draw=black,very thick,fill=\mycyan,circle,inner sep=2pt] (E) [left of=D] {};
    \node[draw=black,fill=magenta,circle,inner sep=2pt] (F) [left of=E] {};
    \node[draw=black,very thick,fill=\myorange,circle,inner sep=2pt] (G) [left of=F] {};
    \node[draw=black,very thick,fill=\mycyan,circle,inner sep=2pt] (H) [left of=G] {};
    \node[draw=black,fill=magenta,circle,inner sep=2pt] (I) [left of=H] {};
    \node[draw=black,fill=\mycyan,circle,inner sep=2pt] (J) [left of=I] {};

    \node[draw=black,fill=\mycyan,circle,inner sep=2pt] (a) [below of=A] {};
    \node[draw=black,fill=\myorange,circle,inner sep=2pt] (b) [below of=B] {};
    \node[draw=black,fill=blue,circle,inner sep=2pt] (c) [below of=C] {};
    \node[draw=black,fill=\mycyan,circle,inner sep=2pt] (d) [below of=D] {};

    \draw[very thick] (A) --(B) (D) -- (E) (G) -- (H);
    \draw (B) -- (C) -- (D)  (E) -- (F) -- (G) (H) -- (I) -- (J);

    \path[every node/.style={font=\sffamily\small}]
    (E) edge[bend left,very thick] node {} (B)  
    (E) edge[bend left] node {} (C)
    (H) edge[bend left, very thick] node {} (D)
    (J) edge[bend left] node {} (F)
    (I) edge[bend left] node {} (A)
    (G) edge[bend left, very thick] node {} (A);      
  \end{tikzpicture}}
  
  \only<3>{
    \centering
    \begin{tikzpicture}
      \clip (-9.5, -1.5) rectangle (.5,1.5);
      \node[draw=black,fill=\mycyan,circle,inner sep=2pt] (A) {};
      \node[draw=black,fill=\myorange,circle,inner sep=2pt] (B) [left of=A] {};
      \node[draw=black,fill=blue,circle,inner sep=2pt] (C) [left of=B] {};
      \node[draw=black,fill=\myorange,circle,inner sep=2pt] (D) [left of=C] {};
      \node[draw=black,very thick,fill=\mycyan,circle,inner sep=2pt] (E) [left of=D] {};
      \node[draw=black,fill=magenta,circle,inner sep=2pt] (F) [left of=E] {};
      \node[draw=black,fill=\myorange,circle,inner sep=2pt] (G) [left of=F] {};
      \node[draw=black,fill=\mycyan,circle,inner sep=2pt] (H) [left of=G] {};
      \node[draw=black,fill=magenta,circle,inner sep=2pt] (I) [left of=H] {};
      \node[draw=black,fill=\mycyan,circle,inner sep=2pt] (J) [left of=I] {};

      \node[draw=black,fill=\mycyan,circle,inner sep=2pt] (a) [below of=A] {};
      \node[draw=black,fill=\myorange,circle,inner sep=2pt] (b) [below of=B] {};
      \node[draw=black,fill=blue,circle,inner sep=2pt] (c) [below of=C] {};
      \node[draw=black,fill=\mycyan,circle,inner sep=2pt] (d) [below of=D] {};

      \draw (A) -- (B) -- (C) -- (D) -- (E) -- (F) -- (G) -- (H) -- (I) -- (J);

      \path[every node/.style={font=\sffamily\small}]
      (E) edge[bend left] node {} (B)
      (E) edge[bend left] node {} (C)
      (H) edge[bend left] node {} (D)
      (J) edge[bend left] node {} (F)
      (I) edge[bend left] node {} (A)
      (G) edge[bend left] node {} (A);
  \end{tikzpicture}}

    \only<4>{
    \centering
    \begin{tikzpicture}
      \clip (-9.5, -1.5) rectangle (.5,1.5);
      \node[draw=black,fill=\mycyan,circle,inner sep=2pt] (A) {};
      \node[draw=black,fill=\myorange,circle,inner sep=2pt] (B) [left of=A] {};
      \node[draw=black,fill=blue,circle,inner sep=2pt] (C) [left of=B] {};
      \node[draw=black,fill=\myorange,circle,inner sep=2pt] (D) [left of=C] {};
      \node[draw=black,fill=magenta,circle,inner sep=2pt] (E) [left of=D] {};
      \node[draw=black,fill=\mycyan,circle,inner sep=2pt] (F) [left of=E] {};
      \node[draw=black,fill=\myorange,circle,inner sep=2pt] (G) [left of=F] {};
      \node[draw=black,very thick,fill=\mycyan,circle,inner sep=2pt] (H) [left of=G] {};
      \node[draw=black,fill=magenta,circle,inner sep=2pt] (I) [left of=H] {};
      \node[draw=black,fill=\myorange,circle,inner sep=2pt] (J) [left of=I] {};

      \node[draw=black,fill=\mycyan,circle,inner sep=2pt] (a) [below of=A] {};
      \node[draw=black,fill=\myorange,circle,inner sep=2pt] (b) [below of=B] {};
      \node[draw=black,fill=blue,circle,inner sep=2pt] (c) [below of=C] {};
      \node[draw=black,fill=\mycyan,circle,inner sep=2pt] (d) [below of=D] {};

      \draw (A) -- (B) -- (C) -- (D) -- (E) -- (F) -- (G) -- (H) -- (I) -- (J);

      \path[every node/.style={font=\sffamily\small}]
      (E) edge[bend left] node {} (B)
      (E) edge[bend left] node {} (C)
      (H) edge[bend left] node {} (D)
      (J) edge[bend left] node {} (F)
      (I) edge[bend left] node {} (A)
      (G) edge[bend left] node {} (A);
  \end{tikzpicture}}

    \only<5>{
    \centering
    \begin{tikzpicture}
      \clip (-9.5, -1.5) rectangle (.5,1.5);
      \node[draw=black,fill=\mycyan,circle,inner sep=2pt] (A) {};
      \node[draw=black,fill=\myorange,circle,inner sep=2pt] (B) [left of=A] {};
      \node[draw=black,fill=blue,circle,inner sep=2pt] (C) [left of=B] {};
      \node[draw=black,fill=\myorange,circle,inner sep=2pt] (D) [left of=C] {};
      \node[draw=black,fill=magenta,circle,inner sep=2pt] (E) [left of=D] {};
      \node[draw=black,fill=\mycyan,circle,inner sep=2pt] (F) [left of=E] {};
      \node[draw=black,fill=\myorange,circle,inner sep=2pt] (G) [left of=F] {};
      \node[draw=black,fill=blue,circle,inner sep=2pt] (H) [left of=G] {};
      \node[draw=black,fill=magenta,circle,inner sep=2pt] (I) [left of=H] {};
      \node[draw=black,fill=\myorange,circle,inner sep=2pt] (J) [left of=I] {};

      \node[draw=black,fill=\mycyan,circle,inner sep=2pt] (a) [below of=A] {};
      \node[draw=black,fill=\myorange,circle,inner sep=2pt] (b) [below of=B] {};
      \node[draw=black,fill=blue,circle,inner sep=2pt] (c) [below of=C] {};
      \node[draw=black,fill=\mycyan,circle,inner sep=2pt] (d) [below of=D] {};

      \draw (A) -- (B) -- (C) -- (D) -- (E) -- (F) -- (G) -- (H) -- (I) -- (J);

      \path[every node/.style={font=\sffamily\small}]
      (E) edge[bend left] node {} (B)
      (E) edge[bend left] node {} (C)
      (H) edge[bend left] node {} (D)
      (J) edge[bend left] node {} (F)
      (I) edge[bend left] node {} (A)
      (G) edge[bend left] node {} (A);
  \end{tikzpicture}}  
%%   \uncover<2>{
%%   \begin{block}{Open questions}
%%     \begin{itemize}
%%     \item What is the mixing time of the associated Markov chain ?
%%     \item If $G$ is $(d-1)$-degenerate, estimate $\diam(R^k(G))$ for $k \ge
%%       d$.
%%     \end{itemize}
%%   \end{block}}
\end{frame}

\begin{frame}{Avoiding the $\Delta^2$ factor: first case}
  \begin{block}{$G$ is 3-connected of diameter more than three}
  Let $u$ and $x$ be two vertices far away in $G$, let $(v,w)\in N(u)^2 \setminus E$\\
  For all $(y,z) \in N(x)^2 \setminus E$, $\exists \alpha$ such that $\alpha(v) =
  \alpha(w)$ and $\alpha(y) = \alpha(z)$
  \end{block}

  \vspace{7cm}
\end{frame}

\begin{frame}{Avoiding the $\Delta^2$ factor: second case}
  \begin{block}{$G$ is not 3-connected}
    If $G_1$ and $G_2$ are $d-1$ degenerate, of degree a most $d$ and $S = G_1
    \cup G_2$ is a complete graph, $\diam(R^k(G_1 \cap G_2)) = O(dn^2)$ for $k
    \ge d$
  \end{block}

  \vspace{7cm}
\end{frame}

\end{document}
