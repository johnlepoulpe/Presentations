\newif\ifinria
\def\ptitle{Recoloring graphs with Kempe changes} % Title
\def\pauthor{Clément Legrand-Duchesne} % Author
\def\padvisors{Marthe Bonamy, Vincent Delecroix} % Advisors
\def\pteam{Graphe et Optimisation, Combinatoire et Interactions} % Team
\def\pinstitute{Univ. Bordeaux, LaBRI, France} % Affiliation
\def\pdate{Thursday April 6, 2023} % Date
\inriafalse % inriatrue/inriafalse to enable/disable Inria logo


\input{boilerplateSDSLaBRI}

\graphicspath{{./fig/}} % Figures and logos directory
\setlength{\columnheight}{588ex} % Tweak this value if columns are too long/short (should be okay with 588ex)

\usepackage{BeamerColor}
\usepackage{tikz}

%%%%%%%%% Colors of Beamer Layout %%%%%%%%%%%%%
\begin{document}
\begin{frame}
\begin{columns}
\begin{column}{.49\textwidth}
\begin{beamercolorbox}[center,wd=\textwidth]{postercolumn}
\begin{minipage}[T]{.95\textwidth}
\parbox[t][\columnheight]{\textwidth}{

\begin{block}{Graph coloring}
\begin{columns}
\begin{column}{.49\textwidth}

  \centering
  \pgfdeclarelayer{background}
  \pgfdeclarelayer{foreground}
  \pgfsetlayers{background,main,foreground}
  \begin{tikzpicture}

  \begin{pgfonlayer}{background} 
  \clip (-5,5) rectangle (5,-5);
  \draw[fill=yellow]  (0,0) -- (126:10cm) -- (198:10cm) -- cycle; 
  \draw[fill=rougeLaBRI]  (0,0) -- (198:10cm) -- (270:10cm) -- cycle;
  \draw[fill=bleuLaBRI]  (0,0) -- (270:10cm) -- (342:10cm) -- cycle;
  \draw[fill=yellow]  (0,0) -- (342:10cm) -- (54:10cm) -- cycle;
  \draw[fill=orange]  (0,0) -- (54:10cm) -- (126:10cm) -- cycle;
  \end{pgfonlayer}

  \begin{scope}[shift={(0,.75)}]
  \draw[fill=bleuLaBRI] (0,0) -- (45:3cm) -- (135:3cm) -- (0,0); 
  \draw[fill=orange] (0,0) -- (135:3cm) -- (225:3cm)-- (-90:2.12cm) -- (0,0); 
  \draw[fill=rougeLaBRI] (0,0) -- (-90:2.12cm) -- (315:3cm) -- (45:3cm) -- (0,0); 
  \end{scope}
  
  \begin{pgfonlayer}{foreground}
  \draw[fill=yellow] (0,-1.37) circle (1cm);
  \end{pgfonlayer}
  \end{tikzpicture}
  \end{column}
  \begin{column}{.49\textwidth}

  \centering
  \begin{tikzpicture}    
  \clip (-5,5) rectangle (5,-5);
  \node[draw=black,fill=orange,circle,inner sep=6pt]   (a) at (90:4cm) {};
  \node[draw=black,fill=yellow,circle,inner sep=6pt]  (b) at (162:4cm) {};
  \node[draw=black,fill=rougeLaBRI,circle,inner sep=6pt] (c) at (234:4cm) {};
  \node[draw=black,fill=bleuLaBRI,circle,inner sep=6pt] (d) at (306:4cm) {};
  \node[draw=black,fill=yellow,circle,inner sep=6pt] (e) at (18:4cm) {};

  \node[draw=black,fill=bleuLaBRI,circle,inner sep=6pt]   (f) at (90:1.7cm) {};
  \node[draw=black,fill=orange,circle,inner sep=6pt]  (g) at (180:1.7cm) {};
  \node[draw=black,fill=yellow,circle,inner sep=6pt] (h) at (270:1.7cm) {};
  \node[draw=black,fill=rougeLaBRI,circle,inner sep=6pt] (i) at (0:1.7cm) {};

  \draw (a) -- (b) -- (c) -- (d) -- (e) -- (a);
  \draw (f) -- (a)  (g) -- (b)  (i) -- (e);
  \draw (g) -- (i) -- (d) -- (h) -- (c) -- (g) -- (f) -- (i) -- (h) -- (g);
  \end{tikzpicture}
  \end{column}
  \end{columns}
  \vspace{1cm}
  [Apple, Haken 1989] Every planar map (or graph) is 4-colorable
\end{block}
            
\vfill

\begin{block}{Kempe change}            
  \centering
  \begin{columns}
  \begin{column}{.49\textwidth}
  \centering
  \begin{tikzpicture}    
  \clip (-5,5) rectangle (5,-5);
  \node[draw=black,fill=orange,circle,inner sep=6pt]   (a) at (90:4cm) {};
  \node[draw=black,fill=yellow,circle,inner sep=6pt]  (b) at (162:4cm) {};
  \node[draw=black,fill=rougeLaBRI,circle,inner sep=6pt] (c) at (234:4cm) {};
  \node[draw=black,fill=bleuLaBRI,circle,inner sep=6pt] (d) at (306:4cm) {};
  \node[draw=black,fill=yellow,circle,inner sep=6pt] (e) at (18:4cm) {};

  \node[draw=black,line width = 4pt,fill=yellow,circle,inner sep=6pt]   (f) at (90:1.7cm) {};
  \node[draw=black,fill=orange,circle,inner sep=6pt]  (g) at (180:1.7cm) {};
  \node[draw=black,fill=yellow,circle,inner sep=6pt] (h) at (270:1.7cm) {};
  \node[draw=black,fill=rougeLaBRI,circle,inner sep=6pt] (i) at (0:1.7cm) {};

  \draw (a) -- (b) -- (c) -- (d) -- (e) -- (a);
  \draw (f) -- (a)  (g) -- (b)  (i) -- (e);
  \draw (g) -- (i) -- (d) -- (h) -- (c) -- (g) -- (f) -- (i) -- (h) -- (g);
  \end{tikzpicture}
  \end{column}
  
  \begin{column}{.49\textwidth}
  \centering
  \begin{tikzpicture}    
  \clip (-5,5) rectangle (5,-5);
  \node[draw=black,fill=orange,circle,inner sep=6pt]   (a) at (90:4cm) {};
  \node[draw=black,fill=yellow,circle,inner sep=6pt]  (b) at (162:4cm) {};
  \node[draw=black,line width = 4pt, fill=orange,circle,inner sep=6pt] (c) at (234:4cm) {};
  \node[draw=black,fill=bleuLaBRI,circle,inner sep=6pt] (d) at (306:4cm) {};
  \node[draw=black,fill=yellow,circle,inner sep=6pt] (e) at (18:4cm) {};

  \node[draw=black,fill=bleuLaBRI,circle,inner sep=6pt]   (f) at (90:1.7cm) {};
  \node[draw=black,line width = 4pt, fill=rougeLaBRI,circle,inner sep=6pt]  (g) at (180:1.7cm) {};
  \node[draw=black,fill=yellow,circle,inner sep=6pt] (h) at (270:1.7cm) {};
  \node[draw=black,line width = 4pt, fill=orange,circle,inner sep=6pt] (i) at (0:1.7cm) {};

  \draw (a) -- (b) -- (c) -- (d) -- (e) -- (a);
  \draw (f) -- (a)  (g) -- (b)  (i) -- (e);
  \draw  (i) -- (d) -- (h) -- (c)  (g) -- (f) -- (i) -- (h) -- (g);
  \draw[line width = 4pt] (c) -- (g) -- (i);
  \end{tikzpicture}
  \end{column}
  \end{columns}
  \vspace{1cm}
  Swap the colors in a maximal bichromatic connected subgraph
            
\end{block}
            
\vfill
            
\begin{block}{Objectives and questions}
            
\centering
\begin{itemize}
\item Color a graph  with as few colors as possible
\item Study of the reconfiguration graph\\ {\small Vertices: $k$-colorings of a
  graph $G$, edge if Kempe change between two colorings}
\begin{itemize}
\item Connectivity 
\item Diameter
\item Complexity of reachability problem
\end{itemize}
\item Sampling colorings
\end{itemize}
\end{block}
            
            \vfill

\begin{block}{Color a graph with as few colors as possible}
Every planar graph is 5-colorable

\centering
\begin{columns}
\begin{column}{.33\textwidth}
\centering
\begin{tikzpicture}
\clip (-4,4) rectangle (4,-4.5);
    \node[draw=black,fill=orange,circle,inner sep=6pt]   (a) at (90:3cm) {};
          \node[draw=black,fill=rougeLaBRI,circle,inner sep=6pt]  (b) at (162:3cm) {};
          \node[draw=black,fill=yellow,circle,inner sep=6pt] (c) at (234:3cm) {};
          \node[draw=black,fill=bleuLaBRI,circle,inner sep=6pt] (d) at (306:3cm) {};
          \node[draw=black,fill=black,circle,inner sep=6pt] (e) at (18:3cm) {};
          \node[draw=black,fill=white,circle,inner sep=6pt] (A) at (0,0) {};
          
          
          \draw (a) -- (b) -- (c) -- (d) -- (e) -- (a);
          \draw (b) -- (A) -- (a) (c) -- (A) -- (d) (A) --(e);
\end{tikzpicture}
\end{column}
$\rightarrow$
\begin{column}{.33\textwidth}
\centering
\begin{tikzpicture}
\clip (-4,4) rectangle (4,-4.5);
    \node[draw=black,fill=orange,circle,inner sep=6pt]   (a) at (90:3cm) {};
          \node[draw=black,fill=rougeLaBRI,circle,inner sep=6pt]  (b) at (162:3cm) {};
          \node[draw=black,fill=yellow,circle,inner sep=6pt] (c) at (234:3cm) {};
          \node[draw=black,fill=bleuLaBRI,circle,inner sep=6pt] (d) at (306:3cm) {};
          \node[draw=black,line width=4pt,fill=yellow,circle,inner sep=6pt] (e) at (18:3cm) {};
          \node[draw=black,fill=black,circle,inner sep=6pt] (A) at (0,0) {};
          
          
          \draw (a) -- (b) -- (c) -- (d) -- (e) -- (a);
          \draw (b) -- (A) -- (a) (c) -- (A) -- (d) (A) --(e);
\end{tikzpicture}
\end{column}
or
\begin{column}{.33\textwidth}
\centering
\begin{tikzpicture}
\clip (-4,4) rectangle (4,-4.5);
    \node[draw=black,fill=orange,circle,inner sep=6pt]   (a) at (90:3cm) {};
          \node[draw=black,line width = 4pt,fill=bleuLaBRI,circle,inner sep=6pt]  (b) at (162:3cm) {};
          \node[draw=black,fill=yellow,circle,inner sep=6pt] (c) at (234:3cm) {};
          \node[draw=black,fill=bleuLaBRI,circle,inner sep=6pt] (d) at (306:3cm) {};
          \node[draw=black,fill=black,circle,inner sep=6pt] (e) at (18:3cm) {};
          \node[draw=black,fill=rougeLaBRI,circle,inner sep=6pt] (A) at (0,0) {};
          \node (D1) at (286:6cm) {};
          \node (D2) at (326:6cm) {};
          
          \draw (a) -- (b) -- (c) -- (d) -- (e) -- (a);
          \draw (b) -- (A) -- (a) (c) -- (A) -- (d) (A) --(e);
          \draw[dashed] (c) .. controls (D1) and (D2) .. (e);
\end{tikzpicture}
\end{column}
\end{columns}

\end{block}
            
            \vfill
            
\begin{block}{Connectivity of the reconfiguration graph}
[Meyniel 1989] For all planar graph $G$, all the 5-colorings of $G$ are Kempe
equivalent

\vspace{.5cm}
False for 4-colorings because of frozen colorings :

\centering
 \begin{columns}
    \begin{column}{.49\textwidth}
    \centering
      \begin{tikzpicture}
        \node[draw=black,fill=orange,circle,inner sep=6pt]   (a) at (45:3cm) {};
        \node[draw=black,fill=bleuLaBRI,circle,inner sep=6pt]  (b) at (135:3cm) {};
        \node[draw=black,fill=yellow,circle,inner sep=6pt] (c) at (225:3cm) {};
        \node[draw=black,fill=rougeLaBRI,circle,inner sep=6pt] (d) at (315:3cm) {};
        \node[draw=black,fill=rougeLaBRI,circle,inner sep=6pt] (B) at (135:1.5cm) {};
        \node[draw=black,fill=bleuLaBRI,circle,inner sep=6pt] (D) at (315:1.5cm) {};
        \node[draw=black,fill=yellow,circle,inner sep=6pt] (A) at (45:5.4cm) {};
        \node[draw=black,fill=orange,circle,inner sep=6pt] (C) at (225:5.4cm) {};
        \node (D') at (315:6cm) {};
        
        \draw (a) -- (b) -- (c) -- (d) -- (a) -- (A) -- (b) -- (B) -- (D) -- (d)
        -- (A);
        \draw (c) -- (B) -- (a) -- (D) -- (c) -- (C)  (b) -- (C) -- (d) ;
        \draw (A) .. controls (D') ..  (C);
      \end{tikzpicture}
    \end{column}
    \hfill
    
    \begin{column}{.49\textwidth}
    \centering
      \begin{tikzpicture}
        \node[draw=black,fill=orange,circle,inner sep=6pt]   (a) at (45:3cm) {};
        \node[draw=black,fill=bleuLaBRI,circle,inner sep=6pt]  (b) at (135:3cm) {};
        \node[draw=black,fill=orange,circle,inner sep=6pt] (c) at (225:3cm) {};
        \node[draw=black,fill=bleuLaBRI,circle,inner sep=6pt] (d) at (315:3cm) {};
        \node[draw=black,fill=rougeLaBRI,circle,inner sep=6pt] (B) at (135:1.5cm) {};
        \node[draw=black,fill=yellow,circle,inner sep=6pt] (D) at (315:1.5cm) {};
        \node[draw=black,fill=yellow,circle,inner sep=6pt] (A) at (45:5.4cm) {};
        \node[draw=black,fill=rougeLaBRI,circle,inner sep=6pt] (C) at (225:5.4cm) {};
        \node (D') at (315:6cm) {};
        
        \draw (a) -- (b) -- (c) -- (d) -- (a) -- (A) -- (b) -- (B) -- (D) -- (d)
        -- (A);
        \draw (c) -- (B) -- (a) -- (D) -- (c) -- (C)  (b) -- (C) -- (d) ;
        \draw (A) .. controls (D') ..  (C);
      \end{tikzpicture}
    \end{column}
  \end{columns}
\end{block}
         
          }
        \end{minipage}
      \end{beamercolorbox}
    \end{column}
    \begin{column}{.49\textwidth}
      \begin{beamercolorbox}[center,wd=\textwidth]{postercolumn}
        \begin{minipage}[T]{.95\textwidth}
          \parbox[t][\columnheight]{\textwidth}{
            
            \begin{block}{Diameter of the reconfiguration graph}
            [Deschamps, Feghali, Kardoš, L-D., Pierron 22] All the 5-colorings
            of a planar graph are equivalent up to $O(n^{195})$ Kempe changes 
            \end{block}
            
            
          }
        \end{minipage}
      \end{beamercolorbox}
    \end{column}
  \end{columns}
  \vskip1ex
\end{frame}
\end{document}
