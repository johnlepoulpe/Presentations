\newif\ifinria
\def\ptitle{Recoloring graphs with Kempe changes} % Title
\def\pauthor{Clément Legrand-Duchesne} % Author
\def\padvisors{Marthe Bonamy, Vincent Delecroix} % Advisors
\def\pteam{Graphe et Optimisation, Combinatoire et Interactions} % Team
\def\pinstitute{Univ. Bordeaux, LaBRI, France} % Affiliation
\def\pdate{Thursday April 6, 2023} % Date
\inriafalse % inriatrue/inriafalse to enable/disable Inria logo


\pdfobjcompresslevel=0
\documentclass[final,hyperref={pdfpagelabels=false},xcolor=dvipsnames]{beamer}
\usepackage[orientation=portrait,size=a0,scale=1.4]{beamerposter}
\usepackage[utf8]{inputenc}
\usepackage[sfdefault]{roboto}
\usepackage[english]{babel}
\usepackage{amsmath, amsthm, amssymb, array, booktabs, grffile, latexsym, tabularx, xspace}
\newcolumntype{Z}{>{\centering\arraybackslash}X}
\newcommand{\pphantom}{\textcolor{ta3aluminium}}
\newlength{\columnheight}

\def\plogo{logo/logo_LaBRI.png}
\def\purl{https://www.labri.fr/}
\def\pmail{Twitter: @labriOfficial}

\mode<presentation>{\usetheme{SDSLaBRI}}
\title[\ptitle]{\texorpdfstring{\huge \ptitle}{\ptitle}}
\author[\pauthor]{\pauthor\ -\ \padvisors}
\institute[\pinstitute]{\pteam\ -\ \pinstitute}
\date[\pdate]{\pdate}
\setlogo{\plogo}
\setauthorurl{\purl}
\setauthoremail{\pmail}


\graphicspath{{./fig/}} % Figures and logos directory
\setlength{\columnheight}{588ex} % Tweak this value if columns are too long/short (should be okay with 588ex)

\usepackage{BeamerColor}
\usepackage{tikz}
\usetikzlibrary{arrows.meta}

\newcommand{\thm}[1]{\textcolor{bleuLaBRI}{\textbf{[#1]\xspace}}}
\newcommand{\conj}{\textcolor{rougeLaBRI}{\textbf{Conjecture:\xspace }}}

\usepackage{ellipsis}
\usetikzlibrary{calc}
\usetikzlibrary{decorations.pathreplacing,decorations.markings,shapes.geometric}
\tikzset{naming/.style={align=center,font=\small}}
\tikzset{antenna/.style={insert path={-- coordinate (ant#1) ++(0,0.25) -- +(135:0.25) + (0,0) -- +(45:0.25)}}}
\tikzset{station/.style ={naming,draw,shape=dart,shape border
    rotate=90, minimum width=10mm, minimum height=10mm,outer sep=0pt,inner
    sep=3pt}}
%\tikzset{mobile/.style={naming,draw,shape=rectangle,minimum width=15mm,minimum height=7.5mm, outer sep=0pt,inner sep=3pt}}
\tikzset{mobile/.style={naming,draw,shape=rectangle,minimum width=12mm,minimum height=6mm, outer sep=0pt,inner sep=3pt}}
\tikzset{radiation/.style={{decorate,decoration={expanding
        waves,angle=90,segment length=4pt}}}}

\newcommand{\BS}[1]{%
\begin{tikzpicture}
\node[station] (base) {};

% %\draw[line join=bevel] (base.110) -- (base.70) -- (base.north west) -- (base.north east) -- cycle;
\draw[thick,line join=bevel,draw=#1] (base.100) -- (base.80) -- (base.110) -- (base.70)
-- (base.north west) -- (base.north east) -- (base.180) -- (base.0);
\draw[thick,line join=bevel,draw=#1] (base.100) -- (base.70) (base.110) -- (base.north
east) (base.north west) -- (base.0);

% % original yshift=.8pt
% \draw[line cap=rect,draw=#1] ([yshift=0pt]base.north) [antenna=1];
\end{tikzpicture}
}


%%%%%%%%% Colors of Beamer Layout %%%%%%%%%%%%%
\begin{document}
\begin{frame}
  \begin{columns}
    \begin{column}{.49\textwidth}
      \begin{beamercolorbox}[center,wd=\textwidth]{postercolumn}
        \begin{minipage}[T]{.95\textwidth}

          \begin{block}{Graph coloring}
            \begin{columns}
              \begin{column}{.49\textwidth}

                \centering
                \pgfdeclarelayer{background}
                \pgfdeclarelayer{foreground}
                \pgfsetlayers{background,main,foreground}
                \begin{tikzpicture}

                  \begin{pgfonlayer}{background} 
                    \clip (-5,5) rectangle (5,-5);
                    \draw[fill=yellow]  (0,0) -- (126:10cm) -- (198:10cm) -- cycle; 
                    \draw[fill=rougeLaBRI]  (0,0) -- (198:10cm) -- (270:10cm) -- cycle;
                    \draw[fill=bleuLaBRI]  (0,0) -- (270:10cm) -- (342:10cm) -- cycle;
                    \draw[fill=yellow]  (0,0) -- (342:10cm) -- (54:10cm) -- cycle;
                    \draw[fill=orange]  (0,0) -- (54:10cm) -- (126:10cm) -- cycle;
                  \end{pgfonlayer}

                  \begin{scope}[shift={(0,.75)}]
                    \draw[fill=bleuLaBRI] (0,0) -- (45:3cm) -- (135:3cm) -- (0,0); 
                    \draw[fill=orange] (0,0) -- (135:3cm) -- (225:3cm)-- (-90:2.12cm) -- (0,0); 
                    \draw[fill=rougeLaBRI] (0,0) -- (-90:2.12cm) -- (315:3cm) -- (45:3cm) -- (0,0); 
                  \end{scope}
                  
                  \begin{pgfonlayer}{foreground}
                    \draw[fill=yellow] (0,-1.37) circle (1cm);
                  \end{pgfonlayer}
                \end{tikzpicture}
              \end{column}
              \begin{column}{.49\textwidth}

                \centering
                \begin{tikzpicture}    
                  \clip (-5,5) rectangle (5,-5);
                  \node[draw=black,fill=orange,circle,inner sep=6pt]   (a) at (90:4cm) {};
                  \node[draw=black,fill=yellow,circle,inner sep=6pt]  (b) at (162:4cm) {};
                  \node[draw=black,fill=rougeLaBRI,circle,inner sep=6pt] (c) at (234:4cm) {};
                  \node[draw=black,fill=bleuLaBRI,circle,inner sep=6pt] (d) at (306:4cm) {};
                  \node[draw=black,fill=yellow,circle,inner sep=6pt] (e) at (18:4cm) {};

                  \node[draw=black,fill=bleuLaBRI,circle,inner sep=6pt]   (f) at (90:1.7cm) {};
                  \node[draw=black,fill=orange,circle,inner sep=6pt]  (g) at (180:1.7cm) {};
                  \node[draw=black,fill=yellow,circle,inner sep=6pt] (h) at (270:1.7cm) {};
                  \node[draw=black,fill=rougeLaBRI,circle,inner sep=6pt] (i) at (0:1.7cm) {};

                  \draw (a) -- (b) -- (c) -- (d) -- (e) -- (a);
                  \draw (f) -- (a)  (g) -- (b)  (i) -- (e);
                  \draw (g) -- (i) -- (d) -- (h) -- (c) -- (g) -- (f) -- (i) -- (h) -- (g);
                \end{tikzpicture}
              \end{column}
            \end{columns}
            \vspace{1cm}
            \thm{Apple, Haken 1989} Every planar map (or graph) is 4-colorable
          \end{block}
        \end{minipage}
      \end{beamercolorbox}
    \end{column}
          
    \begin{column}{.49\textwidth}
      \begin{beamercolorbox}[center,wd=\textwidth]{postercolumn}
        \begin{minipage}[T]{.95\textwidth}

          \begin{block}{Kempe change}            
            \centering
            \begin{columns}
              \begin{column}{.49\textwidth}
                \centering
                \begin{tikzpicture}    
                  \clip (-5,5) rectangle (5,-5);
                  \node[draw=black,fill=orange,circle,inner sep=6pt]   (a) at (90:4cm) {};
                  \node[draw=black,fill=yellow,circle,inner sep=6pt]  (b) at (162:4cm) {};
                  \node[draw=black,fill=rougeLaBRI,circle,inner sep=6pt] (c) at (234:4cm) {};
                  \node[draw=black,fill=bleuLaBRI,circle,inner sep=6pt] (d) at (306:4cm) {};
                  \node[draw=black,fill=yellow,circle,inner sep=6pt] (e) at (18:4cm) {};

                  \node[draw=black,line width = 4pt,fill=yellow,circle,inner sep=6pt]   (f) at (90:1.7cm) {};
                  \node[draw=black,fill=orange,circle,inner sep=6pt]  (g) at (180:1.7cm) {};
                  \node[draw=black,fill=yellow,circle,inner sep=6pt] (h) at (270:1.7cm) {};
                  \node[draw=black,fill=rougeLaBRI,circle,inner sep=6pt] (i) at (0:1.7cm) {};

                  \draw (a) -- (b) -- (c) -- (d) -- (e) -- (a);
                  \draw (f) -- (a)  (g) -- (b)  (i) -- (e);
                  \draw (g) -- (i) -- (d) -- (h) -- (c) -- (g) -- (f) -- (i) -- (h) -- (g);
                \end{tikzpicture}
              \end{column}
              
              \begin{column}{.49\textwidth}
                \centering
                \begin{tikzpicture}    
                  \clip (-5,5) rectangle (5,-5);
                  \node[draw=black,fill=orange,circle,inner sep=6pt]   (a) at (90:4cm) {};
                  \node[draw=black,fill=yellow,circle,inner sep=6pt]  (b) at (162:4cm) {};
                  \node[draw=black,line width = 4pt, fill=orange,circle,inner sep=6pt] (c) at (234:4cm) {};
                  \node[draw=black,fill=bleuLaBRI,circle,inner sep=6pt] (d) at (306:4cm) {};
                  \node[draw=black,fill=yellow,circle,inner sep=6pt] (e) at (18:4cm) {};

                  \node[draw=black,fill=bleuLaBRI,circle,inner sep=6pt]   (f) at (90:1.7cm) {};
                  \node[draw=black,line width = 4pt, fill=rougeLaBRI,circle,inner sep=6pt]  (g) at (180:1.7cm) {};
                  \node[draw=black,fill=yellow,circle,inner sep=6pt] (h) at (270:1.7cm) {};
                  \node[draw=black,line width = 4pt, fill=orange,circle,inner sep=6pt] (i) at (0:1.7cm) {};

                  \draw (a) -- (b) -- (c) -- (d) -- (e) -- (a);
                  \draw (f) -- (a)  (g) -- (b)  (i) -- (e);
                  \draw  (i) -- (d) -- (h) -- (c)  (g) -- (f) -- (i) -- (h) -- (g);
                  \draw[line width = 4pt] (c) -- (g) -- (i);
                \end{tikzpicture}
              \end{column}
            \end{columns}
            \vspace{1cm}
            Swap the colors in a maximal bichromatic connected subgraph
            
          \end{block}

        \end{minipage}
      \end{beamercolorbox}
    \end{column}
  \end{columns}

  \begin{beamercolorbox}[center,wd=\textwidth]{postercolumn}
    \begin{minipage}[T]{.97\textwidth}
      \begin{block}{Objectives and questions}            
        \centering
        \begin{columns}
          \begin{column}{.49\textwidth}
            \begin{itemize}
            \item Color a graph  with as few colors as possible
            \item Sampling colorings efficiently
            \item Counting colorings
            \item Efficient enumeration of colorings 
            \end{itemize}
            
          \end{column}
          \begin{column}{.49\textwidth}
            \begin{itemize}
            \item Study of the reconfiguration graph\\ {\small Vertices: $k$-colorings of a
                graph $G$, edge if Kempe change between two colorings}
              \begin{itemize}
              \item Connectivity 
              \item Diameter
              \item Complexity of reachability problem
              \end{itemize}
            \end{itemize}
          \end{column}
        \end{columns}
      \end{block}
      
      \begin{block}{Applications}
        \begin{columns}
          \begin{column}{.45\textwidth}
            \begin{block}{Statistical physics}
              \centering
              \begin{tikzpicture}
                \pgfmathsetmacro{\r}{5}
                \pgfmathsetmacro{\long}{5}
                \pgfmathsetmacro{\haut}{3}

                
                \foreach \j [evaluate=\j] in {1,...,3}
                {\foreach \i [evaluate=\i] in {1,...,5}
                  {
                    \coordinate (v\i\j) at ({\i*\r}, {\j*\r}) {};
                  }
                }

                \foreach \j [evaluate=\j] in {1,2}
                {\foreach \i [evaluate=\i] in {1,...,4}
                  {
                    \pgfmathsetmacro{\ii}{int(\i +1)}
                    \pgfmathsetmacro{\jj}{int(\j +1)}
                    \draw (v\i\j) -- (v\ii\jj);
                    \draw (v\i\j) -- (v\ii\j);
                    \draw (v\i\j) -- (v\i\jj);
                  }
                }

                \foreach \j [evaluate=\j] in {1,2}
                {
                  \pgfmathsetmacro{\jj}{int(\j +1)}
                  \draw (v5\j) -- (v5\jj);
                }
                \foreach \i [evaluate=\i] in {1,...,4}
                {
                  \pgfmathsetmacro{\ii}{int(\i +1)}
                  \draw (v\i3) -- (v\ii3);
                }

                
                \foreach \x [evaluate=\x] in {11,32,23,53}
                {  
                  \node[draw=black,circle,fill=bleuLaBRI,inner sep = 6pt] (v\x) at (v\x) {};
                }
                \foreach \x [evaluate=\x] in {12,21,33,42,51}
                {  
                  \node[draw=black,circle,fill=rougeLaBRI,inner sep = 6pt] (v\x) at (v\x) {};
                }
                \foreach \x [evaluate=\x] in {41,13,43}
                {  
                  \node[draw=black,circle,fill=orange,inner sep = 6pt] (v\x) at (v\x) {};
                }
                \foreach \x [evaluate=\x] in {22,52,31}
                {  
                  \node[draw=black,circle,fill=yellow,inner sep = 6pt] (v\x) at (v\x) {};
                }
              \end{tikzpicture}
              
              Antiferromagnetic potts model
            \end{block}
          \end{column}
          \hfill
          \begin{column}{.45\textwidth}
            \begin{block}{Networks}
              \centering
              \begin{tikzpicture}
                \coordinate (a) at (0:4cm);
                \coordinate (b) at (72:4cm);
                \coordinate (c) at (144:4cm);
                \coordinate (d) at (216:4cm);
                \coordinate (e) at (288:4cm);
                \coordinate (f) at (162:6cm);
                \coordinate (g) at (126:6cm);

                \path[shorten >=1.3cm, shorten <= 1cm] (a) edge (b);
                \path[shorten >=1cm, shorten <= 1cm] (b) edge (c);
                \path[shorten >=1cm, shorten <= 1cm] (c) edge (d);
                \path[shorten >=1cm, shorten <= 1cm] (d) edge (e);
                \path[shorten >=1.3cm, shorten <= 1cm] (e) edge (a);
                \path[shorten >=1cm, shorten <= 1cm] (g) edge (b);
                \path[shorten >=1.2cm, shorten <= 1cm] (g) edge (c);
                \path[shorten >=1cm, shorten <= 1.3cm] (g) edge (f);
                \path[shorten >=1cm, shorten <= 1cm] (c) edge (f);
                \path[shorten >=1cm, shorten <= 1.2cm] (f) edge (d);

                \begin{scope}[scale=1.5,shift=(a),every node/.append style={transform shape}]
                  \node{\BS{rougeLaBRI}};
                \end{scope}
                \begin{scope}[scale=1.5,shift=(b),every node/.append style={transform shape}]
                  \node{\BS{bleuLaBRI}};
                \end{scope}
                \begin{scope}[scale=1.5,shift=(c),every node/.append style={transform shape}]
                  \node{\BS{orange}};
                \end{scope}
                \begin{scope}[scale=1.5,shift=(d),every node/.append style={transform shape}]
                  \node{\BS{rougeLaBRI}};
                \end{scope}
                \begin{scope}[scale=1.5,shift=(e),every node/.append style={transform shape}]
                  \node{\BS{yellow}};
                \end{scope}
                \begin{scope}[scale=1.5,shift=(f),every node/.append style={transform shape}]
                  \node{\BS{orange}};
                \end{scope}
                \begin{scope}[scale=1.5,shift=(g),every node/.append style={transform shape}]
                  \node{\BS{yellow}};
                \end{scope}

              \end{tikzpicture}
              \vspace{-.5cm}
              
              Telecommunication network with non-interfering frequences
              \vspace{-.6cm}
              
            \end{block}
          \end{column}
        \end{columns}
      \end{block}

    \end{minipage}
  \end{beamercolorbox}

  \begin{columns}
    \begin{column}{.49\textwidth}
      \begin{beamercolorbox}[center,wd=\textwidth]{postercolumn}
        \begin{minipage}[T]{.95\textwidth}
          \begin{block}{Color a graph with as few colors as possible}
            \thm{Heawood, Kempe 1890}
            Every planar graph is 5-colorable
            
            \textit{Proof:} 5-degeneracy: There exist a vertex of degree at most 5 in $G$
            
            \centering
            \begin{tikzpicture}
              \clip (-5,5) rectangle (5,-5.5);
              \node[draw=black,fill=orange,circle,inner sep=6pt]   (a) at (90:4cm) {};
              \node[draw=black,fill=rougeLaBRI,circle,inner sep=6pt]  (b) at (162:4cm) {};
              \node[draw=black,fill=yellow,circle,inner sep=6pt] (c) at (234:4cm) {};
              \node[draw=black,fill=bleuLaBRI,circle,inner sep=6pt] (d) at (306:4cm) {};
              \node[draw=black,fill=white,circle,inner sep=6pt] (e) at (18:4cm) {};
              \node[draw=black,fill=black,circle,inner sep=6pt] (A) at (0,0) {};
              
              
              \draw[semitransparent] (a) -- (b) -- (c) -- (d) -- (e) -- (a);
              \draw (b) -- (A) -- (a) (c) -- (A) -- (d) (A) --(e);
            \end{tikzpicture}

            \begin{columns}
              \begin{column}{.45\textwidth}
                \centering
                \begin{tikzpicture}
                  \clip (-5,5) rectangle (5,-5.5);
                  \draw[-{Stealth[scale=4]}] (5,5) -- (4,4);
                  \node[draw=black,fill=orange,circle,inner sep=6pt]   (a) at (90:4cm) {};
                  \node[draw=black,fill=rougeLaBRI,circle,inner sep=6pt]  (b) at (162:4cm) {};
                  \node[draw=black,fill=yellow,circle,inner sep=6pt] (c) at (234:4cm) {};
                  \node[draw=black,fill=bleuLaBRI,circle,inner sep=6pt] (d) at (306:4cm) {};
                  \node[draw=black,line width=4pt,fill=yellow,circle,inner sep=6pt] (e) at (18:4cm) {};
                  \node[draw=black,fill=white,circle,inner sep=6pt] (A) at (0,0) {};
                  
                  
                  \draw[semitransparent] (a) -- (b) -- (c) -- (d) -- (e) -- (a);
                  \draw (b) -- (A) -- (a) (c) -- (A) -- (d) (A) --(e);
                \end{tikzpicture}
              \end{column}
              or
              \begin{column}{.45\textwidth}
                \centering
                \begin{tikzpicture}
                  \clip (-5,5) rectangle (5,-5.5);
                  \draw[-{Stealth[scale=4]}] (-5,5) -- (-4,4);
                  \node[draw=black,fill=orange,circle,inner sep=6pt]   (a) at (90:4cm) {};
                  \node[draw=black,line width = 4pt,fill=bleuLaBRI,circle,inner sep=6pt]  (b) at (162:4cm) {};
                  \node[draw=black,fill=yellow,circle,inner sep=6pt] (c) at (234:4cm) {};
                  \node[draw=black,fill=bleuLaBRI,circle,inner sep=6pt] (d) at (306:4cm) {};
                  \node[draw=black,fill=white,circle,inner sep=6pt] (e) at (18:4cm) {};
                  \node[draw=black,fill=rougeLaBRI,circle,inner sep=6pt] (A) at (0,0) {};
                  \node (D1) at (286:7cm) {};
                  \node (D2) at (326:7cm) {};
                  
                  \draw[semitransparent] (a) -- (b) -- (c) -- (d) -- (e) -- (a);
                  \draw (b) -- (A) -- (a) (c) -- (A) -- (d) (A) --(e);
                  \draw[loosely dashed] (c) .. controls (D1) and (D2) .. (e);
                \end{tikzpicture}
              \end{column}
            \end{columns}

          \end{block}

          \vfill

          \begin{block}{Connectivity of the reconfiguration graph}
            
            \thm{Meyniel 1989} For all planar graph $G$, all the 5-colorings of $G$ are Kempe
            equivalent
            
            \vspace{.5cm}
            False for 4-colorings because of frozen colorings :

            \begin{center}
            \begin{columns}
              \begin{column}{.49\textwidth}
                \centering
                \begin{tikzpicture}
                  \node[draw=black,fill=orange,circle,inner sep=6pt]   (a) at (45:3cm) {};
                  \node[draw=black,fill=bleuLaBRI,circle,inner sep=6pt]  (b) at (135:3cm) {};
                  \node[draw=black,fill=yellow,circle,inner sep=6pt] (c) at (225:3cm) {};
                  \node[draw=black,fill=rougeLaBRI,circle,inner sep=6pt] (d) at (315:3cm) {};
                  \node[draw=black,fill=rougeLaBRI,circle,inner sep=6pt] (B) at (135:1.5cm) {};
                  \node[draw=black,fill=bleuLaBRI,circle,inner sep=6pt] (D) at (315:1.5cm) {};
                  \node[draw=black,fill=yellow,circle,inner sep=6pt] (A) at (45:5.4cm) {};
                  \node[draw=black,fill=orange,circle,inner sep=6pt] (C) at (225:5.4cm) {};
                  \node (D') at (315:6cm) {};
                  
                  \draw (a) -- (b) -- (c) -- (d) -- (a) -- (A) -- (b) -- (B) -- (D) -- (d)
                  -- (A);
                  \draw (c) -- (B) -- (a) -- (D) -- (c) -- (C)  (b) -- (C) -- (d) ;
                  \draw (A) .. controls (D') ..  (C);
                \end{tikzpicture}
              \end{column}
              \hfill
              
              \begin{column}{.49\textwidth}
                \centering
                \begin{tikzpicture}
                  \node[draw=black,fill=orange,circle,inner sep=6pt]   (a) at (45:3cm) {};
                  \node[draw=black,fill=bleuLaBRI,circle,inner sep=6pt]  (b) at (135:3cm) {};
                  \node[draw=black,fill=orange,circle,inner sep=6pt] (c) at (225:3cm) {};
                  \node[draw=black,fill=bleuLaBRI,circle,inner sep=6pt] (d) at (315:3cm) {};
                  \node[draw=black,fill=rougeLaBRI,circle,inner sep=6pt] (B) at (135:1.5cm) {};
                  \node[draw=black,fill=yellow,circle,inner sep=6pt] (D) at (315:1.5cm) {};
                  \node[draw=black,fill=yellow,circle,inner sep=6pt] (A) at (45:5.4cm) {};
                  \node[draw=black,fill=rougeLaBRI,circle,inner sep=6pt] (C) at (225:5.4cm) {};
                  \node (D') at (315:6cm) {};
                  
                  \draw (a) -- (b) -- (c) -- (d) -- (a) -- (A) -- (b) -- (B) -- (D) -- (d)
                  -- (A);
                  \draw (c) -- (B) -- (a) -- (D) -- (c) -- (C)  (b) -- (C) -- (d) ;
                  \draw (A) .. controls (D') ..  (C);
                \end{tikzpicture}
              \end{column}
            \end{columns}
            \end{center}
            The graphs induced by any two colors are connected, hence, the
            partition induced by the colors is invariant !
          \end{block}
        \end{minipage}
      \end{beamercolorbox}

    \end{column}
    \begin{column}{.49\textwidth}
      \begin{beamercolorbox}[center,wd=\textwidth]{postercolumn}
        \begin{minipage}[T]{.95\textwidth}
          \begin{block}{Diameter of the reconfiguration graph}
            \thm{Deschamps, Feghali, Kardoš, L-D., Pierron 22} All the 5-colorings
            of a planar graph are equivalent up to $O(n^{195})$ Kempe changes

            \vspace{.5cm}
            \textit{Proof:} If $G$ is 4-degenerate:
            \begin{columns}
              \begin{column}{.33\textwidth}
                \centering
                \begin{tikzpicture}
                  \node[draw,line width = 4pt,fill=rougeLaBRI, circle,inner sep=6pt] (A) at (0,0) {};
                  \node[draw,fill=bleuLaBRI, circle, inner sep=6pt] (a) at (20:4.5cm) {};
                  \node[draw,line width = 4pt,fill=yellow, circle, inner sep=6pt] (b) at (0:4.5cm) {};
                  \node[draw,line width = 4pt,fill=yellow, circle, inner sep=6pt] (c) at (320:4.5cm) {};
                  \node[draw,fill=orange, circle, inner sep=6pt] (d) at (340:4.5cm) {};
                  
                  \node (e) at (335:6.6cm) {};
                  \node (f) at (345:6.6cm) {};

                  \node (h) at (315:6.6cm) {};
                  \node (g) at (320:6.66cm) {};
                  \node (i) at (325:6.54cm) {};

                  \node (j) at (0:6.6cm) {};

                  \node (k) at (15:6.54cm) {};
                  \node (l) at (20:6.6cm) {};
                  \node (m) at (25:6.54cm) {};

                  \begin{scope}[shift={(350:8.4cm)}]
                    \draw[domain=135:205] plot ({6*cos(\x)}, {6*sin(\x)});
                  \end{scope}
                  
                  \draw[line width = 4pt] (j) -- (b) -- (A) -- (c);
                  \draw[line width = 4pt] (h) -- (c) -- (g)  (i) -- (c);
                  \draw (a) -- (A) -- (d);
                  \draw (f) -- (d) -- (e);
                  \draw (m) -- (a) -- (l)  (a) -- (k);
                \end{tikzpicture}                
              \end{column}
              \hfill
              \begin{column}{.3\textwidth}
                \centering
                \begin{tikzpicture}
                  \draw[-{Stealth[scale=4]}] (-4,-1) -- (-2,-1);                  
                  \node[draw,fill=white, circle,inner sep=6pt] (A) at (0,0) {};
                  \node[draw,fill=bleuLaBRI, circle, inner sep=6pt] (a) at (20:4.5cm) {};
                  \node[draw,fill=yellow, circle, inner sep=6pt] (b) at (0:4.5cm) {};
                  \node[draw,line width = 4pt,fill=yellow, circle, inner sep=6pt] (c) at (320:4.5cm) {};
                  \node[draw,fill=orange, circle, inner sep=6pt] (d) at (340:4.5cm) {};
                  
                  \node (e) at (335:6.6cm) {};
                  \node (f) at (345:6.6cm) {};

                  \node (h) at (315:6.6cm) {};
                  \node (g) at (320:6.66cm) {};
                  \node (i) at (325:6.54cm) {};

                  \node (j) at (0:6.6cm) {};

                  \node (k) at (15:6.54cm) {};
                  \node (l) at (20:6.6cm) {};
                  \node (m) at (25:6.54cm) {};

                  \begin{scope}[shift={(350:8.4cm)}]
                    \draw[domain=135:205] plot ({6*cos(\x)}, {6*sin(\x)});
                  \end{scope}
                  
                  \draw (j) -- (b) -- (A) -- (c);
                  \draw[line width = 4pt] (h) -- (c) -- (g)  (i) -- (c);
                  \draw (a) -- (A) -- (d);
                  \draw (f) -- (d) -- (e);
                  \draw (m) -- (a) -- (l)  (a) -- (k);
                \end{tikzpicture}                
              \end{column}
              \hfill
              \begin{column}{.3\textwidth}
                \centering
                \begin{tikzpicture}
                  \draw[-{Stealth[scale=4]}] (-4,-1) -- (-2,-1);
                  \node[draw,fill=white, circle,inner sep=6pt] (A) at (0,0) {};
                  \node[draw,fill=bleuLaBRI, circle, inner sep=6pt] (a) at (20:4.5cm) {};
                  \node[draw,fill=yellow, circle, inner sep=6pt] (b) at (0:4.5cm) {};
                  \node[draw,fill=rougeLaBRI, circle, inner sep=6pt] (c) at (320:4.5cm) {};
                  \node[draw,fill=orange, circle, inner sep=6pt] (d) at (340:4.5cm) {};
                  
                  \node (e) at (335:6.6cm) {};
                  \node (f) at (345:6.6cm) {};

                  \node (h) at (315:6.6cm) {};
                  \node (g) at (320:6.66cm) {};
                  \node (i) at (325:6.54cm) {};

                  \node (j) at (0:6.6cm) {};

                  \node (k) at (15:6.54cm) {};
                  \node (l) at (20:6.6cm) {};
                  \node (m) at (25:6.54cm) {};

                  \begin{scope}[shift={(350:8.4cm)}]
                    \draw[domain=135:205] plot ({6*cos(\x)}, {6*sin(\x)});
                  \end{scope}
                  
                  \draw (j) -- (b) -- (A) -- (c);
                  \draw (h) -- (c) -- (g);
                  \draw (a) -- (A) -- (d);
                  \draw (f) -- (d) -- (e);
                  \draw (i) -- (c);
                  \draw (m) -- (a) -- (l)  (a) -- (k);
                \end{tikzpicture}                
              \end{column}
            \end{columns}
            \begin{itemize}
            \item Emulate a 4-degenerate graph
            \item Add vertices by bunches of linear size
            \end{itemize}
          \end{block}

          \begin{block}{Sampling coloring}
            Do random Kempe changes until reaching a close to random uniform
            $k$-coloring $\leadsto$ study of a Markov chain

            \vspace{.7cm}
            \thm{Vigoda 2000} For single vertex recoloring, polynomial mixing
            time when $k > \frac{11}{6}\Delta$

            \conj Sampling with random Kempe changes instead of single vertex
            changes is always more efficient.

            \vspace{.5cm}
            \thm{Hayes, Vera, Vigoda 2018} For planar graphs and single vertex recoloring,
            polynomial mixing time when $k = \Omega\left(\frac{\Delta}{\log\Delta}\right)$

            \conj $k$ can be significantly reduced  when working with Kempe changes
          \end{block}

          \begin{block}{Recoloring version of Grötzsch's theorem}
            \thm{Grötzsch 1959} All triangle-free planar graphs are 3-colorable

            \conj All the 3-colorings of a triangle-free planar graph are Kempe equivalent
          \end{block}
        \end{minipage}
      \end{beamercolorbox}
    \end{column}
  \end{columns}
  \vskip1ex
\end{frame}
\end{document}
