\documentclass[11pt,xcolor=dvipsnames,presentation]{beamer}

\usepackage{BeamerColor}
\usepackage{etoolbox}

\usepackage[T1]{fontenc}%
\usepackage[utf8]{inputenc}%
\usepackage[main= english,francais]{babel}%
\usepackage{mathtools}

\usepackage{graphicx}%
\usepackage{courier}%
\usepackage{algorithm2e}

\usepackage{url}%
\urlstyle{sf}%

\usepackage{subcaption}%
\usepackage{caption}
\captionsetup[figure]{labelformat=empty}% redefines the caption setup of the figures environment in the beamer class.
\usepackage{tikz}
\usetikzlibrary{arrows}

\usepackage{listings}%

\usepackage{faktor}%

\usepackage{stmaryrd}%

\usepackage[style=verbose,backend=biber]{biblatex}
\addbibresource{biblio.bib}

\newcommand\blfootnote[1]{%
  \begingroup
  \renewcommand\thefootnote{}\footnote{#1}%
  \addtocounter{footnote}{-1}%
  \endgroup
}

%%%%%%%%% Begin of Beamer Layout %%%%%%%%%%%%%
\ProcessOptionsBeamer
\usecolortheme{whale}
\usecolortheme[named=DarkGreen]{structure}
\useinnertheme{rounded}
\useoutertheme{infolines}
\setbeamertemplate{footline}[frame number]
\setbeamertemplate{headline}[default]
\setbeamertemplate{navigation symbols}{}
\setbeamertemplate{itemize items}[circle]
\defbeamertemplate*{headline}{info theme}{}
\defbeamertemplate*{footline}{info theme}{\leavevmode%
\hbox{%
\begin{beamercolorbox}[wd=.2\paperwidth,ht=2.25ex,dp=1ex,center]{author in head/foot}%
\usebeamerfont{author in head/foot}\insertshortauthor
\end{beamercolorbox}%
\begin{beamercolorbox}[wd=.71\paperwidth,ht=2.25ex,dp=1ex,center]{title in head/foot}%
\usebeamerfont{title in head/foot}\insertsectionhead
\end{beamercolorbox}%
\begin{beamercolorbox}[wd=.09\paperwidth,ht=2.25ex,dp=1ex,right]{section in head/foot}%
\usebeamerfont{section in head/foot}\insertframenumber{}~/~\inserttotalframenumber\hspace*{2ex}
\end{beamercolorbox}
}\vskip0pt}
\setbeamercolor*{block title example}{fg=LightSeaGreen}
\AtBeginEnvironment{exampleblock}{\setbeamercolor{itemize item}{fg=LightSeaGreen}}
\setbeamercolor*{block title alerted}{fg=DarkOrange2}
\AtBeginEnvironment{alertblock}{\setbeamercolor{itemize item}{fg=DarkOrange2}}
\setbeamertemplate{footline}[info theme]
\defbeamertemplate{description item}{align left}{\insertdescriptionitem\hfill}
%%%%%%%%% End of Beamer Layout %%%%%%%%%%%%%
\newcommand{\myorange}{DarkOrange2}
\newcommand{\mygreen}{DarkGreen}
\newcommand{\mycyan}{LightSeaGreen}


\lstset{%
	basicstyle=\sffamily,%
	columns=fullflexible,%
	language=C++,% % À ajuster dans chaque projet!
	frame=lb,%
	frameround=fftf,%
}%

\sloppy

%\setbeamertemplate{headline}{}%

\setbeamertemplate{navigation symbols}{%
	\insertslidenavigationsymbol%
	% \insertframenavigationsymbol%
	% \insertsubsectionnavigationsymbol%
	% \insertsectionnavigationsymbol%
	% \insertdocnavigationsymbol%
	% \insertbackfindforwardnavigationsymbol%
}

\newcommand{\backupbegin}{
   \newcounter{finalframe}
   \setcounter{finalframe}{\value{framenumber}}
}
\newcommand{\backupend}{
   \setcounter{framenumber}{\value{finalframe}}
}

\DeclareMathOperator{\diam}{diam}
\DeclareMathOperator{\tw}{tw}

\begin{document}

\title{Kempe recoloring in bounded treewidth graphs}
\author[Clément Legrand]{Clément Legrand-Duchesne}
\institute{LaBRI, Université de Bordeaux}
\date{\today}

\begin{frame}
  \titlepage
  \centering
  joint work with Marthe Bonamy and Vincent Delecroix
\end{frame}

\section*{Some context}

\begin{frame}{Recoloring with Kempe changes}
  \uncover<1->{
  \begin{columns}
    \begin{column}{.66\textwidth}
      \begin{exampleblock}{Kempe chain (1879)}
        Maximal bichromatic component in $G$
      \end{exampleblock}
    \end{column}
    \begin{column}{.33\textwidth}
      \centering
      \begin{tikzpicture}
        \node[draw=black,fill=\myorange,circle,inner sep=2pt]   (a) at (90:.8cm) {};
        \node[draw=black,fill=\mycyan,circle,inner sep=2pt]  (b) at (162:.8cm) {};
        \node[draw=black,fill=magenta,circle,inner sep=2pt] (c) at (234:.8cm) {};
        \node[draw=black,fill=blue,circle,inner sep=2pt] (d) at (306:.8cm) {};
        \node[draw=black,fill=magenta,circle,inner sep=2pt] (e) at (18:.8cm) {};

        \draw (a) -- (b) -- (c) -- (d) -- (e) -- (a);
      \end{tikzpicture}
      
      %\includegraphics[width=.8\textwidth]{../figures/pdf/knot_example.pdf}
    \end{column}
  \end{columns}
  }

  \vspace{1.5cm}
  
  \begin{columns}
    \uncover<2->{
      \begin{column}{.33\textwidth}
        \centering
        \begin{tikzpicture}
          \node[draw=black,fill=\myorange,circle,inner sep=2pt]   (a) at (90:.8cm) {};
          \node[draw=black,fill=\mycyan,circle,inner sep=2pt]  (b) at (162:.8cm) {};
          \node[draw=black,very thick,fill=blue,circle,inner sep=2pt] (c) at (234:.8cm) {};
          \node[draw=black,very thick,fill=magenta,circle,inner sep=2pt] (d) at (306:.8cm) {};
          \node[draw=black,very thick,fill=blue,circle,inner sep=2pt] (e) at (18:.8cm) {};

          \draw (e) -- (a) -- (b) -- (c);
          \draw[very thick] (c) -- (d) -- (e);
        \end{tikzpicture}      
    \end{column}
    }
    \uncover<3->{
    \begin{column}{.33\textwidth}
       \centering
      \begin{tikzpicture}
        \node[draw=black,fill=\myorange,circle,inner sep=2pt]   (a) at (90:.8cm) {};
        \node[draw=black,very thick,fill=magenta,circle,inner sep=2pt]  (b) at (162:.8cm) {};
        \node[draw=black,very thick,fill=\mycyan,circle,inner sep=2pt] (c) at (234:.8cm) {};
        \node[draw=black,fill=blue,circle,inner sep=2pt] (d) at (306:.8cm) {};
        \node[draw=black,fill=magenta,circle,inner sep=2pt] (e) at (18:.8cm) {};

        \draw[very thick] (b) -- (c);
        \draw (a) -- (b);
        \draw (c) -- (d) -- (e) -- (a);
      \end{tikzpicture}
    \end{column}
    }
    \uncover<4->{
    \begin{column}{.33\textwidth}
       \centering
      \begin{tikzpicture}
        \node[draw=black,fill=\myorange,circle,inner sep=2pt]   (a) at (90:.8cm) {};
        \node[draw=black,very thick,fill=blue,circle,inner sep=2pt]  (b) at (162:.8cm) {};
        \node[draw=black,fill=magenta,circle,inner sep=2pt] (c) at (234:.8cm) {};
        \node[draw=black,fill=blue,circle,inner sep=2pt] (d) at (306:.8cm) {};
        \node[draw=black,fill=magenta,circle,inner sep=2pt] (e) at (18:.8cm) {};

        \draw (a) -- (b) -- (c) -- (d) -- (e) -- (a);
      \end{tikzpicture}
    \end{column}
    }
  \end{columns}
\end{frame}

\begin{frame}{Natural questions}
    \begin{block}{Application}
    \begin{itemize}
    \item Powerful tool (ex: Vizing theorem)
    \item Sampling coloring with a Markov chain
    \end{itemize}
  \end{block}

  \begin{exampleblock}{Reconfiguration graph}
    \begin{itemize}
    \item $V(R^k(G)) =$ $k$-colorings of $G$.
    \item $\alpha$ and $\beta$ adjacent  if $\alpha
      \xleftrightarrow[\text{Kempe}]{} \beta$
    \end{itemize}
  \end{exampleblock}
  \setbeamertemplate{description item}[align left] 
  \begin{description}
  \item[Reachability] Are $\alpha$ and $\beta$ Kempe-equivalent ? 
  \item[Connectivity] Is $R^k(G)$ connected ?
  \item[Diameter] Estimate $\diam(R^k(G))$ ?
  \end{description}
\end{frame}

\begin{frame}{Warm up}
  \begin{exampleblock}{$d$-degenerate graph}
    $G$ is $d$-degenerate if for any $H \subset G$, $\delta(H)\le
    d$\\ Equivalently, $G$ admits an elimination ordering $v_1 \prec v_2 \dots
    \prec v_n$ such that $$\forall i, |N^+(v_i)| \le d$$
  \end{exampleblock}

%%   \begin{itemize}
%%   \item $G$ has degree less than $k$ implies $G$ $k$-degenerate
%%   \item $G$ not regular of degree less than $k$ implies $G$ $(k-1)$-degenerate
%%   \end{itemize}

  \vspace{1cm}
  \centering
    \begin{tikzpicture}[auto,node distance=1cm,thick]
      \clip (-.5,.8) rectangle (7.5,-1.5);
       
      \node[draw=black,circle,inner sep=2pt] (1) {};
      \node[draw=black,circle,inner sep=2pt] (2) [right of=1] {};
      \node[draw=black,circle,inner sep=2pt] (3) [right of=2] {};
      \node[draw=black,circle,inner sep=2pt] (4) [right of=3] {};
      \node[draw=black,circle,inner sep=2pt] (5) [right of=4] {};
      \node[draw=black,circle,inner sep=2pt] (6) [right of=5] {};
      \node[draw=black,circle,inner sep=2pt] (7) [right of=6] {};


      \draw (1) -- (2);
      \draw (3) -- (4) -- (5) -- (6) -- (7);
      \path[every node/.style={font=\sffamily\small}]
      (4) edge[bend right] node {} (6)
      (7) edge[bend right] node {} (5)
      (4) edge[bend left] node {} (2)
      (2) edge[bend left] node {} (5)
      (1) edge[bend left] node {} (4)
      (3) edge[bend right] node {} (7);

  \end{tikzpicture}
  
%%   \begin{block}{With $\Delta+ 1$ colors}
%%     \begin{itemize}
%%     \item $\forall G, \chi(G) \le \Delta +1$
%%     \item $\forall G, \forall k \ge \Delta+1, R^k(G)$ is connected
%%     \end{itemize}
%%   \end{block}
\end{frame}

\begin{frame}{Warm up}
  \begin{block}{Las Vergnas, Meyniel 1981}
  All $k$-colorings of a $d$-degenerate graph are Kempe-equivalent for
    $k \ge d+1$. 
  \end{block}
  
  \begin{columns}
    \begin{column}{.6\textwidth}
      Induction: Let $\alpha$ and $\beta$ be two colorings of $G$\\
      Consider $G' = G\setminus\{v_1\}$
      $$\alpha_{|G'} \mathop{\leadsto}_{K}
      \beta_{|G'}$$
      Lift sequence to $G$ then recolor $v_1$
    \end{column}
    \hfill
    \begin{column}{.3\textwidth}
      \centering
      \only<1>{
        \begin{tikzpicture}
          \clip (-.5,.7) rectangle (3,-2);
          \node[draw,fill=magenta, circle,inner sep=2pt] (a) at (0,0) {};
          \node[above] (A) at (0,0) {$v_1$};
          \node[draw,fill=blue, circle, inner sep=2pt] (b) at (0:1.5cm) {};
          \node[draw,very thick,fill=blue, circle, inner sep=2pt] (c) at (320:1.5cm) {};
          \node[draw,fill=\myorange, circle, inner sep=2pt] (d) at (340:1.5cm) {};
          
          \node (e) at (335:2.2cm) {};
          \node (f) at (345:2.2cm) {};

          \node (h) at (315:2.2cm) {};
          \node (g) at (320:2.22cm) {};
          \node (i) at (325:2.18cm) {};

          \node (j) at (0:2.2cm) {};

          \node[above] (G) at (2.6, -1.2) {$G'$};
          \draw[domain=120:180] plot ({2*(cos(\x)+1.3)}, {2*(sin(\x)-.6)});

          \draw (j) -- (b) -- (a) -- (c);
          \draw[very thick] (h) -- (c) -- (g);
          \draw (a) -- (d);
          \draw (f) -- (d) -- (e);
          \draw (i) -- (c);
          
      \end{tikzpicture}}

      \only<2>{
        \begin{tikzpicture}
          \clip (-.5,.7) rectangle (3,-2);
          \node[draw,very thick,fill=magenta, circle,inner sep=2pt] (a) at (0,0) {};
          \node[above] (A) at (0,0) {$v_1$};
          \node[draw,very thick,fill=blue, circle, inner sep=2pt] (b) at (0:1.5cm) {};
          \node[draw,very thick,fill=blue, circle, inner sep=2pt] (c) at (320:1.5cm) {};
          \node[draw,fill=\myorange, circle, inner sep=2pt] (d) at (340:1.5cm) {};
          
          \node (e) at (335:2.2cm) {};
          \node (f) at (345:2.2cm) {};

          \node (h) at (315:2.2cm) {};
          \node (g) at (320:2.22cm) {};
          \node (i) at (325:2.18cm) {};

          \node (j) at (0:2.2cm) {};

          \node[above] (G) at (2.6, -1.2) {$G'$};
          \draw[domain=120:180] plot ({2*(cos(\x)+1.3)}, {2*(sin(\x)-.6)});

          \draw[very thick] (j) -- (b) -- (a) -- (c);
          \draw[very thick] (h) -- (c) -- (g);
          \draw (a) -- (d);
          \draw (f) -- (d) -- (e);
          \draw (i) -- (c);
        \end{tikzpicture}}

      \only<3->{
        \begin{tikzpicture}
          \clip (-.5,.7) rectangle (3,-2);
          \node[draw,fill=\mycyan, circle,inner sep=2pt] (a) at (0,0) {};
          \node[above] (A) at (0,0) {$v_1$};
          \node[draw,fill=blue, circle, inner sep=2pt] (b) at (0:1.5cm) {};
          \node[draw,very thick,fill=blue, circle, inner sep=2pt] (c) at (320:1.5cm) {};
          \node[draw,fill=\myorange, circle, inner sep=2pt] (d) at (340:1.5cm) {};
          
          \node (e) at (335:2.2cm) {};
          \node (f) at (345:2.2cm) {};

          \node (h) at (315:2.2cm) {};
          \node (g) at (320:2.22cm) {};
          \node (i) at (325:2.18cm) {};

          \node (j) at (0:2.2cm) {};

          \node[above] (G) at (2.6, -1.2) {$G'$};
          \draw[domain=120:180] plot ({2*(cos(\x)+1.3)}, {2*(sin(\x)-.6)});

          \draw (j) -- (b) -- (a) -- (c);
          \draw[very thick] (h) -- (c) -- (g);
          \draw (a) -- (d);
          \draw (f) -- (d) -- (e);
          \draw (i) -- (c);
        \end{tikzpicture}}
    \end{column}
  \end{columns}
  \uncover<4>{
  \begin{alertblock}{Natural question}
   What is the diameter of the reconfiguration graph in this setting ?
  \end{alertblock}
  }
\end{frame}

\section{Recoloring under bounded treewidth}

\begin{frame}{Recoloring via trivial Kempe changes}

  \begin{block}{Cerecedas '07}
    $R_{\text{Glauber}}^k(G)$ is connected if $G$ is $d$-degenerate and $k \ge d+2$
  \end{block}

  \begin{alertblock}{Cerecedas' conjecture '07}
    $\diam(R_{\text{Glauber}}^k(G)) \le O(n^2)$ if $G$ is $d$-degenerate and $k \ge d+2$
  \end{alertblock}

  \pause
  
  \begin{exampleblock}{Treewidth}
    Graph parameter that measures how close a graph is from being a tree\\
    $\tw(G) \le k$ implies $G$ is  $k$-degenerate
  \end{exampleblock}

  \begin{block}{Bonamy, Bousquet '13}
    $\diam(R_{\text{Glauber}}^k(G)) \le O(n^2)$ if $k \ge \tw(G) + 2$
  \end{block}

  \pause
  \begin{block}{Bousquet, Heinrich '19}
    $\diam(R_{\text{Glauber}}^k(G)) \le O(n^{d+1})$ if $G$ is $d$-degenerate and $k \ge d+2$
  \end{block}
  
\end{frame}

\begin{frame}{Recoloring with Kempe changes}
  \begin{exampleblock}{Chordal}
    \begin{itemize}
    \item A graph is chordal if it all its induced cycles are triangles
    \item A graph is chordal iff there exists an ordering of the vertices $v_1
      \prec v_2 \dots \prec v_n$, such that $\forall i, N^+[v_i]$ is a clique
    \item Chordal graphs are perfect : $\chi(H) = \omega(H)$
    \end{itemize}
  \end{exampleblock}
  
  \begin{block}{Bonamy, Heinrich, Ito, Kobayashi, Mizuta, Mühlenthaler, Suzuki,
      Wasa '20}
    If $H$ is chordal, $\diam(R^k(G)) \le n$ for $k \ge \chi(H)$ 
  \end{block}

  \only<1>{
    \centering
    \begin{tikzpicture}[auto,node distance=1cm,thick]
      \clip (-.5,.8) rectangle (6.6,-1.5);
       
      \node[draw=black,fill=\myorange,circle,inner sep=2pt] (b) {};
      \node[draw=black,fill=blue,circle,inner sep=2pt] (c) [right of=b] {};
      \node[draw=black,fill=\mycyan,circle,inner sep=2pt] (d) [right of=c] {};
      \node[draw=black,fill=blue,circle,inner sep=2pt] (e) [right of=d] {};
      \node[draw=black,fill=magenta,circle,inner sep=2pt] (f) [right of=e] {};
      \node[draw=black,fill=\myorange,circle,inner sep=2pt] (g) [right of=f] {};
      \node[draw=black,fill=\mycyan,circle,inner sep=2pt] (h) [right of=g] {};


      \draw (b) -- (c) -- (d) -- (e) -- (f) -- (g) -- (h);
      \path[every node/.style={font=\sffamily\small}]
      (e) edge[bend right] node {} (g)
      (g) edge[bend right] node {} (d)
      (d) edge[bend right] node {} (f)
      (f) edge[bend left] node {} (h);

      \path[every node/.style={font=\sffamily\small}]
      (f) edge[bend right] node {} (c);

      \path[every node/.style={font=\sffamily\small}]
      (b) edge[bend left] node {} (f);

      \node[draw=black,fill=\mycyan,circle,inner sep=2pt] (B) [below of=b] {};
      \node[draw=black,fill=blue,circle,inner sep=2pt] (C) [below of=c] {};
      \node[draw=black,fill=\mycyan,circle,inner sep=2pt] (D) [below of=d] {};
      \node[draw=black,fill=magenta,circle,inner sep=2pt] (E) [below of=e] {};
      \node[draw=black,fill=\myorange,circle,inner sep=2pt] (F) [below of=f] {};
      \node[draw=black,fill=blue,circle,inner sep=2pt] (G) [below of=g] {};
      \node[draw=black,fill=magenta,circle,inner sep=2pt] (H) [below of=h] {};
  \end{tikzpicture}}

  
  \only<2>{
    \centering
    \begin{tikzpicture}[auto,node distance=1cm,thick]
      \clip (-.5,.8) rectangle (6.6,-1.5);
      
      \node[draw=black,fill=\myorange,circle,inner sep=2pt] (b) {};
      \node[draw=black,fill=blue,circle,inner sep=2pt] (c) [right of=b] {};
      \node[draw=black,very thick,fill=\mycyan,circle,inner sep=2pt] (d) [right of=c] {};
      \node[draw=black,fill=blue,circle,inner sep=2pt] (e) [right of=d] {};
      \node[draw=black,very thick,fill=magenta,circle,inner sep=2pt] (f) [right of=e] {};
      \node[draw=black,fill=\myorange,circle,inner sep=2pt] (g) [right of=f] {};
      \node[draw=black,very thick,fill=\mycyan,circle,inner sep=2pt] (h) [right of=g] {};


      \draw (b) -- (c) -- (d) -- (e) -- (f) -- (g) -- (h);
      \path[every node/.style={font=\sffamily\small}]
      (e) edge[bend right] node {} (g)
      (g) edge[bend right] node {} (d)
      (d) edge[bend right,very thick] node {} (f)
      (f) edge[bend left, very thick] node {} (h);

      \path[every node/.style={font=\sffamily\small}]
      (f) edge[bend right] node {} (c);
      
      \path[every node/.style={font=\sffamily\small}]
      (b) edge[bend left] node {} (f);

      \node[draw=black,fill=\mycyan,circle,inner sep=2pt] (B) [below of=b] {};
      \node[draw=black,fill=blue,circle,inner sep=2pt] (C) [below of=c] {};
      \node[draw=black,fill=\mycyan,circle,inner sep=2pt] (D) [below of=d] {};
      \node[draw=black,fill=magenta,circle,inner sep=2pt] (E) [below of=e] {};
      \node[draw=black,fill=\myorange,circle,inner sep=2pt] (F) [below of=f] {};
      \node[draw=black,fill=blue,circle,inner sep=2pt] (G) [below of=g] {};
      \node[draw=black,fill=magenta,circle,inner sep=2pt] (H) [below of=h] {};
  \end{tikzpicture}}

  
  \only<3>{
    \centering
    \begin{tikzpicture}[auto,node distance=1cm,thick]
      \clip (-.5,.8) rectangle (6.6,-1.5);
      
      \node[draw=black,fill=\myorange,circle,inner sep=2pt] (b) {};
      \node[draw=black,fill=blue,circle,inner sep=2pt] (c) [right of=b] {};
      \node[draw=black,fill=magenta,circle,inner sep=2pt] (d) [right of=c] {};
      \node[draw=black,very thick,fill=blue,circle,inner sep=2pt] (e) [right of=d] {};
      \node[draw=black,fill=\mycyan,circle,inner sep=2pt] (f) [right of=e] {};
      \node[draw=black,very thick,fill=\myorange,circle,inner sep=2pt] (g) [right of=f] {};
      \node[draw=black,fill=magenta,circle,inner sep=2pt] (h) [right of=g] {};


      \draw (b) -- (c) -- (d) -- (e) -- (f) -- (g) -- (h);
      \path[every node/.style={font=\sffamily\small}]
      (e) edge[bend right, very thick] node {} (g)
      (g) edge[bend right] node {} (d)
      (d) edge[bend right] node {} (f)
      (f) edge[bend left] node {} (h);

      \path[every node/.style={font=\sffamily\small}]
      (f) edge[bend right] node {} (c);
      
      \path[every node/.style={font=\sffamily\small}]
      (b) edge[bend left] node {} (f);

      \node[draw=black,fill=\mycyan,circle,inner sep=2pt] (B) [below of=b] {};
      \node[draw=black,fill=blue,circle,inner sep=2pt] (C) [below of=c] {};
      \node[draw=black,fill=\mycyan,circle,inner sep=2pt] (D) [below of=d] {};
      \node[draw=black,fill=magenta,circle,inner sep=2pt] (E) [below of=e] {};
      \node[draw=black,fill=\myorange,circle,inner sep=2pt] (F) [below of=f] {};
      \node[draw=black,fill=blue,circle,inner sep=2pt] (G) [below of=g] {};
      \node[draw=black,fill=magenta,circle,inner sep=2pt] (H) [below of=h] {};
  \end{tikzpicture}}

  
  \only<4>{
    \centering
    \begin{tikzpicture}[auto,node distance=1cm,thick]
      \clip (-.5,.8) rectangle (6.6,-1.5);

      \node[draw=black,very thick,fill=\myorange,circle,inner sep=2pt] (b) {};
      \node[draw=black,fill=blue,circle,inner sep=2pt] (c) [right of=b] {};
      \node[draw=black,fill=magenta,circle,inner sep=2pt] (d) [right of=c] {};
      \node[draw=black,very thick,fill=\myorange,circle,inner sep=2pt] (e) [right of=d] {};
      \node[draw=black,very thick,fill=\mycyan,circle,inner sep=2pt] (f) [right of=e] {};
      \node[draw=black,fill=blue,circle,inner sep=2pt] (g) [right of=f] {};
      \node[draw=black,fill=magenta,circle,inner sep=2pt] (h) [right of=g] {};


      \draw (b) -- (c) -- (d) -- (e);
      \draw[very thick] (e)-- (f);
      \draw (f) -- (g) -- (h);
      \path[every node/.style={font=\sffamily\small}]
      (e) edge[bend right] node {} (g)
      (g) edge[bend right] node {} (d)
      (d) edge[bend right] node {} (f)
      (f) edge[bend left] node {} (h);

      \path[every node/.style={font=\sffamily\small}]
      (f) edge[bend right] node {} (c);
      
      \path[every node/.style={font=\sffamily\small}]
      (b) edge[bend left,very thick] node {} (f);

      \node[draw=black,fill=\mycyan,circle,inner sep=2pt] (B) [below of=b] {};
      \node[draw=black,fill=blue,circle,inner sep=2pt] (C) [below of=c] {};
      \node[draw=black,fill=\mycyan,circle,inner sep=2pt] (D) [below of=d] {};
      \node[draw=black,fill=magenta,circle,inner sep=2pt] (E) [below of=e] {};
      \node[draw=black,fill=\myorange,circle,inner sep=2pt] (F) [below of=f] {};
      \node[draw=black,fill=blue,circle,inner sep=2pt] (G) [below of=g] {};
      \node[draw=black,fill=magenta,circle,inner sep=2pt] (H) [below of=h] {};
  \end{tikzpicture}}
  
  \only<5>{
    \centering
    \begin{tikzpicture}[auto,node distance=1cm,thick]
      \clip (-.5,.8) rectangle (6.6,-1.5);
      
      \node[draw=black,fill=\mycyan,circle,inner sep=2pt] (b) {};
      \node[draw=black,fill=blue,circle,inner sep=2pt] (c) [right of=b] {};
      \node[draw=black,very thick,fill=magenta,circle,inner sep=2pt] (d) [right of=c] {};
      \node[draw=black,very thick,fill=\mycyan,circle,inner sep=2pt] (e) [right of=d] {};
      \node[draw=black,fill=\myorange,circle,inner sep=2pt] (f) [right of=e] {};
      \node[draw=black,fill=blue,circle,inner sep=2pt] (g) [right of=f] {};
      \node[draw=black,fill=magenta,circle,inner sep=2pt] (h) [right of=g] {};


      \draw (b) -- (c) -- (d);
      \draw[very thick] (d)-- (e);
      \draw (e) -- (f) -- (g) -- (h);
      \path[every node/.style={font=\sffamily\small}]
      (e) edge[bend right] node {} (g)
      (g) edge[bend right] node {} (d)
      (d) edge[bend right] node {} (f)
      (f) edge[bend left] node {} (h);

      \path[every node/.style={font=\sffamily\small}]
      (f) edge[bend right] node {} (c);
      
      \path[every node/.style={font=\sffamily\small}]
      (b) edge[bend left] node {} (f);

      \node[draw=black,fill=\mycyan,circle,inner sep=2pt] (B) [below of=b] {};
      \node[draw=black,fill=blue,circle,inner sep=2pt] (C) [below of=c] {};
      \node[draw=black,fill=\mycyan,circle,inner sep=2pt] (D) [below of=d] {};
      \node[draw=black,fill=magenta,circle,inner sep=2pt] (E) [below of=e] {};
      \node[draw=black,fill=\myorange,circle,inner sep=2pt] (F) [below of=f] {};
      \node[draw=black,fill=blue,circle,inner sep=2pt] (G) [below of=g] {};
      \node[draw=black,fill=magenta,circle,inner sep=2pt] (H) [below of=h] {};
  \end{tikzpicture}}

  \only<6>{
    \centering    
    \begin{tikzpicture}[auto,node distance=1cm,thick]
      \clip (-.5,.8) rectangle (6.6,-1.5);

      \node[draw=black,fill=\mycyan,circle,inner sep=2pt] (b) {};
      \node[draw=black,fill=blue,circle,inner sep=2pt] (c) [right of=b] {};
      \node[draw=black,fill=\mycyan,circle,inner sep=2pt] (d) [right of=c] {};
      \node[draw=black,fill=magenta,circle,inner sep=2pt] (e) [right of=d] {};
      \node[draw=black,fill=\myorange,circle,inner sep=2pt] (f) [right of=e] {};
      \node[draw=black,fill=blue,circle,inner sep=2pt] (g) [right of=f] {};
      \node[draw=black,fill=magenta,circle,inner sep=2pt] (h) [right of=g] {};


      \draw (b) -- (c) -- (d) -- (e)-- (f) -- (g) -- (h);
      \path[every node/.style={font=\sffamily\small}]
      (e) edge[bend right] node {} (g)
      (g) edge[bend right] node {} (d)
      (d) edge[bend right] node {} (f)
      (f) edge[bend left] node {} (h);

      \path[every node/.style={font=\sffamily\small}]
      (f) edge[bend right] node {} (c);
      
      \path[every node/.style={font=\sffamily\small}]
      (b) edge[bend left] node {} (f);

      \node[draw=black,fill=\mycyan,circle,inner sep=2pt] (B) [below of=b] {};
      \node[draw=black,fill=blue,circle,inner sep=2pt] (C) [below of=c] {};
      \node[draw=black,fill=\mycyan,circle,inner sep=2pt] (D) [below of=d] {};
      \node[draw=black,fill=magenta,circle,inner sep=2pt] (E) [below of=e] {};
      \node[draw=black,fill=\myorange,circle,inner sep=2pt] (F) [below of=f] {};
      \node[draw=black,fill=blue,circle,inner sep=2pt] (G) [below of=g] {};
      \node[draw=black,fill=magenta,circle,inner sep=2pt] (H) [below of=h] {};
  \end{tikzpicture}}
\end{frame}
  
\begin{frame}{Kempe recoloring with bounded treewidth}  
  \begin{block}{Bonamy, Delecroix, L. '21}
    $\diam(R^k(G)) \le O(\tw n^2)$ if $k \ge \tw(G)+1$
  \end{block}
  $G$ has treewidth $\tw$ iff there exists $H$ chordal with $\omega(H) = \tw+1$.
  
  
  Let $G$ and $k \ge \tw(G)+1$, $H$ an overlying chordal graph with elimination
  ordering $v_1 \prec v_2 \prec \dots \prec v_n$

  
  \vspace{.3cm}
  \centering
  \begin{tikzpicture}
    \node (G) at (-2,0) {Coloring of $G$};
    \node (H) at (-2,-1) {Coloring of $H$};
    \node (a) at (0,0) {$\alpha$};
    \node (b) at (5,0) {$\beta$};
    \node (A) at (1.5,-1) {$\alpha'$};
    \node (B) at (3.5,-1) {$\beta'$};

    \draw[->] (a) -- (A);
    \draw[->] (A) -- (B);
    \draw[->] (B) -- (b);
  \end{tikzpicture}

  \vspace{.3cm}
  \only<1>{
    \centering
    \begin{tikzpicture}[auto,node distance=1cm,thick]
      \clip (-.5,.8) rectangle (7.5,-.8);
      \node[draw=black,fill=blue,circle,inner sep=2pt] (a) {};
      \node[draw=black,fill=\mycyan,circle,inner sep=2pt] (b) [right of=a] {};
      \node[draw=black,fill=\myorange,circle,inner sep=2pt] (c) [right of=b] {};
      \node[draw=black,fill=\mycyan,circle,inner sep=2pt] (d) [right of=c] {};
      \node[draw=black,fill=magenta,circle,inner sep=2pt] (e) [right of=d] {};
      \node[draw=black,fill=magenta,circle,inner sep=2pt] (f) [right of=e] {};
      \node[draw=black,fill=\myorange,circle,inner sep=2pt] (g) [right of=f] {};
      \node[draw=black,fill=\myorange,circle,inner sep=2pt] (h) [right of=g] {};


      \draw (a) --(b) -- (c) -- (d) -- (e);
      \draw[dashed] (e) -- (f);
      \draw (f) -- (g);
      \draw[dashed] (g)-- (h);

      \path[every node/.style={font=\sffamily\small}]
      (a) edge[bend left] node {} (c);

      \path[every node/.style={font=\sffamily\small}]
      (g) edge[bend left] node {} (e)
      (e) edge[bend left] node {} (b)
      (b) edge[bend left] node {} (f);

      \path[every node/.style={font=\sffamily\small}]
      (d) edge[bend left] node {} (f)
      (f) edge[bend left] node {} (h)
      (h) edge[bend right] node {} (e)
      (e) edge[bend left] node {} (c)
      (c) edge[bend left] node {} (f);
  \end{tikzpicture}}

  \only<2>{
    \centering
    \begin{tikzpicture}[auto,node distance=1cm,thick]
      \clip (-.5,.8) rectangle (7.5,-.8);
      \node[draw=black,fill=blue,circle,inner sep=2pt] (a) {};
      \node[draw=black,fill=\mycyan,circle,inner sep=2pt] (b) [right of=a] {};
      \node[draw=black,fill=\myorange,circle,inner sep=2pt] (c) [right of=b] {};
      \node[draw=black,very thick,fill=\mycyan,circle,inner sep=2pt] (d) [right of=c] {};
      \node[draw=black,fill=magenta,circle,inner sep=2pt] (e) [right of=d] {};
      \node[draw=black,fill=magenta,circle,inner sep=2pt] (f) [right of=e] {};
      \node[draw=black,fill=\myorange,circle,inner sep=2pt] (g) [right of=f] {};
      \node[draw=black,fill=\myorange,circle,inner sep=2pt] (h) [right of=g] {};


      \draw (a) --(b);
      \draw (b) -- (c) -- (d) -- (e);
      \draw[dashed] (e) -- (f);
      \draw (f) -- (g);
      \draw[dashed] (g)-- (h);

      \path[every node/.style={font=\sffamily\small}]
      (a) edge[bend left] node {} (c);

      \path[every node/.style={font=\sffamily\small}]
      (g) edge[bend left] node {} (e)
      (e) edge[bend left] node {} (b)
      (b) edge[bend left] node {} (f);

      \path[every node/.style={font=\sffamily\small}]
      (d) edge[bend left] node {} (f)
      (f) edge[bend left] node {} (h)
      (h) edge[bend right] node {} (e)
      (e) edge[bend left] node {} (c)
      (c) edge[bend left] node {} (f);
  \end{tikzpicture}}

  \only<3>{
    \centering
    \begin{tikzpicture}[auto,node distance=1cm,thick]
      \clip (-.5,.8) rectangle (7.5,-.8);
      \node[draw=black,very thick,fill=blue,circle,inner sep=2pt] (a) {};
      \node[draw=black,very thick,fill=\mycyan,circle,inner sep=2pt] (b) [right of=a] {};
      \node[draw=black,fill=\myorange,circle,inner sep=2pt] (c) [right of=b] {};
      \node[draw=black,fill=blue,circle,inner sep=2pt] (d) [right of=c] {};
      \node[draw=black,fill=magenta,circle,inner sep=2pt] (e) [right of=d] {};
      \node[draw=black,fill=magenta,circle,inner sep=2pt] (f) [right of=e] {};
      \node[draw=black,fill=\myorange,circle,inner sep=2pt] (g) [right of=f] {};
      \node[draw=black,fill=\myorange,circle,inner sep=2pt] (h) [right of=g] {};


      \draw[very thick] (a) --(b);
      \draw (b) -- (c) -- (d) -- (e);
      \draw[dashed] (e) -- (f);
      \draw (f) -- (g);
      \draw[dashed] (g)-- (h);

      \path[every node/.style={font=\sffamily\small}]
      (a) edge[bend left] node {} (c);

      \path[every node/.style={font=\sffamily\small}]
      (g) edge[bend left] node {} (e)
      (e) edge[bend left] node {} (b)
      (b) edge[bend left] node {} (f);

      \path[every node/.style={font=\sffamily\small}]
      (d) edge[bend left] node {} (f)
      (f) edge[bend left] node {} (h)
      (h) edge[bend right] node {} (e)
      (e) edge[bend left] node {} (c)
      (c) edge[bend left] node {} (f);
  \end{tikzpicture}}

    \only<4>{
    \centering
    \begin{tikzpicture}[auto,node distance=1cm,thick]
      \clip (-.5,.8) rectangle (7.5,-.8);
      \node[draw=black,fill=\mycyan,circle,inner sep=2pt] (a) {};
      \node[draw=black,fill=blue,circle,inner sep=2pt] (b) [right of=a] {};
      \node[draw=black,fill=\myorange,circle,inner sep=2pt] (c) [right of=b] {};
      \node[draw=black,fill=blue,circle,inner sep=2pt] (d) [right of=c] {};
      \node[draw=black,very thick,fill=magenta,circle,inner sep=2pt] (e) [right of=d] {};
      \node[draw=black,fill=magenta,circle,inner sep=2pt] (f) [right of=e] {};
      \node[draw=black,fill=\myorange,circle,inner sep=2pt] (g) [right of=f] {};
      \node[draw=black,fill=\myorange,circle,inner sep=2pt] (h) [right of=g] {};


      \draw (a) --(b) -- (c) -- (d) -- (e);
      \draw[dashed] (e) -- (f);
      \draw (f) -- (g);
      \draw[dashed] (g)-- (h);

      \path[every node/.style={font=\sffamily\small}]
      (a) edge[bend left] node {} (c);

      \path[every node/.style={font=\sffamily\small}]
      (g) edge[bend left] node {} (e)
      (e) edge[bend left] node {} (b)
      (b) edge[bend left] node {} (f);

      \path[every node/.style={font=\sffamily\small}]
      (d) edge[bend left] node {} (f)
      (f) edge[bend left] node {} (h)
      (h) edge[bend right] node {} (e)
      (e) edge[bend left] node {} (c)
      (c) edge[bend left] node {} (f);
  \end{tikzpicture}}

  \only<5>{
    \centering
    \begin{tikzpicture}[auto,node distance=1cm,thick]
      \clip (-.5,.8) rectangle (7.5,-.8);
      \node[draw=black,fill=\mycyan,circle,inner sep=2pt] (a) {};
      \node[draw=black,fill=blue,circle,inner sep=2pt] (b) [right of=a] {};
      \node[draw=black,fill=\myorange,circle,inner sep=2pt] (c) [right of=b] {};
      \node[draw=black,fill=blue,circle,inner sep=2pt] (d) [right of=c] {};
      \node[draw=black,fill=\mycyan,circle,inner sep=2pt] (e) [right of=d] {};
      \node[draw=black,fill=magenta,circle,inner sep=2pt] (f) [right of=e] {};
      \node[draw=black,very thick,fill=\myorange,circle,inner sep=2pt] (g) [right of=f] {};
      \node[draw=black,fill=\myorange,circle,inner sep=2pt] (h) [right of=g] {};


      \draw (a) --(b) -- (c) -- (d) -- (e) -- (f) -- (g);
      \draw[dashed] (g)-- (h);

      \path[every node/.style={font=\sffamily\small}]
      (a) edge[bend left] node {} (c);

      \path[every node/.style={font=\sffamily\small}]
      (g) edge[bend left] node {} (e)
      (e) edge[bend left] node {} (b)
      (b) edge[bend left] node {} (f);

      \path[every node/.style={font=\sffamily\small}]
      (d) edge[bend left] node {} (f)
      (f) edge[bend left] node {} (h)
      (h) edge[bend right] node {} (e)
      (e) edge[bend left] node {} (c)
      (c) edge[bend left] node {} (f);
  \end{tikzpicture}}

    
  \only<6->{
    \centering
    \begin{tikzpicture}[auto,node distance=1cm,thick]
      \clip (-.5,.8) rectangle (7.5,-.8);
      \node[draw=black,fill=\mycyan,circle,inner sep=2pt] (a) {};
      \node[draw=black,fill=blue,circle,inner sep=2pt] (b) [right of=a] {};
      \node[draw=black,fill=\myorange,circle,inner sep=2pt] (c) [right of=b] {};
      \node[draw=black,fill=blue,circle,inner sep=2pt] (d) [right of=c] {};
      \node[draw=black,fill=\mycyan,circle,inner sep=2pt] (e) [right of=d] {};
      \node[draw=black,fill=magenta,circle,inner sep=2pt] (f) [right of=e] {};
      \node[draw=black,fill=blue,circle,inner sep=2pt] (g) [right of=f] {};
      \node[draw=black,fill=\myorange,circle,inner sep=2pt] (h) [right of=g] {};


      \draw (a) --(b) -- (c) -- (d) -- (e) -- (f) -- (g) -- (h);

      \path[every node/.style={font=\sffamily\small}]
      (a) edge[bend left] node {} (c);

      \path[every node/.style={font=\sffamily\small}]
      (g) edge[bend left] node {} (e)
      (e) edge[bend left] node {} (b)
      (b) edge[bend left] node {} (f);

      \path[every node/.style={font=\sffamily\small}]
      (d) edge[bend left] node {} (f)
      (f) edge[bend left] node {} (h)
      (h) edge[bend right] node {} (e)
      (e) edge[bend left] node {} (c)
      (c) edge[bend left] node {} (f);
  \end{tikzpicture}}

  \uncover<7>{
  \begin{description}
  \item[Length of the sequence] $(\tw n + n + \tw n) \times n = O(\tw n^2)$
  \end{description}}
\end{frame}

\begin{frame}{Further improvements of Las Vergnas and Meyniel's lemma}
  \begin{block}{Bonamy, Delecroix, L. '21}
    $G$ $d$-degenerate, with all vertices
    but one of degree at most $d+1$, then for $k \ge d+1$, $\diam(R^k(G)) = O(dn^2)$
  \end{block}

  \begin{alertblock}{Open question}
    Can we bound the diameter without making additional assumptions ?
  \end{alertblock}
\end{frame}
%% %% \backupbegin

%% %% \backupend

\end{document}
