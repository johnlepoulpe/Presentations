\documentclass[10pt,xcolor=dvipsnames,presentation,aspectratio=169]{beamer}

\usepackage{BeamerColor}
\usepackage{etoolbox}

\usepackage[T1]{fontenc}%
\usepackage[utf8]{inputenc}%
\usepackage[main= english,francais]{babel}%
\usepackage{mathtools}
\usepackage{amssymb}


\usepackage{graphicx}%
\usepackage{courier}%
\usepackage{algorithm2e}

\usepackage{url}%
\urlstyle{sf}%

\usepackage{subcaption}%
\usepackage{caption}
\captionsetup[figure]{labelformat=empty}% redefines the caption setup of the figures environment in the beamer class.
\usepackage{tikz}
\usetikzlibrary{decorations.pathreplacing}
\tikzset{%
  show curve controls/.style={
    postaction={
      decoration={
        show path construction,
        curveto code={
          \draw [blue] 
            (\tikzinputsegmentfirst) -- (\tikzinputsegmentsupporta)
            (\tikzinputsegmentlast) -- (\tikzinputsegmentsupportb);
          \fill [red, opacity=0.5] 
            (\tikzinputsegmentsupporta) circle [radius=.5ex]
            (\tikzinputsegmentsupportb) circle [radius=.5ex];
        }
      },
      decorate
}}}


\usepackage{listings}%

\usepackage{faktor}%
\usepackage{cancel}%
\usepackage{stmaryrd}%

\usepackage[style=verbose,backend=biber]{biblatex}
\addbibresource{biblio.bib}

\newcommand\blfootnote[1]{%
  \begingroup
  \renewcommand\thefootnote{}\footnote{#1}%
  \addtocounter{footnote}{-1}%
  \endgroup
}

%%%%%%%%% Colors of Beamer Layout %%%%%%%%%%%%%
\newcommand{\myorange}{Orange}
\newcommand{\mygreen}{DarkGreen}
\newcommand{\mycyan}{LightSeaGreen}
\newcommand{\myred}{Fuchsia}
\newcommand{\myyellow}{Goldenrod}
\newcommand{\myblue}{RoyalBlue}
\newcommand{\mydarkblue}{MidnightBlue}


%%%%%%%%% Begin of Beamer Layout %%%%%%%%%%%%%
\ProcessOptionsBeamer
\usecolortheme{whale}
\usecolortheme[named=\myblue]{structure}
\useinnertheme{rounded}
\useoutertheme{infolines}


\setbeamertemplate{footline}[frame number]
\setbeamertemplate{headline}[default]
\setbeamertemplate{navigation symbols}{}
\setbeamertemplate{itemize items}[circle]
\defbeamertemplate*{headline}{info theme}{}
\defbeamertemplate*{footline}{info theme}{\leavevmode%


  \hbox{%
\begin{beamercolorbox}[wd=.2\paperwidth,ht=2.25ex,dp=1ex,center]{author in head/foot}%
\usebeamerfont{author in head/foot}\insertshortauthor
\end{beamercolorbox}%
\begin{beamercolorbox}[wd=.71\paperwidth,ht=2.25ex,dp=1ex,center]{title in head/foot}%
\usebeamerfont{title in head/foot}\insertsectionhead
\end{beamercolorbox}%
\begin{beamercolorbox}[wd=.09\paperwidth,ht=2.25ex,dp=1ex,right]{section in head/foot}%
\usebeamerfont{section in head/foot}\insertframenumber{}~/~\inserttotalframenumber\hspace*{2ex}
\end{beamercolorbox}
}\vskip0pt}
\setbeamercolor*{block title example}{fg=\myorange}
\AtBeginEnvironment{exampleblock}{\setbeamercolor{itemize item}{fg=\myorange}}
\setbeamercolor*{block title alerted}{fg=\myred}
\AtBeginEnvironment{alertblock}{\setbeamercolor{itemize item}{fg=\myred}}
\setbeamertemplate{footline}[info theme]
\defbeamertemplate{description item}{align left}{\insertdescriptionitem\hfill}

%%%%%%%%% End of Beamer Layout %%%%%%%%%%%%%





\lstset{%
	basicstyle=\sffamily,%
	columns=fullflexible,%
	language=C++,% % À ajuster dans chaque projet!
	frame=lb,%
	frameround=fftf,%
}%

\sloppy

%\setbeamertemplate{headline}{}%

\setbeamertemplate{navigation symbols}{%
	\insertslidenavigationsymbol%
	% \insertframenavigationsymbol%
	% \insertsubsectionnavigationsymbol%
	% \insertsectionnavigationsymbol%
	% \insertdocnavigationsymbol%
	% \insertbackfindforwardnavigationsymbol%
}

\newcommand{\backupbegin}{
   \newcounter{finalframe}
   \setcounter{finalframe}{\value{framenumber}}
}
\newcommand{\backupend}{
   \setcounter{framenumber}{\value{finalframe}}
}

\newcommand{\semitransp}[2][50]{\textcolor{fg!#1}{#2}}
\newcommand{\hide}[3]{\only<#1>{#3}\only<#2>{\semitransp{#3}}}

\DeclareMathOperator{\diam}{diam}
\DeclareMathOperator{\tw}{tw}
\DeclareMathOperator{\mad}{mad}

\begin{document}

\title{Exploration de l'espace des colorations d'un graphe}
\author[Clément Legrand]{Clément Legrand-Duchesne}
\institute{LaBRI, Université de Bordeaux}
\date{\today}

\begin{frame}
  \titlepage

  \centering
\end{frame}

\begin{frame}{Reconfiguration de colorations}

  \begin{columns}
    \begin{column}{.66\textwidth}
      \begin{exampleblock}{Chaine de Kempe (1879)}
        Composante bichromatique connexe maximale dans $G$
      \end{exampleblock}
    \end{column}
    \begin{column}{.33\linewidth}
      \centering
      \begin{tikzpicture}
        \clip (-1,1) rectangle (1,-1);
        \node[draw=black,fill=\myorange,circle,inner sep=2pt]   (a) at (90:.8cm) {};
        \node[draw=black,fill=\myyellow,circle,inner sep=2pt]  (b) at (162:.8cm) {};
        \node[draw=black,fill=\myred,circle,inner sep=2pt] (c) at (234:.8cm) {};
        \node[draw=black,fill=\myblue,circle,inner sep=2pt] (d) at (306:.8cm) {};
        \node[draw=black,fill=\myred,circle,inner sep=2pt] (e) at (18:.8cm) {};

        \draw (a) -- (b) -- (c) -- (d) -- (e) -- (a);
      \end{tikzpicture}
      
      %\includegraphics[width=.8\textwidth]{../figures/pdf/knot_example.pdf}
    \end{column}
  \end{columns}

  \begin{columns}
    \begin{column}{.33\linewidth}
      \centering
      \begin{tikzpicture}
        \clip (-1,1) rectangle (1,-1);
        \node[draw=black,fill=\myorange,circle,inner sep=2pt]   (a) at (90:.8cm) {};
        \node[draw=black,fill=\myyellow,circle,inner sep=2pt]  (b) at (162:.8cm) {};
        \node[draw=black,very thick,fill=\myblue,circle,inner sep=2pt] (c) at (234:.8cm) {};
        \node[draw=black,very thick,fill=\myred,circle,inner sep=2pt] (d) at (306:.8cm) {};
        \node[draw=black,very thick,fill=\myblue,circle,inner sep=2pt] (e) at (18:.8cm) {};

        \draw (e) -- (a) -- (b) -- (c);
        \draw[very thick] (c) -- (d) -- (e);
      \end{tikzpicture}
    \end{column}
    \hfill
    \begin{column}{.33\linewidth}
      \centering
      \begin{tikzpicture}
        \clip (-1,1) rectangle (1,-1);
        \node[draw=black,fill=\myorange,circle,inner sep=2pt]   (a) at (90:.8cm) {};
        \node[draw=black,very thick,fill=\myred,circle,inner sep=2pt]  (b) at (162:.8cm) {};
        \node[draw=black,very thick,fill=\myyellow,circle,inner sep=2pt] (c) at (234:.8cm) {};
        \node[draw=black,fill=\myblue,circle,inner sep=2pt] (d) at (306:.8cm) {};
        \node[draw=black,fill=\myred,circle,inner sep=2pt] (e) at (18:.8cm) {};
        
        \draw[very thick] (b) -- (c);
        \draw (a) -- (b);
        \draw (c) -- (d) -- (e) -- (a);
      \end{tikzpicture}
    \end{column}
    \hfill
    \begin{column}{.33\linewidth}
      \centering
      \begin{tikzpicture}
        
        \clip (-1,1) rectangle (1,-1);
        \node[draw=black,fill=\myorange,circle,inner sep=2pt]   (a) at (90:.8cm) {};
        \node[draw=black,fill=\myyellow,circle,inner sep=2pt]  (b) at (162:.8cm) {};
        \node[draw=black,fill=\myred,circle,inner sep=2pt] (c) at (234:.8cm) {};
        \node[draw=black,very thick,fill=\myyellow,circle,inner sep=2pt] (d) at (306:.8cm) {};
        \node[draw=black,fill=\myred,circle,inner sep=2pt] (e) at (18:.8cm) {};
        
        \draw (a) -- (b) -- (c) -- (d) -- (e) -- (a);
      \end{tikzpicture}
    \end{column}
  \end{columns}

  \uncover<3>{
  \begin{block}{Questions}
    \begin{itemize}
    \item Connexité et Diamètre du graphe de reconfiguration
    \item Temps de mélange de la chaine de Markov associée
    \item Dénombrement des colorations
    \item Énumération efficace des colorations
    \end{itemize}
  \end{block}}
\end{frame}

\begin{frame}{Publications}
  \begin{block}{Publications}
    \begin{itemize}
    \item The structure of quasi-transitive graphs avoiding a minor with
      applications to the domino problem, Eurocomb23, soumis à
      JCTb.\\
      Avec Louis Esperet et Ugo Giocanti.
    \item \semitransp{Strengthening a theorem of Meyniel, SIDMA.\\ Avec Quentin Deschamps, Carl
      Feghali, František Kardoš et Théo Pierron.}
    \item \semitransp{Recoloring version of Hadwiger's conjecture, Eurocomb23, soumis à
      JCTb.\\ Avec Marthe Bonamy, Marc Heinrich et Jonathan Narboni.}
    \item \semitransp{Kempe changes in degenerate graphs, European Journal of
      Combinatorics.\\ Avec Marthe Bonamy et Vincent Delecroix.}
    \item \semitransp{Kempe changes in bounded treewidth graphs, EuroComb 2021\\ Avec Marthe
      Bonamy and Vincent Delecroix.}
    \item \semitransp{Parameterized complexity of untangling knots, ICALP 2022, soumis à
      SICOMP.\\ Avec Ashutosh Rai et Martin Tancer.}
    \end{itemize}
  \end{block}
\end{frame}

\begin{frame}{Collaborations}
  \begin{block}{Visites}
    \begin{itemize}
    \item Juin: 1 semaine a Bristol avec Luke Jeffreys
    \item Novembre: 1 semaine à Beyrouth avec Amer Mouawad, Clément Dallard, Quentin Deschamps,
      Amadeus Reinald
    \end{itemize}
  \end{block}
 \vspace{-.2cm}
  \uncover<2>{
  \begin{block}{Conférences, Workshops}
    \begin{itemize}
    \item Juin: International Hiking Workshop for Young Combinatorialists
    \item Mars: Journées reconfiguration et bases de données
    \item Févier: Sage days 117
    \item Novembre: Journées Graphes et Algorithmique (exposé)
    \item Novembre: séminaire bordelaisJournées ALEA (exposé)
    \item Juillet: ICALP 2022 (exposé)
    \end{itemize}
  \end{block}}
\end{frame}

\begin{frame}{Projets en cours}
  \begin{itemize}
  \item \hide12{Version recoloration de la conjecture de Reed, avec Marthe
      Bonamy et Tomas Kaiser}
  \item \hide12{Reconstruction de Polytope, avec Jane Tan}
  \item \hide12{Coloration simultanée d'arètes, avec Amedeo Sgueglia}
  \item \hide12{Kempe mixing chain, avec Dan Cranston}
  \item \hide12{Reconfiguration processeur quantique, avec Clément Dallard, Quentin
    Deschamps, Amer Mouawad et Amadeus Reinald}
  \uncover<2>{
  \item Reconfiguration de quadrangulations, avec Vincent Delecroix
  \item Reconfiguration de $(s,t)$-séparateur, projet BANFF
  \item Soft Meyniel conjecture, avec Paul Bastide, Marthe Bonamy et Jane Tan
  \item Recoloration de cographes, avec Marthe Bonamy
  \item Version recoloration du théorème de Grötzsch}
  \end{itemize}
\end{frame}

\begin{frame}{Implication dans la communauté}
  \begin{block}{Reviews}
    \begin{itemize}
    \item Switching 3-Edge colorings of Cubic Graphs (Discrete Mathematics)
    \item Triangle-free planar graphs with at most $64^{n^{0.731}}$ 3-colorings (JCTb)
    \item List-Recoloring of Sparse Graphs (European Journal of Combinatorics)
    \item Kempe Equivalent List Colorings (Combinatorica)
    \item Optimally Reconfiguring List and Correspondence Colourings (European
      Symposium on Algorithms)
    \end{itemize}
  \end{block}

  \uncover<2->{
  \begin{block}{Co-organisateur du séminaire bordelais}
    Avec Marthe Bonamy et Claire Hilaire, depuis septembre
  \end{block}

  \begin{block}{Co-organisateur du groupe de lecture pour jeunes grapheux
      francophones}
    Avec Paul Bastide. Nouveau groupe de lecture, première séance prévu le 22 mai
  \end{block}}

  \uncover<3>{
  \begin{block}{Encadrement d'un stagiaire L3}
  \end{block}}
  
\end{frame}

\end{document}
