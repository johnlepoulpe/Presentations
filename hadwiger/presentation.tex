\documentclass[11pt,xcolor=dvipsnames,presentation,aspectratio=169]{beamer}

\usepackage{BeamerColor}
\usepackage{etoolbox}

\usepackage[T1]{fontenc}%
\usepackage[utf8]{inputenc}%
\usepackage[main= english,francais]{babel}%
\usepackage{mathtools}

\usepackage{graphicx}%
\usepackage{courier}%
\usepackage{algorithm2e}

\usepackage{url}%
\urlstyle{sf}%

\usepackage{subcaption}%
\usepackage{caption}
\captionsetup[figure]{labelformat=empty}% redefines the caption setup of the figures environment in the beamer class.
\usepackage{tikz}

\usepackage{listings}%

\usepackage{faktor}%

\usepackage{stmaryrd}%

\usepackage[style=verbose,backend=biber]{biblatex}
\addbibresource{biblio.bib}

\newcommand\blfootnote[1]{%
  \begingroup
  \renewcommand\thefootnote{}\footnote{#1}%
  \addtocounter{footnote}{-1}%
  \endgroup
}

%%%%%%%%% Colors of Beamer Layout %%%%%%%%%%%%%
\newcommand{\myorange}{Orange}
\newcommand{\mygreen}{DarkGreen}
\newcommand{\mycyan}{LightSeaGreen}
\newcommand{\myred}{Sepia}
\newcommand{\myyellow}{Goldenrod}
\newcommand{\myblue}{RoyalBlue}
\newcommand{\mydarkblue}{MidnightBlue}



%%%%%%%%% Begin of Beamer Layout %%%%%%%%%%%%%
\ProcessOptionsBeamer
\usecolortheme{whale}
\usecolortheme[named=\myblue]{structure}
\useinnertheme{rounded}
\useoutertheme{infolines}
\setbeamertemplate{footline}[frame number]
\setbeamertemplate{headline}[default]
\setbeamertemplate{navigation symbols}{}
\setbeamertemplate{itemize items}[circle]
\defbeamertemplate*{headline}{info theme}{}
\defbeamertemplate*{footline}{info theme}{\leavevmode%
\hbox{%
\begin{beamercolorbox}[wd=.2\paperwidth,ht=2.25ex,dp=1ex,center]{author in head/foot}%
\usebeamerfont{author in head/foot}\insertshortauthor
\end{beamercolorbox}%
\begin{beamercolorbox}[wd=.71\paperwidth,ht=2.25ex,dp=1ex,center]{title in head/foot}%
\usebeamerfont{title in head/foot}\insertsectionhead
\end{beamercolorbox}%
\begin{beamercolorbox}[wd=.09\paperwidth,ht=2.25ex,dp=1ex,right]{section in head/foot}%
\usebeamerfont{section in head/foot}\insertframenumber{}~/~\inserttotalframenumber\hspace*{2ex}
\end{beamercolorbox}
}\vskip0pt}
\setbeamercolor*{block title example}{fg=\myorange}
\AtBeginEnvironment{exampleblock}{\setbeamercolor{itemize item}{fg=\myorange}}
\setbeamercolor*{block title alerted}{fg=\myred}
\AtBeginEnvironment{alertblock}{\setbeamercolor{itemize item}{fg=\myred}}
\setbeamertemplate{footline}[info theme]
\defbeamertemplate{description item}{align left}{\insertdescriptionitem\hfill}
%%%%%%%%% End of Beamer Layout %%%%%%%%%%%%%

\lstset{%
	basicstyle=\sffamily,%
	columns=fullflexible,%
	language=C++,% % À ajuster dans chaque projet!
	frame=lb,%
	frameround=fftf,%
}%

\sloppy

%\setbeamertemplate{headline}{}%

\setbeamertemplate{navigation symbols}{%
	\insertslidenavigationsymbol%
	% \insertframenavigationsymbol%
	% \insertsubsectionnavigationsymbol%
	% \insertsectionnavigationsymbol%
	% \insertdocnavigationsymbol%
	% \insertbackfindforwardnavigationsymbol%
}

\newcommand{\backupbegin}{
   \newcounter{finalframe}
   \setcounter{finalframe}{\value{framenumber}}
}
\newcommand{\backupend}{
   \setcounter{framenumber}{\value{finalframe}}
}

\DeclareMathOperator{\diam}{diam}
\DeclareMathOperator{\tw}{tw}

\begin{document}

\title{Kempe recoloring version of Hadwiger's conjecture}
\author[Clément Legrand]{Clément Legrand-Duchesne}
\institute{LaBRI, Bordeaux}
%\date{\today}
\date{August 31, 2023}

\begin{frame}
  \titlepage

  \centering
  Joint work with Marthe Bonamy, Marc Heinrich and Jonathan Narboni
\end{frame}

\section*{Some context}
% \begin{frame}{Recoloring with Kempe changes}
%   \uncover<1->{
%   \begin{columns}
%     \begin{column}{.66\textwidth}
%       \begin{exampleblock}{Kempe chain (1879)}
%         Maximal bichromatic connected component in $G$
%       \end{exampleblock}
%     \end{column}
%     \begin{column}{.33\textwidth}
%       \centering
%       \begin{tikzpicture}
%         \node[draw=black,fill=\myorange,circle,inner sep=2pt]   (a) at (90:.8cm) {};
%         \node[draw=black,fill=\mycyan,circle,inner sep=2pt]  (b) at (162:.8cm) {};
%         \node[draw=black,fill=magenta,circle,inner sep=2pt] (c) at (234:.8cm) {};
%         \node[draw=black,fill=blue,circle,inner sep=2pt] (d) at (306:.8cm) {};
%         \node[draw=black,fill=magenta,circle,inner sep=2pt] (e) at (18:.8cm) {};

%         \draw (a) -- (b) -- (c) -- (d) -- (e) -- (a);
%       \end{tikzpicture}
      
%       %\includegraphics[width=.8\textwidth]{../figures/pdf/knot_example.pdf}
%     \end{column}
%   \end{columns}
%   }

%   \vspace{1.5cm}
  
%   \begin{columns}
%     \uncover<2->{
%       \begin{column}{.33\textwidth}
%         \centering
%         \begin{tikzpicture}
%           \node[draw=black,fill=\myorange,circle,inner sep=2pt]   (a) at (90:.8cm) {};
%           \node[draw=black,fill=\mycyan,circle,inner sep=2pt]  (b) at (162:.8cm) {};
%           \node[draw=black,very thick,fill=blue,circle,inner sep=2pt] (c) at (234:.8cm) {};
%           \node[draw=black,very thick,fill=magenta,circle,inner sep=2pt] (d) at (306:.8cm) {};
%           \node[draw=black,very thick,fill=blue,circle,inner sep=2pt] (e) at (18:.8cm) {};

%           \draw (e) -- (a) -- (b) -- (c);
%           \draw[very thick] (c) -- (d) -- (e);
%         \end{tikzpicture}      
%     \end{column}
%     }
%     \uncover<3->{
%     \begin{column}{.33\textwidth}
%        \centering
%       \begin{tikzpicture}
%         \node[draw=black,fill=\myorange,circle,inner sep=2pt]   (a) at (90:.8cm) {};
%         \node[draw=black,very thick,fill=magenta,circle,inner sep=2pt]  (b) at (162:.8cm) {};
%         \node[draw=black,very thick,fill=\mycyan,circle,inner sep=2pt] (c) at (234:.8cm) {};
%         \node[draw=black,fill=blue,circle,inner sep=2pt] (d) at (306:.8cm) {};
%         \node[draw=black,fill=magenta,circle,inner sep=2pt] (e) at (18:.8cm) {};

%         \draw[very thick] (b) -- (c);
%         \draw (a) -- (b);
%         \draw (c) -- (d) -- (e) -- (a);
%       \end{tikzpicture}
%     \end{column}
%     }
%     \uncover<4->{
%     \begin{column}{.33\textwidth}
%        \centering
%       \begin{tikzpicture}
%         \node[draw=black,fill=\myorange,circle,inner sep=2pt]   (a) at (90:.8cm) {};
%         \node[draw=black,fill=\mycyan,circle,inner sep=2pt]  (b) at (162:.8cm) {};
%         \node[draw=black,fill=magenta,circle,inner sep=2pt] (c) at (234:.8cm) {};
%         \node[draw=black,very thick,fill=\mycyan,circle,inner sep=2pt] (d) at (306:.8cm) {};
%         \node[draw=black,fill=magenta,circle,inner sep=2pt] (e) at (18:.8cm) {};

%         \draw (a) -- (b) -- (c) -- (d) -- (e) -- (a);
%       \end{tikzpicture}
%     \end{column}
%     }
%   \end{columns}
% \end{frame}

\section*{Some context}
\begin{frame}{Recoloring with Kempe changes}

  \begin{columns}
    \begin{column}{.55\linewidth}
      \begin{exampleblock}{Kempe chain (1879)}
        Maximal bichromatic connected component in $G$
      \end{exampleblock}
    \end{column}
    \begin{column}{.2\linewidth}
      \centering
      \begin{tikzpicture}
        \clip (-1,1) rectangle (1,-1);
        \node[draw=black,fill=\myorange,circle,inner sep=2pt]   (a) at (90:.8cm) {};
        \node[draw=black,fill=\myyellow,circle,inner sep=2pt]  (b) at (162:.8cm) {};
        \node[draw=black,fill=\myred,circle,inner sep=2pt] (c) at (234:.8cm) {};
        \node[draw=black,fill=\myblue,circle,inner sep=2pt] (d) at (306:.8cm) {};
        \node[draw=black,fill=\myred,circle,inner sep=2pt] (e) at (18:.8cm) {};

        \draw (a) -- (b) -- (c) -- (d) -- (e) -- (a);
      \end{tikzpicture}
      
      %\includegraphics[width=.8\textwidth]{../figures/pdf/knot_example.pdf}
    \end{column}

    \begin{column}{.2\linewidth}
      \centering
      \begin{tikzpicture}
        \only<1>{
          \clip (-1,1) rectangle (1,-1);
          \node[draw=black,fill=\myorange,circle,inner sep=2pt]   (a) at (90:.8cm) {};
          \node[draw=black,fill=\myyellow,circle,inner sep=2pt]  (b) at (162:.8cm) {};
          \node[draw=black,very thick,fill=\myblue,circle,inner sep=2pt] (c) at (234:.8cm) {};
          \node[draw=black,very thick,fill=\myred,circle,inner sep=2pt] (d) at (306:.8cm) {};
          \node[draw=black,very thick,fill=\myblue,circle,inner sep=2pt] (e) at (18:.8cm) {};

          \draw (e) -- (a) -- (b) -- (c);
          \draw[very thick] (c) -- (d) -- (e);
        }

        \only<2>{
          \clip (-1,1) rectangle (1,-1);
          \node[draw=black,fill=\myorange,circle,inner sep=2pt]   (a) at (90:.8cm) {};
          \node[draw=black,very thick,fill=\myred,circle,inner sep=2pt]  (b) at (162:.8cm) {};
          \node[draw=black,very thick,fill=\myyellow,circle,inner sep=2pt] (c) at (234:.8cm) {};
          \node[draw=black,fill=\myblue,circle,inner sep=2pt] (d) at (306:.8cm) {};
          \node[draw=black,fill=\myred,circle,inner sep=2pt] (e) at (18:.8cm) {};
          
          \draw[very thick] (b) -- (c);
          \draw (a) -- (b);
          \draw (c) -- (d) -- (e) -- (a);
        }

        \only<3->{
          \clip (-1,1) rectangle (1,-1);
          \node[draw=black,fill=\myorange,circle,inner sep=2pt]   (a) at (90:.8cm) {};
          \node[draw=black,fill=\myyellow,circle,inner sep=2pt]  (b) at (162:.8cm) {};
          \node[draw=black,fill=\myred,circle,inner sep=2pt] (c) at (234:.8cm) {};
          \node[draw=black,very thick,fill=\myyellow,circle,inner sep=2pt] (d) at (306:.8cm) {};
          \node[draw=black,fill=\myred,circle,inner sep=2pt] (e) at (18:.8cm) {};
          
          \draw (a) -- (b) -- (c) -- (d) -- (e) -- (a);
        }
      \end{tikzpicture}
    \end{column}
  \end{columns}
  
  % \uncover<4->{
  % \begin{exampleblock}{Reconfiguration graph}
  %   \begin{itemize}
  %   \item $V(R^k(G)) =$ $k$-colorings of $G$.
  %   \item $\alpha$ and $\beta$ adjacent  if $\alpha
  %     \xleftrightarrow[\text{Kempe}]{} \beta$
  %   \end{itemize}
  % \end{exampleblock}}
  % \vspace{-.3cm}
  \uncover<4->{
  \begin{alertblock}{Usual questions}
    \begin{itemize}
    \item Are any two $k$-colorings of a graph $G$ equivalent ?\\
      Are all $k$-colorings equivalent to a $\chi(G)$-coloring ?
    \item How many Kempe changes separate any two $k$-colorings ?
    \item Application to sampling : Does the corresponding Markov chain mix well ?
    \end{itemize}
  \end{alertblock}}

\end{frame}

% \begin{frame}{Natural questions in reconfiguration}
%   \begin{exampleblock}{Reconfiguration graph}
%     \begin{itemize}
%     \item $V(R^k(G)) =$ $k$-colorings of $G$.
%     \item $\alpha$ and $\beta$ adjacent  if $\alpha
%       \xleftrightarrow[\text{Kempe}]{} \beta$
%     \end{itemize}
%   \end{exampleblock}

%   \setbeamertemplate{description item}[align left] 
%   \begin{description}
%   \item[Reachability] Are $\alpha$ and $\beta$ Kempe-equivalent ? 
%   \item[Connectivity] Is $R^k(G)$ connected ?
%   \item[Diameter] Estimate $\diam(R^k(G))$ ?
%   \end{description}

%   \begin{block}{Application}
%     \begin{itemize}
%     \item Powerful tool (ex: Vizing theorem)
%     \item Sampling coloring with a Markov chain
%     \end{itemize}
%   \end{block}
% \end{frame}

\begin{frame}{Graph minor}
  \begin{exampleblock}{Graph minor}
    $H$ is a minor of $G$ if $H$ can be obtained be deleting vertices, edges and
    contracting edges of $G$
  \end{exampleblock}

  \begin{columns}
    \begin{column}{.48\textwidth}
      \centering
      \begin{tikzpicture}
        \node[draw=\myorange,fill=\myorange,circle,inner sep=1.5pt]   (a1) at (75:.8cm) {};
        \node[draw=\myorange,fill=\myorange,circle,inner sep=1.5pt]   (a2) at (105:.8cm) {};
        \node[draw=\myorange,fill=\myorange,circle,inner sep=1.5pt]   (a) at (90:.45cm) {};
        \node[draw=\myblue,fill=\myblue,circle,inner sep=1.5pt]  (b) at (147:.8cm) {};
        \node[draw=\myblue,fill=\myblue,circle,inner sep=1.5] (b2) at   (177:.8cm) {};
        \node[draw=\myred,fill=\myred,circle,inner sep=1.5pt] (c) at (234:.8cm) {};
        \node[draw=\mydarkblue,fill=\mydarkblue,circle,inner sep=1.5pt] (d) at (291:.8cm) {};
        \node[draw=\mydarkblue,fill=\mydarkblue,circle,inner sep=1.5pt] (d2) at (321:.8cm) {};
        \node[draw=\mydarkblue,fill=\mydarkblue,circle,inner sep=1.5pt] (d3) at (306:.45cm) {};
        \node[draw=yellow,fill=yellow,circle,inner sep=1.5pt] (e) at (18:.8cm) {};

        \draw (a2) -- (b) -- (d3) (e) -- (b2) -- (c) -- (a2) (a1) -- (e) -- (d2)
        (d3) -- (a) (e) -- (c) -- (d); \draw[very thick,\myblue] (b) -- (b2);
        \draw[very thick,\mydarkblue] (d) -- (d2) -- (d3); \draw[very thick,\myorange]
        (a) -- (a1) -- (a2);
      \end{tikzpicture}
    \end{column}
    \hfill
    \begin{column}{.48\textwidth}
      \centering
      \begin{tikzpicture}
        \node[draw=\myorange,fill=\myorange,circle,inner sep=1.5pt]   (a) at (90:.8cm) {};
        \node[draw=\myblue,fill=\myblue,circle,inner sep=1.5pt]  (b) at (162:.8cm) {};
        \node[draw=\myred,fill=\myred,circle,inner sep=1.5pt] (c) at (234:.8cm) {};
        \node[draw=\mydarkblue,fill=\mydarkblue,circle,inner sep=1.5pt] (d) at (306:.8cm) {};
        \node[draw=yellow,fill=yellow,circle,inner sep=1.5pt] (e) at (18:.8cm) {};

        \draw (c) -- (e) -- (b) -- (d) --(a) -- (b) -- (c) -- (a) -- (e) -- (d) --  (c) -- (d) ;
      \end{tikzpicture}
    \end{column}
  \end{columns}
  
  $K_t$ is a minor of $G$ if and only if $V_1 \sqcup \dots \sqcup V_t \subseteq
  V(G)$, with $V_i$ connected and $G[V_1, \dots V_t] = K_t$ 

  \uncover<2>{
  \begin{block}{Wagner, Kuratowski 1930}
    A graph is planar iff $K_5$-minor and $K_{3,3}$-minor free
  \end{block}

    \begin{columns}
    \begin{column}{.48\textwidth}
      \centering
      \begin{tikzpicture}
        \node[draw=black,fill=black,circle,inner sep=1.5pt]   (a) at (90:.8cm) {};
        \node[draw=black,fill=black,circle,inner sep=1.5pt]  (b) at (162:.8cm) {};
        \node[draw=black,fill=black,circle,inner sep=1.5pt] (c) at (234:.8cm) {};
        \node[draw=black,fill=black,circle,inner sep=1.5pt] (d) at (306:.8cm) {};
        \node[draw=black,fill=black,circle,inner sep=1.5pt] (e) at (18:.8cm) {};

        \draw (a) -- (b) -- (c) -- (d) -- (e) -- (a) -- (c) -- (e) -- (b) -- (d)
        -- (a);
      \end{tikzpicture}
    \end{column}
    \hfill
    \begin{column}{.48\textwidth}
      \centering
      \begin{tikzpicture}[auto,node distance=1cm]
      \node[draw=black,fill=black,circle,inner sep=1.5pt] (a) {};
      \node[draw=black,fill=black,circle,inner sep=1.5pt] (b) [right of=a] {};
      \node[draw=black,fill=black,circle,inner sep=1.5pt] (c) [right of=b] {};
      \node[draw=black,fill=black,circle,inner sep=1.5pt] (A) [below of=a] {};
      \node[draw=black,fill=black,circle,inner sep=1.5pt] (B) [below of=b] {};
      \node[draw=black,fill=black,circle,inner sep=1.5pt] (C) [below of=c] {};

      \draw (a) -- (A) (a) --(B) (a) -- (C);
      \draw (b) -- (A) (b) --(B) (b) -- (C);
      \draw (c) -- (A) (c) --(B) (c) -- (C);
      \end{tikzpicture}
    \end{column}
  \end{columns}}
\end{frame}

\begin{frame}{Hadwiger's conjecture}

  \begin{block}{Appel, Haken 1976}
    If $G$ is planar, then $\chi(G) \le 4$
  \end{block}

  \begin{block}{Robertson, Sanders, Seymour, Thomas 1997}
    Much simpler proof, but still computer assisted 
  \end{block}
  
  \begin{alertblock}{Hadwiger's conjecture 1943}
    If $G$ is $K_t$-minor free then $\chi(G) \le t-1$\\
    Proved for $1 \le t \le 6$, widely open for $t > 6$
  \end{alertblock}
\end{frame}

\section{Recoloring version of Hadwiger's conjecture}
\begin{frame}{Reconfiguration counterpoint}

  \begin{block}{Meyniel 1978}
    All 5-colorings of a planar graph are Kempe-equivalent (tight)
  \end{block}

  \begin{block}{Las Vergnas and Meyniel 1981}
    All 5-colorings of a $K_5$-minor free graph are Kempe-equivalent
  \end{block}

  \uncover<2>{
  \begin{alertblock}{Conjecture 1 [Las Vergnas and Meyniel 1981]}
    All the $t$-colorings of a $K_t$-minor free graph are Kempe-equivalent
  \end{alertblock}

  \begin{alertblock}{Conjecture 2 [Las Vergnas and Meyniel 1981]}
    All the $t$-colorings of a $K_t$-minor free graph are Kempe-equivalent to a $(t-1)$-coloring
  \end{alertblock}}
\end{frame}

\begin{frame}{Frozen colorings}
  \begin{exampleblock}{Frozen coloring}
    $\alpha$ is frozen if $\forall i,j$, the graph induced by colors $i$
    and $j$ is connected
  \end{exampleblock}

  \begin{columns}
    \begin{column}{.45\textwidth}
    \centering
      \begin{tikzpicture}
        \node[draw=black,fill=\myorange,circle,inner sep=2pt]   (a) at (45:1cm) {};
        \node[draw=black,fill=\myblue,circle,inner sep=2pt]  (b) at (135:1cm) {};
        \node[draw=black,fill=\myyellow,circle,inner sep=2pt] (c) at (225:1cm) {};
        \node[draw=black,fill=\myred,circle,inner sep=2pt] (d) at (315:1cm) {};
        \node[draw=black,fill=\myred,circle,inner sep=2pt] (B) at (135:.5cm) {};
        \node[draw=black,fill=\myblue,circle,inner sep=2pt] (D) at (315:.5cm) {};
        \node[draw=black,fill=\myyellow,circle,inner sep=2pt] (A) at (45:1.8cm) {};
        \node[draw=black,fill=\myorange,circle,inner sep=2pt] (C) at (225:1.8cm) {};
        \node (D') at (315:2cm) {};
        
        \draw (a) -- (b) -- (c) -- (d) -- (a) -- (A) -- (b) -- (B) -- (D) -- (d)
        -- (A);
        \draw (c) -- (B) -- (a) -- (D) -- (c) -- (C)  (b) -- (C) -- (d) ;
        \draw (A) .. controls (D') ..  (C);
      \end{tikzpicture}
    \end{column}
    \hfill
    
    \begin{column}{.45\textwidth}
    \centering
      \begin{tikzpicture}
        \node[draw=black,fill=\myorange,circle,inner sep=2pt]   (a) at (45:1cm) {};
        \node[draw=black,fill=\myblue,circle,inner sep=2pt]  (b) at (135:1cm) {};
        \node[draw=black,fill=\myorange,circle,inner sep=2pt] (c) at (225:1cm) {};
        \node[draw=black,fill=\myblue,circle,inner sep=2pt] (d) at (315:1cm) {};
        \node[draw=black,fill=\myred,circle,inner sep=2pt] (B) at (135:.5cm) {};
        \node[draw=black,fill=\myyellow,circle,inner sep=2pt] (D) at (315:.5cm) {};
        \node[draw=black,fill=\myyellow,circle,inner sep=2pt] (A) at (45:1.8cm) {};
        \node[draw=black,fill=\myred,circle,inner sep=2pt] (C) at (225:1.8cm) {};
        \node (D') at (315:2cm) {};
        
        \draw (a) -- (b) -- (c) -- (d) -- (a) -- (A) -- (b) -- (B) -- (D) -- (d)
        -- (A);
        \draw (c) -- (B) -- (a) -- (D) -- (c) -- (C)  (b) -- (C) -- (d) ;
        \draw (A) .. controls (D') ..  (C);
      \end{tikzpicture}
    \end{column}
  \end{columns}

  
  \begin{exampleblock}{Quasi-minor}
    $K_t$ is quasi-minor of $G$ if there exists $V_1 \sqcup \dots \sqcup V_t$ such
    that $\forall i \neq j, G[V_i \cup V_j]$ is connected and $G[V_1, \dots V_t]
    = K_t$
  \end{exampleblock}
  
  $K_t$-minor $\Rightarrow$ quasi $K_t$-minor \hspace{3cm}
  Frozen $t$-coloring $\Rightarrow$ quasi $K_t$-minor 
\end{frame}

\begin{frame}{Last conjecture}
  \begin{alertblock}{Conjecture 3 [Las Vergnas and Meyniel 1981]}
    No $K_t$-minor $\Rightarrow$ No quasi $K_t$-minor
  \end{alertblock}
  \begin{block}{Motivation}
    If $G$ has no $K_t$ minor and all its $t$-colorings are Kempe equivalent
    then either 
    \begin{itemize}
    \item no frozen $t$-coloring
    \item only one $t$-coloring up to color permutation $\leadsto$ Hadwiger's
      conjecture is false
    \end{itemize}
  \end{block}

  \begin{block}{Conjecture 3 holds for}
    \begin{description}
    \item[{[Jørgensen '94]}] $t= 8$
    \item[{[Song and Thomas '06]}] $t=9$
    \item[{[Kriesell '21]}] $t=10$
    \end{description}
  \end{block}
  
\end{frame}

\begin{frame}{To sum up}

  \begin{block}{No $K_t$-minor implies ...}
  \begin{center}
    \begin{tikzpicture}
      \node[anchor=south west,draw=black,rectangle,minimum height = .6cm] (conj1) at (0,1.5) {1. $t$-recolorable};
      \node[anchor=south west,draw=black,rectangle,minimum height = .6cm] (conj2) at (0,0) {2. Every
        $t$-coloring is equivalent to a $(t-1)$-coloring};
      \node[anchor=south west,draw=black,rectangle,minimum height = .6cm] (conj3) at (3.5,1.5) {3. No quasi-$K_t$-minor};
      \draw[->,\myorange] (1.5,1.5) -- (1.5,.7);
      \node[anchor=south west,\myorange] (hyp) at (1.5, .8) {Assuming Hadwiger};;

      \node[anchor=south west] (plus) at (2.9,1.5) {$\wedge$};
            
      \draw[->] (7.2,1.8) -- (8,1.8);
      \node[anchor=south west] (hadwiger) at (8, 1.5) {Hadwiger's conjecture is false};
      
    \end{tikzpicture}
  \end{center}
  \end{block}

  \pause
  \begin{block}{Bonamy, Heinrich, L., Narboni '23}
    \begin{itemize}
    \item Strongly disproved for large $t$: for every $\varepsilon >0$ and large
      enough $t$, there exists a graph $G$ with a quasi-$K_t$-minor but no
      $K_{\left(\frac{2}{3} + \varepsilon\right)t}$-minor. This graph admits
      another $t$-coloring.
      \pause
    \item All $\frac{t}{2}$-colorings of a $K_t$-minor free graph are
      Kempe-equivalent
    \end{itemize}
  \end{block}
\end{frame}

\begin{frame}{Sketch of proof}
  \begin{block}{Random construction of $G_t$}
    \vspace{-.5cm}
    \begin{columns}
      \begin{column}{.6\linewidth}
        \begin{itemize}
        \item Start with a clique on $V = \{a_1, b_1, \dots a_t, b_t\}$
        \item<2-> For all $i$, remove $a_ib_i$
        \item<3-> For all $i,j$, pick independently at random an edge in $\{a_i,b_i\} \times
          \{a_j, b_j\}$ and remove it
        \end{itemize}
      \end{column}
      \hfill
      \begin{column}{.35\linewidth}
        \begin{center}
          \begin{tikzpicture}
            \node[draw=black,fill=\myyellow,circle,inner sep=1.5pt]   (a1) at (45:.8cm) {};
            \node[draw=black,fill=\myyellow,circle,inner sep=1.5pt]  (b1) at (45:1.5cm) {};            
            \node[draw=black,fill=\myorange,circle,inner sep=1.5pt]   (a2) at (135:.8cm) {};
            \node[draw=black,fill=\myorange,circle,inner sep=1.5pt]  (b2) at (135:1.5cm) {};            
            \node[draw=black,fill=\myblue,circle,inner sep=1.5pt]   (a3) at (225:.8cm) {};
            \node[draw=black,fill=\myblue,circle,inner sep=1.5pt]  (b3) at (225:1.5cm) {};            
            \node[draw=black,fill=\myred,circle,inner sep=1.5pt]   (a4) at (315:.8cm) {};
            \node[draw=black,fill=\myred,circle,inner sep=1.5pt]  (b4) at
            (315:1.5cm) {};

            % \draw[gray!120] (a1) -- (b2) -- (b1) -- (a2);
            % \draw[gray!120] (b1) -- (b4) -- (a1) -- (a4);
            % \draw[gray!120] (b3) -- (b4) -- (a3) -- (a4);

            % \draw[gray!120] (a1) -- (a3);
            % \draw[gray!120] (a3) edge[bend right] (b1);
            % \draw[gray!120] (b1) edge[bend right] (b3);

            % \draw[gray!120] (a2) edge[bend left] (b4);
            % \draw[gray!120] (b4) edge[bend right] (b2);
            % \draw[gray!120] (b2) edge[bend right] (a4);

            % \draw (b2) -- (b3) -- (a2) -- (a3);

            \only<2->{
            \draw[dashed] (a1) -- (b1);
            \draw[dashed] (a2) -- (b2);
            \draw[dashed] (a3) -- (b3);
            \draw[dashed] (a4) -- (b4);
          }
            \only<3->{
            \draw[dashed] (b1) -- (b2);
            \draw[dashed] (a2) -- (b3);
            \draw[dashed] (a3) -- (b4);
            \draw[dashed] (b1) -- (b4);
            \draw[dashed] (b1) edge[bend left] (a3);
            \draw[dashed] (b2) edge[bend left] (b4);
            }
          \end{tikzpicture}
        \end{center}
      \end{column}
    \end{columns}
  \end{block}

  \uncover<4>{
  \begin{block}{Properties of $G_t$}
    \vspace{-.7cm}
    \begin{columns}
      \begin{column}{.6\linewidth}
        \begin{itemize}
        \item has a frozen $t$-coloring
        \item with high probability, has another $t$-coloring
        \item with high probability, admits no $K_{(\frac23 + \varepsilon)t}$-minor
        \end{itemize}
      \end{column}
      \hfill
      \begin{column}{.35\linewidth}
        \begin{center}
          \begin{tikzpicture}
            \node[draw=black,fill=\myyellow,circle,inner sep=1.5pt]   (a1) at (45:.8cm) {};
            \node[draw=black,fill=\myorange,circle,inner sep=1.5pt]  (b1) at (45:1.5cm) {};            
            \node[draw=black,fill=\myblue,circle,inner sep=1.5pt]   (a2) at (135:.8cm) {};
            \node[draw=black,fill=\myyellow,circle,inner sep=1.5pt]  (b2) at (135:1.5cm) {};            
            \node[draw=black,fill=\myred,circle,inner sep=1.5pt]   (a3) at (225:.8cm) {};
            \node[draw=black,fill=\myblue,circle,inner sep=1.5pt]  (b3) at (225:1.5cm) {};            
            \node[draw=black,fill=\myorange,circle,inner sep=1.5pt]   (a4) at (315:.8cm) {};
            \node[draw=black,fill=\myred,circle,inner sep=1.5pt]  (b4) at
            (315:1.5cm) {};

            \draw[dashed] (a1) -- (b1);
            \draw[dashed] (a2) -- (b2);
            \draw[dashed] (a3) -- (b3);
            \draw[dashed] (a4) -- (b4);

            \draw[dashed] (a1) -- (b2);
            \draw[dashed] (a2) -- (b3);
            \draw[dashed] (a3) -- (b4);
            \draw[dashed] (a4) -- (b1);
          \end{tikzpicture}
        \end{center}
      \end{column}
    \end{columns}
  \end{block}
  }
\end{frame}

\begin{frame}{Open questions}
  \begin{block}{Open questions}
    \begin{itemize}
    \item What it is the infimum $c$ such that for $t$ large enough, there
      is $G$ with a quasi $K_t$-minor but no $K_{ct}$-minor ?
      $$\frac12 \le c \le \frac23$$
    \item<2-> Is there $c'$ such that for every $t$, all the
      $c't$-colourings of a graph with no $K_t$-minor are equivalent?
      $$\frac32 \le c' \quad \text{ and all }O(t\sqrt{\log(t)})\text{-colorings are equivalent}$$
    \end{itemize}
  \end{block}

  \uncover<3>{
  \centering
  Thanks !}
\end{frame}

\end{document}
