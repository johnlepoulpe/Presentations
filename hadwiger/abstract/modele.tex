%Compiler avec pdflatex, le résultat ne doit pas dépasser une page
%Merci de respecter scrupuleusement ce modèle.
\documentclass[a4paper,titlepage,12pt,normalheadings,makeidx]{article}
\usepackage{graphicx} 
\usepackage[utf8]{inputenc}
\usepackage[francais]{babel}
\usepackage{amsmath, amsthm, amssymb, amsfonts}
\usepackage{thmtools}
\usepackage[colorlinks=true, citecolor=green]{hyperref}
\usepackage[noabbrev,capitalize]{cleveref}
\declaretheorem[name=Theorem]{theorem}
\declaretheorem[name=Conjecture, sibling=theorem]{conjecture}
\crefname{conjecture}{Conjecture}{Conjectures}

\begin{document}
\newpage

%%%%%%%% Champs titre %%%%%%%
% Pour le titre
\subsection*{M. Bonamy, M. Heinrich, C. Legrand-Duchesne et J. Narboni : Autour d'une
  variante par recoloration de la conjecture d'Hadwiger}
%

% Pour le sommaire
\addcontentsline{toc}{section}{M. Bonamy, M. Heinrich, C. Legrand--Duchesne, J. Narboni : Autour d'une
  variante par recoloration de la conjecture d'Hadwiger}
%

% Pour la liste des auteurs
\index{Bonamy, M.}\index{Heinrich, M.}\index{Legrand--Duchesne,
  C.}\index{Narboni, J.}
%

% Citer un par un les auteurs et leur affiliation
Marthe Bonamy, CNRS, LaBRI, Bordeaux, {\tt marthe.bonamy@u-bordeaux.fr}\\
\indent
Marc Heinrich, University of Leeds, United Kingdom, {\tt
  M.Heinrich@leeds.ac.uk}\\
\indent
\underline{Clément Legrand-Duchesne}, LaBRI, Bordeaux, {\tt clement.legrand@u-bordeaux.fr}\\ %souligner l'orateur
\indent
Jonathan Narboni, LaBRI, Bordeaux, {\tt jonathan.narboni@u-bordeaux.fr}\\
\\
%%%%%% Fin Champs titre %%%%%%%%%%%%%%

% Ensuite, le résumé
En 1879, Kempe introduit la notion de \emph{changement de Kempe} en essayant de
montrer le théorème des 4 couleurs: étant donné un graphe et une $k$-coloration
de ce graphe, une \emph{chaîne de Kempe} est une composante connexe
bichromatique maximale. Un changement de Kempe consiste à intervertir les deux
couleurs au sein d'une chaîne de Kempe. On dit que deux $k$-colorations sont
\emph{Kempe-equivalentes} s'il est possible de passer de l'une l'autre via une
suite de changements de Kempe. Il est alors naturel de se demander sous quelles
conditions toutes les $k$-colorations d'un graphe sont équivalentes.

Meyniel a prouvé en 1977 que les 5-colorations d'un graphe planaire sont toutes
Kempe-équivalentes~\cite{meyniel}. Ce résultat a ensuite été étendu à tous les
graphes sans $K_5$-mineurs en 1979 par Las Vergnas et
Meyniel~\cite{lasvergnas}. Dans le même article, Las Vergnas et Meyniel ont posé
la conjecture suivante, qui peut être vue comme le pendant reconfiguration de la
conjecture d'Hadwiger:

\begin{conjecture}
  Pour tout $t$, toutes les $t$-colorations d'un graphe sans $K_t$-mineur sont
  Kempe-équivalentes.
\end{conjecture}

Nous montrons que cette conjecture ainsi que deux autres issues du même article
sont fausses :
\begin{theorem}
  Pour tout $\varepsilon > 0$, pour tout $t$ suffisamment grand, il existe un
  graphe $G$ sans $K_{\left(\frac23 + \varepsilon\right)t}$-mineur dont les
  $t$-colorations ne sont pas toutes Kempe-équivalentes.
\end{theorem}

\begin{thebibliography}{99}

\bibitem{meyniel}
H.~Meyniel, \emph{Les 5-colorations d’un graphe planaire forment une classe de commuta-
  tion unique}, J. Combinatorial Theory Ser. B {\bf 24} (1978), 251–257.
\bibitem{lasvergnas}
M.~Las Vergnas, H. Meyniel, \emph{Kempe classes and the Hadwiger conjecture},
J. Combinatorial Theory Ser. B {\bf 31} (1981), 95–104.

\end{thebibliography}


\vfill
\end{document}
