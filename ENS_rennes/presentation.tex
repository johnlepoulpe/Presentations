\documentclass[11pt,xcolor=dvipsnames,presentation]{beamer}

\usepackage[T1]{fontenc}%
\usepackage[utf8]{inputenc}%
\usepackage[main= english,francais]{babel}%
\usepackage{mathtools}

\usepackage{graphicx}%
\usepackage{courier}%
\usepackage{algorithm2e}

\usepackage{url}%
\urlstyle{sf}%

\usepackage{subcaption}%
\usepackage{caption}
\captionsetup[figure]{labelformat=empty}% redefines the caption setup of the figures environment in the beamer class.
\usepackage{tikz}
\usetikzlibrary{arrows}

\usepackage{listings}%

\usepackage{faktor}%

\usepackage{stmaryrd}%

\usepackage[style=verbose,backend=biber]{biblatex}
\addbibresource{biblio.bib}

\newcommand\blfootnote[1]{%
  \begingroup
  \renewcommand\thefootnote{}\footnote{#1}%
  \addtocounter{footnote}{-1}%
  \endgroup
}

%%%%%%%%% Begin of Beamer Layout %%%%%%%%%%%%%
\ProcessOptionsBeamer
\usecolortheme{whale}
\usecolortheme[named=ForestGreen]{structure}
\useinnertheme{rounded}
\useoutertheme{infolines}
\defbeamertemplate{description item}{align left}{\insertdescriptionitem\hfill}
\setbeamertemplate{description item}[align left]
\setbeamertemplate{footline}[frame number]
\setbeamertemplate{headline}[default]
\setbeamertemplate{navigation symbols}{}
\defbeamertemplate*{headline}{info theme}{}
\defbeamertemplate*{footline}{info theme}{\leavevmode%
\hbox{%
\begin{beamercolorbox}[wd=.2\paperwidth,ht=2.25ex,dp=1ex,center]{author in head/foot}%
\usebeamerfont{author in head/foot}\insertshortauthor
\end{beamercolorbox}%
\begin{beamercolorbox}[wd=.71\paperwidth,ht=2.25ex,dp=1ex,center]{title in head/foot}%
\usebeamerfont{title in head/foot}\insertsectionhead
\end{beamercolorbox}%
\begin{beamercolorbox}[wd=.09\paperwidth,ht=2.25ex,dp=1ex,right]{section in head/foot}%
\usebeamerfont{section in head/foot}\insertframenumber{}~/~\inserttotalframenumber\hspace*{2ex}
\end{beamercolorbox}
}\vskip0pt}
\setbeamertemplate{footline}[info theme]
%%%%%%%%% End of Beamer Layout %%%%%%%%%%%%%


\lstset{%
	basicstyle=\sffamily,%
	columns=fullflexible,%
	language=C++,% % À ajuster dans chaque projet!
	frame=lb,%
	frameround=fftf,%
}%

\sloppy

%\setbeamertemplate{headline}{}%

\setbeamertemplate{navigation symbols}{%
	\insertslidenavigationsymbol%
	% \insertframenavigationsymbol%
	% \insertsubsectionnavigationsymbol%
	% \insertsectionnavigationsymbol%
	% \insertdocnavigationsymbol%
	% \insertbackfindforwardnavigationsymbol%
}

\newcommand{\backupbegin}{
   \newcounter{finalframe}
   \setcounter{finalframe}{\value{framenumber}}
}
\newcommand{\backupend}{
   \setcounter{framenumber}{\value{finalframe}}
}

\DeclareMathOperator{\diam}{diam}
\DeclareMathOperator{\tw}{tw}

\begin{document}

\title{Kempe changes}
\author[Clément Legrand]{Clément Legrand-Duchesne}
\institute{ENS Rennes}
%\date{\today}
\date{\today}

\begin{frame}
  \titlepage

  Internship carried out from February 15 to July 28, under the
  supervision of Marthe Bonamy and Vincent Delecroix at LaBRI in Bordeaux
\end{frame}

\section*{Some context}

\begin{frame}{Recoloring with Kempe changes}
  \uncover<1->{
  \begin{columns}
    \begin{column}{.66\textwidth}
      \begin{block}{Kempe chain (1879)}
        Maximal bichromatic component in $G$
      \end{block}
    \end{column}
    \begin{column}{.33\textwidth}
      \centering
      \begin{tikzpicture}
        \node[draw=black,fill=orange,circle,inner sep=2pt]   (a) at (90:.8cm) {};
        \node[draw=black,fill=blue,circle,inner sep=2pt]  (b) at (162:.8cm) {};
        \node[draw=black,fill=magenta,circle,inner sep=2pt] (c) at (234:.8cm) {};
        \node[draw=black,fill=blue,circle,inner sep=2pt] (d) at (306:.8cm) {};
        \node[draw=black,fill=magenta,circle,inner sep=2pt] (e) at (18:.8cm) {};

        \draw (a) -- (b) -- (c) -- (d) -- (e) -- (a);
      \end{tikzpicture}
      
      %\includegraphics[width=.8\textwidth]{../figures/pdf/knot_example.pdf}
    \end{column}
  \end{columns}
  }

  \vspace{1.5cm}
  
  \begin{columns}
    \uncover<2->{
      \begin{column}{.33\textwidth}
        \centering
        \begin{tikzpicture}
          \node[draw=black,fill=orange,circle,inner sep=2pt]   (a) at (90:.8cm) {};
          \node[draw=black,very thick,fill=magenta,circle,inner sep=2pt]  (b) at (162:.8cm) {};
          \node[draw=black,very thick,fill=blue,circle,inner sep=2pt] (c) at (234:.8cm) {};
          \node[draw=black,very thick,fill=magenta,circle,inner sep=2pt] (d) at (306:.8cm) {};
          \node[draw=black,very thick,fill=blue,circle,inner sep=2pt] (e) at (18:.8cm) {};

          \draw (e) -- (a) -- (b);
          \draw[very thick] (b) -- (c) -- (d) -- (e);
        \end{tikzpicture}      
    \end{column}
    }
    \uncover<3->{
    \begin{column}{.33\textwidth}
       \centering
      \begin{tikzpicture}
        \node[draw=black,very thick,fill=blue,circle,inner sep=2pt]   (a) at (90:.8cm) {};
        \node[draw=black,very thick,fill=orange,circle,inner sep=2pt]  (b) at (162:.8cm) {};
        \node[draw=black,fill=magenta,circle,inner sep=2pt] (c) at (234:.8cm) {};
        \node[draw=black,fill=blue,circle,inner sep=2pt] (d) at (306:.8cm) {};
        \node[draw=black,fill=magenta,circle,inner sep=2pt] (e) at (18:.8cm) {};

        \draw[very thick] (a) -- (b);
        \draw (b) -- (c) -- (d) -- (e) -- (a);
      \end{tikzpicture}
    \end{column}
    }
    \uncover<4->{
    \begin{column}{.33\textwidth}
       \centering
      \begin{tikzpicture}
        \node[draw=black,fill=orange,circle,inner sep=2pt]   (a) at (90:.8cm) {};
        \node[draw=black,fill=blue,circle,inner sep=2pt]  (b) at (162:.8cm) {};
        \node[draw=black,fill=magenta,circle,inner sep=2pt] (c) at (234:.8cm) {};
        \node[draw=black,very thick,fill=orange,circle,inner sep=2pt] (d) at (306:.8cm) {};
        \node[draw=black,fill=magenta,circle,inner sep=2pt] (e) at (18:.8cm) {};

        \draw (a) -- (b) -- (c) -- (d) -- (e) -- (a);
      \end{tikzpicture}
    \end{column}
    }
  \end{columns}
\end{frame}

\begin{frame}{Motivation}
  \begin{block}{Powerful tool}
    Vizing theorem: The edges of $G$ can be colored with $\Delta(G) +1$ colors
  \end{block}
  
  \begin{block}{Reconfiguration graph}
    \begin{itemize}
    \item $V(R^k(G)) =$ $k$-colorings of $G$.
    \item $\alpha$ and $\beta$ adjacent  if $\alpha
      \xleftrightarrow[\text{Kempe}]{} \beta$
    \end{itemize}
  \end{block}
  
  \uncover<2>{
  \begin{description}
  \item[Reachability] Are $\alpha$ and $\beta$ Kempe-equivalent ? 
  \item[Connectivity] Is $R^k(G)$ connected ?
  \item[Diameter] Estimate $\diam(R^k(G))$ ?
  \end{description}

  \begin{block}{Sampling coloring with a Markov chain}
    \begin{itemize}
    \item What is the mixing time of the associated Markov chain ?
    \item Estimate the number of colorings
    \end{itemize}
  \end{block}}
\end{frame}

\begin{frame}{Warm up}
  \begin{block}{$k$-degenerate graph}
    $G$ is $k$-degenerate if there exists an elimination ordering of the vertices $v_1 \prec
      v_2 \dots \prec v_n$: $$\forall i, |N^+[v_i]| \le k$$
  \end{block}

  \begin{itemize}
  \item $G$ has degree less than $k$ implies $G$ $(k+1)$-degenerate
  \item $G$ not regular of degree less than $k$ implies $G$ $k$-degenerate
  \end{itemize}

  \vspace{1cm}
  \centering
    \begin{tikzpicture}[auto,node distance=1cm,thick]
      \clip (-.5,.8) rectangle (7.5,-1.5);
       
      \node[draw=black,circle,inner sep=2pt] (1) {};
      \node[draw=black,circle,inner sep=2pt] (2) [right of=1] {};
      \node[draw=black,circle,inner sep=2pt] (3) [right of=2] {};
      \node[draw=black,circle,inner sep=2pt] (4) [right of=3] {};
      \node[draw=black,circle,inner sep=2pt] (5) [right of=4] {};
      \node[draw=black,circle,inner sep=2pt] (6) [right of=5] {};
      \node[draw=black,circle,inner sep=2pt] (7) [right of=6] {};


      \draw (1) -- (2);
      \draw (3) -- (4) -- (5) -- (6) -- (7);
      \path[every node/.style={font=\sffamily\small}]
      (4) edge[bend right] node {} (6)
      (7) edge[bend right] node {} (5)
      (4) edge[bend left] node {} (2)
      (2) edge[bend left] node {} (5)
      (1) edge[bend left] node {} (4)
      (3) edge[bend right] node {} (7);

  \end{tikzpicture}
  
%%   \begin{block}{With $\Delta+ 1$ colors}
%%     \begin{itemize}
%%     \item $\forall G, \chi(G) \le \Delta +1$
%%     \item $\forall G, \forall k \ge \Delta+1, R^k(G)$ is connected
%%     \end{itemize}
%%   \end{block}
\end{frame}

\begin{frame}{Warm up}
  \begin{block}{Las Vergnas, Meyniel 1981}
  All $\ell$-colorings of a $k$-degenerate graph are Kempe-equivalent for
    $\ell > k$. 
  \end{block}
  
  \begin{columns}
    \begin{column}{.6\textwidth}
      Induction: Let $\alpha$ and $\beta$ be two colorings of $G$\\
      Consider $G' = G\setminus\{v_1\}$
      $$\alpha_{|G'} \mathop{\leadsto}_{K}
      \beta_{|G'}$$
      Lift sequence to $G$ then recolor $v_1$
    \end{column}
    \hfill
    \begin{column}{.3\textwidth}
      \centering
      \only<1>{
        \begin{tikzpicture}
          \clip (-.5,.7) rectangle (3,-2);
          \node[draw,fill=magenta, circle,inner sep=2pt] (a) at (0,0) {};
          \node[above] (A) at (0,0) {$v_1$};
          \node[draw,fill=blue, circle, inner sep=2pt] (b) at (0:1.5cm) {};
          \node[draw,very thick,fill=blue, circle, inner sep=2pt] (c) at (320:1.5cm) {};
          \node[draw,fill=orange, circle, inner sep=2pt] (d) at (340:1.5cm) {};
          
          \node (e) at (335:2.2cm) {};
          \node (f) at (345:2.2cm) {};

          \node (h) at (315:2.2cm) {};
          \node (g) at (320:2.22cm) {};
          \node (i) at (325:2.18cm) {};

          \node (j) at (0:2.2cm) {};

          \node[above] (G) at (2.6, -1.2) {$G'$};
          \draw[domain=120:180] plot ({2*(cos(\x)+1.3)}, {2*(sin(\x)-.6)});

          \draw (j) -- (b) -- (a) -- (c);
          \draw[very thick] (h) -- (c) -- (g);
          \draw (a) -- (d);
          \draw (f) -- (d) -- (e);
          \draw (i) -- (c);
          
      \end{tikzpicture}}

      \only<2>{
        \begin{tikzpicture}
          \clip (-.5,.7) rectangle (3,-2);
          \node[draw,very thick,fill=magenta, circle,inner sep=2pt] (a) at (0,0) {};
          \node[above] (A) at (0,0) {$v_1$};
          \node[draw,very thick,fill=blue, circle, inner sep=2pt] (b) at (0:1.5cm) {};
          \node[draw,very thick,fill=blue, circle, inner sep=2pt] (c) at (320:1.5cm) {};
          \node[draw,fill=orange, circle, inner sep=2pt] (d) at (340:1.5cm) {};
          
          \node (e) at (335:2.2cm) {};
          \node (f) at (345:2.2cm) {};

          \node (h) at (315:2.2cm) {};
          \node (g) at (320:2.22cm) {};
          \node (i) at (325:2.18cm) {};

          \node (j) at (0:2.2cm) {};

          \node[above] (G) at (2.6, -1.2) {$G'$};
          \draw[domain=120:180] plot ({2*(cos(\x)+1.3)}, {2*(sin(\x)-.6)});

          \draw[very thick] (j) -- (b) -- (a) -- (c);
          \draw[very thick] (h) -- (c) -- (g);
          \draw (a) -- (d);
          \draw (f) -- (d) -- (e);
          \draw (i) -- (c);
        \end{tikzpicture}}

      \only<3>{
        \begin{tikzpicture}
          \clip (-.5,.7) rectangle (3,-2);
          \node[draw,fill=cyan, circle,inner sep=2pt] (a) at (0,0) {};
          \node[above] (A) at (0,0) {$v_1$};
          \node[draw,fill=blue, circle, inner sep=2pt] (b) at (0:1.5cm) {};
          \node[draw,very thick,fill=blue, circle, inner sep=2pt] (c) at (320:1.5cm) {};
          \node[draw,fill=orange, circle, inner sep=2pt] (d) at (340:1.5cm) {};
          
          \node (e) at (335:2.2cm) {};
          \node (f) at (345:2.2cm) {};

          \node (h) at (315:2.2cm) {};
          \node (g) at (320:2.22cm) {};
          \node (i) at (325:2.18cm) {};

          \node (j) at (0:2.2cm) {};

          \node[above] (G) at (2.6, -1.2) {$G'$};
          \draw[domain=120:180] plot ({2*(cos(\x)+1.3)}, {2*(sin(\x)-.6)});

          \draw (j) -- (b) -- (a) -- (c);
          \draw[very thick] (h) -- (c) -- (g);
          \draw (a) -- (d);
          \draw (f) -- (d) -- (e);
          \draw (i) -- (c);
        \end{tikzpicture}}
    \end{column}
  \end{columns}

  \uncover<3>{
  \begin{block}{Natural question}
   What is the diameter of the reconfiguration graph in this setting ?
  \end{block}
  }
\end{frame}

\section{Some recoloring problems}
\begin{frame}{Recoloring with $\Delta$ colors}
  \begin{block}{Brooks' theorem}
    All graphs but cliques and odd cycles can be colored with $\Delta$
    colors
  \end{block}

  \begin{columns}
    \begin{column}{.65\textwidth}
      \begin{block}{Mohar conjecture 06}
        $\forall k \ge \Delta, R^k(G)$ is connected
      \end{block}

      \begin{block}{Feghali, Johnson, Paulusma 16}
        True for all 3-regular graphs but the 3-prism
      \end{block}

      \begin{block}{Bonamy, Bousquet, Feghali, Johnson 19}
        True for all $k$-regular graphs with $k \ge 4$
      \end{block}
    \end{column}
    \hfill
    \begin{column}{.3\textwidth}
      \centering
      \begin{tikzpicture}
        \node[draw=black,fill=orange,circle,inner sep=2pt]   (a) at (90:.5cm) {};
        \node[draw=black,fill=blue,circle,inner sep=2pt]  (b) at (210:.5cm) {};
        \node[draw=black,fill=magenta,circle,inner sep=2pt] (c) at (330:.5cm) {};

        \only<1>{
          \node[draw=black,fill=blue,circle,inner sep=2pt]   (A) at (90:1.5cm) {};
          \node[draw=black,fill=magenta,circle,inner sep=2pt]  (B) at
          (210:1.5cm) {};
          \node[draw=black,fill=orange,circle,inner sep=2pt] (C) at
          (330:1.5cm) {};
        }
        \only<2->{
          \node[draw=black,fill=magenta,circle,inner sep=2pt]   (A) at (90:1.5cm) {};
          \node[draw=black,fill=orange,circle,inner sep=2pt]  (B) at
          (210:1.5cm) {};
          \node[draw=black,fill=blue,circle,inner sep=2pt] (C) at
          (330:1.5cm) {};
        }

        \draw (a) -- (A);
        \draw (b) -- (B);
        \draw (c) -- (C);
        \draw (a) -- (b) -- (c) -- (a);
        \draw (A) -- (B) -- (C) -- (A);
      \end{tikzpicture}
    \end{column}
  \end{columns}

  \uncover<3>{
  \begin{block}{Bonamy, Delecroix, L. 21+}
    $R^k(G)$ has diameter $O(\Delta n^2)$ for $k \ge \Delta$
  \end{block}}
\end{frame}

\begin{frame}{Other result}
  \begin{block}{Treewidth}
    Parameter that measures how much a graph looks like a tree.\\
    $\tw(G) \le k \Rightarrow G$ is $k$-degenerate
  \end{block}
  
  \begin{block}{Bonamy, Delecroix, L. 21}
    If $G$ has treewidth $\tw$, then $\forall k \ge \tw+1, \diam(R^k(G)) = O(\tw n^2)$  
  \end{block}
\end{frame}

\begin{frame}{What next ?}
  \begin{itemize}
  \item Diameter for $k$-degenerate graphs
  \item Sampling and mixing time
  \item Aproximate counting of the number of colorings
  \item Glauber dynamics: trivial Kempe changes
  \end{itemize}
\end{frame}
\backupbegin
\begin{frame}{Recoloring version of Hadwiger conjecture}
  \begin{block}{Graph minor}
    $H$ is a minor of $G$ is $H$ can be obtained from $G$ by deleting vertices
    and contracting along edges
  \end{block}

  \begin{block}{Hadwiger conjecture 1943}
    For all graph $G$, with no $K_t$-minor, $\chi(G) \le t-1$\\
    True for $t \le 6$ 
  \end{block}

  \begin{block}{Las Vergnas and Meyniel conjectures 1981}
    If $G$ has no $K_t$ minor, $R^k(G)$ is connected for $k \ge t-1$\\
    True for $t \le 9$    
  \end{block}

  \begin{block}{Bonamy, Delecroix, L.,  Narboni 21}
    $\forall t, \forall \varepsilon, \exists G$ with a frozen $(\frac{3}{2} + \varepsilon)t$-coloring
  \end{block}

  
\end{frame}


\backupend

\end{document}
