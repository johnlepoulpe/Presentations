\documentclass[11pt,dvipsnames,presentation,aspectratio=169]{beamer}
%\documentclass[11pt,xcolor=dvipsnames,presentation,aspectratio=169]{beamer}

\usepackage{BeamerColor}
\usepackage{etoolbox}

\usepackage[T1]{fontenc}%
\usepackage[utf8]{inputenc}%
\usepackage[main= english,francais]{babel}%
\usepackage{mathtools}

\usepackage{graphicx}%
\usepackage{courier}%
\usepackage{algorithm2e}

\usepackage{url}%
\urlstyle{sf}%

\usepackage{subcaption}%
\usepackage{caption}
\captionsetup[figure]{labelformat=empty}% redefines the caption setup of the
% figures environment in the beamer class.

\usepackage{tikz}
\usetikzlibrary{calc}

\usepackage{listings}%

\usepackage{faktor}%

\usepackage{stmaryrd}%

\usepackage[style=verbose,backend=biber]{biblatex}
\addbibresource{biblio.bib}

\newcommand\blfootnote[1]{%
  \begingroup
  \renewcommand\thefootnote{}\footnote{#1}%
  \addtocounter{footnote}{-1}%
  \endgroup
}

%%%%%%%%% Colors of Beamer Layout %%%%%%%%%%%%%
\newcommand{\myorange}{Orange}
\newcommand{\mygreen}{DarkGreen}
\newcommand{\mycyan}{LightSeaGreen}
\newcommand{\myred}{Sepia}
\newcommand{\myyellow}{Goldenrod}
\newcommand{\myblue}{RoyalBlue}
\newcommand{\mydarkblue}{MidnightBlue}



%%%%%%%%% Begin of Beamer Layout %%%%%%%%%%%%%
\ProcessOptionsBeamer
\usecolortheme{whale}
\usecolortheme[named=\myblue]{structure}
\useinnertheme{rounded}
\useoutertheme{infolines}
\setbeamertemplate{footline}[frame number]
\setbeamertemplate{headline}[default]
\setbeamertemplate{navigation symbols}{}
\setbeamertemplate{itemize items}[circle]
\defbeamertemplate*{headline}{info theme}{}
\defbeamertemplate*{footline}{info theme}{\leavevmode%
\hbox{%
\begin{beamercolorbox}[wd=.2\paperwidth,ht=2.25ex,dp=1ex,center]{author in head/foot}%
\usebeamerfont{author in head/foot}\insertshortauthor
\end{beamercolorbox}%
\begin{beamercolorbox}[wd=.71\paperwidth,ht=2.25ex,dp=1ex,center]{title in head/foot}%
\usebeamerfont{title in head/foot}\insertsectionhead
\end{beamercolorbox}%
\begin{beamercolorbox}[wd=.09\paperwidth,ht=2.25ex,dp=1ex,right]{section in head/foot}%
\usebeamerfont{section in head/foot}\insertframenumber{}~/~\inserttotalframenumber\hspace*{2ex}
\end{beamercolorbox}
}\vskip0pt}
\setbeamercolor*{block title example}{fg=\myorange}
\AtBeginEnvironment{exampleblock}{\setbeamercolor{itemize item}{fg=\myorange}}
\setbeamercolor*{block title alerted}{fg=\myred}
\AtBeginEnvironment{alertblock}{\setbeamercolor{itemize item}{fg=\myred}}
\setbeamertemplate{footline}[info theme]
\defbeamertemplate{description item}{align left}{\insertdescriptionitem\hfill}
%%%%%%%%% End of Beamer Layout %%%%%%%%%%%%%

\lstset{%
	basicstyle=\sffamily,%
	columns=fullflexible,%
	language=C++,% % À ajuster dans chaque projet!
	frame=lb,%
	frameround=fftf,%
}%

\sloppy

%\setbeamertemplate{headline}{}%

\setbeamertemplate{navigation symbols}{%
	\insertslidenavigationsymbol%
	% \insertframenavigationsymbol%
	% \insertsubsectionnavigationsymbol%
	% \insertsectionnavigationsymbol%
	% \insertdocnavigationsymbol%
	% \insertbackfindforwardnavigationsymbol%
}

\newcommand{\backupbegin}{
   \newcounter{finalframe}
   \setcounter{finalframe}{\value{framenumber}}
}
\newcommand{\backupend}{
   \setcounter{framenumber}{\value{finalframe}}
}


\newcommand{\tile}[5]{
  \fill[fill=#2] (#1) --++ (-.5,.5) --++ (1,0) -- cycle; 
  \fill[fill=#3] (#1) --++ (.5,-.5) --++ (0,1) -- cycle; 
  \fill[fill=#4] (#1) --++ (-.5,-.5) --++ (1,0) -- cycle; 
  \fill[fill=#5] (#1) --++ (-.5,-.5) --++ (0,1) -- cycle;
  \draw (#1)++(-.5,-.5) --++ (1,0) --++ (0,1) --++ (-1,0) -- cycle;}
  
\newcommand{\shadedtile}[6]{
  \fill[fill=#2,opacity=.5] (#1) --++ (-.5,.5) --++ (1,0) -- cycle; 
  \fill[fill=#3,opacity=.5] (#1) --++ (.5,-.5) --++ (0,1) -- cycle; 
  \fill[fill=#4,opacity=.5] (#1) --++ (-.5,-.5) --++ (1,0) -- cycle; 
  \fill[fill=#5,opacity=.5] (#1) --++ (-.5,-.5) --++ (0,1) --
  cycle;
  \draw (#1)++(-.5,-.5) --++ (1,0) --++ (0,1) --++ (-1,0) -- cycle;
  \node at (#1) {#6};}

\newcommand{\labeledtile}[8]{
  \only<#1>{\tile{#4}{#5}{#6}{#7}{#8}}
  \only<#2>{\shadedtile{#4}{#5}{#6}{#7}{#8}{#3}}}
    
    
\DeclareMathOperator{\diam}{diam}
\DeclareMathOperator{\tw}{tw}

\begin{document}

\title{The structure of quasi-transitive graphs avoiding a minor with
  applications to the domino problem}
\author[Clément Legrand]{Clément Legrand-Duchesne}
\institute{LaBRI, Bordeaux}
%\date{\today}
\date{October 13, 2023}

\begin{frame}
  \titlepage

  \centering
  Joint work with Louis Esperet and Ugo Giocanti.
\end{frame}

\section*{Motivation: The domino problem}
\begin{frame}{Wang's dominos}

  \begin{block}{Wang's tiling problem}
  {\bf Entry:} A finite family of tiles $\mathcal{T}$\\
  {\bf Question:} Does there exist a tiling of $\mathbb{Z}^2$ using tiles of
  $\mathcal{T}$ ?
  \end{block}
  \begin{columns}
    \begin{column}{.5\linewidth}
      \centering
      \begin{tikzpicture}
        \tile{0,0}{\myorange}{\myyellow}{\myblue}{\mydarkblue}
        \tile{2,0}{\myblue}{\mydarkblue}{\myorange}{\myred}
        \tile{1,-1.5}{\myorange}{\myred}{\myorange}{\myyellow}
      \end{tikzpicture}
    \end{column}
    \hfill
    \begin{column}{.5\linewidth}
      \centering
      \uncover<2->{
      \begin{tikzpicture}
        \foreach \i [evaluate=\i] in {1, ...,3}{
          \tile{{Mod(\i,3)},{Mod(-\i,3)}}{\myorange}{\myyellow}{\myblue}{\mydarkblue}
          \tile{{Mod(\i,3)},{Mod(1-\i,3)}}{\myorange}{\myred}{\myorange}{\myyellow}
          \tile{{Mod(\i,3)},{Mod(2-\i,3)}}{\myblue}{\mydarkblue}{\myorange}{\myred}
        }
      \end{tikzpicture}}
    \end{column}
  \end{columns}

  \uncover<3>{
    \begin{block}{Theorem}
      \begin{itemize}
      \item Wang's tiling problem is undecidable.
      \item There exist aperiodic sets of tiles
      \end{itemize}
    \end{block}}
\end{frame}

\begin{frame}{Generalization of Wang's dominos}

  \begin{alertblock}{What about tiling other spaces ?}
    On $\mathbb{Z}$, there exists a tiling if and only if there exists a
    periodic tiling, so the problem is decidable.
  \end{alertblock}

  \begin{exampleblock}{Alternative definition of the problem}
    {\bf Entry:} $k$ colors and a finite set of forbidden patterns
    $\mathcal{F}$\\
    {\bf Question:} Is there a coloring of $\mathbb{Z}^2$ that avoids
    $\mathcal{F}$?
  \end{exampleblock}
\begin{columns}
    \begin{column}{.3\linewidth}
      \centering
      \begin{tikzpicture}
        \labeledtile{2}{3}{1}{0,0}{\myorange}{\myyellow}{\myblue}{\mydarkblue}
        \labeledtile{2}{3}{2}{2,0}{\myblue}{\mydarkblue}{\myorange}{\myred}
        \labeledtile{2}{3}{3}{1,-1.5}{\myorange}{\myred}{\myorange}{\myyellow}
      \end{tikzpicture}
    \end{column}
    \hfill
    \begin{column}{.7\linewidth}
      \centering
      \uncover<2->{
        $\mathcal{F} = \left\{\quad 
          \begin{minipage}{.6\linewidth}
          \begin{columns}
            \begin{column}{.3\linewidth}
              \begin{tikzpicture}
                \labeledtile{2}{3}{1}{0,0}{\myorange}{\myyellow}{\myblue}{\mydarkblue}
                \labeledtile{2}{3}{2}{1,0}{\myblue}{\mydarkblue}{\myorange}{\myred}
              \end{tikzpicture}
            \end{column}
            \hfill
            \begin{column}{.3\linewidth}
              \centering
              \begin{tikzpicture}
                \labeledtile{2}{3}{2}{0,0}{\myblue}{\mydarkblue}{\myorange}{\myred}
                \labeledtile{2}{3}{3}{1,0}{\myorange}{\myred}{\myorange}{\myyellow}
              \end{tikzpicture}
            \end{column}
            \hfill
            \begin{column}{.3\linewidth}
              \centering
              \begin{tikzpicture}
                \labeledtile{2}{3}{3}{0,0}{\myorange}{\myred}{\myorange}{\myyellow}
                \labeledtile{2}{3}{1}{1,0}{\myorange}{\myyellow}{\myblue}{\mydarkblue}
              \end{tikzpicture}
            \end{column}
          \end{columns}
          \vspace{.5cm}
          \begin{columns}
            \begin{column}{.22\linewidth}
              \centering
              \begin{tikzpicture}
                \labeledtile{2}{3}{2}{0,0}{\myblue}{\mydarkblue}{\myorange}{\myred}
                \labeledtile{2}{3}{3}{0,1}{\myorange}{\myred}{\myorange}{\myyellow}
              \end{tikzpicture}
            \end{column}
            \hfill
            \begin{column}{.22\linewidth}
              \centering
              \begin{tikzpicture}
                \labeledtile{2}{3}{2}{0,0}{\myblue}{\mydarkblue}{\myorange}{\myred}
                \labeledtile{2}{3}{2}{0,1}{\myblue}{\mydarkblue}{\myorange}{\myred}
              \end{tikzpicture}
            \end{column}
            \hfill
            \begin{column}{.22\linewidth}
              \centering
              \begin{tikzpicture}
                \labeledtile{2}{3}{3}{0,0}{\myorange}{\myred}{\myorange}{\myyellow}
                \labeledtile{2}{3}{1}{0,1}{\myorange}{\myyellow}{\myblue}{\mydarkblue}
              \end{tikzpicture}
            \end{column}
            \hfill
            \begin{column}{.22\linewidth}
              \centering
              \begin{tikzpicture}
                \labeledtile{2}{3}{1}{0,0}{\myorange}{\myyellow}{\myblue}{\mydarkblue}
                \labeledtile{2}{3}{1}{0,1}{\myorange}{\myyellow}{\myblue}{\mydarkblue}
              \end{tikzpicture}
            \end{column}
          \end{columns}
        \end{minipage}
      \quad \right\}$}
    \end{column}
  \end{columns}
  \vspace{5cm}
\end{frame}

\begin{frame}{Natural generalization of $\mathbb{Z}^d$: Cayley graphs}

  \begin{columns}
    \hfill
    \begin{column}{.65\textwidth}
      \begin{exampleblock}{Some examples of Cayley graphs}
        \begin{itemize}
        \item $\mathbb{Z}^d$
        \item The infinite $d$-valent trees and their blow-ups
        \end{itemize}
      \end{exampleblock}

      \begin{block}{Strong structural properties of Cayley graphs}
        \begin{itemize}
        \item Regular
        \item Transitive: For all $u$,$v$, $\exists \phi \in Aut(G), \phi(u) = \phi(v)$
        \item Strong connections with expanders
        \end{itemize}
      \end{block}
    \end{column}
    \begin{column}{.35\textwidth}
      \begin{center}
        
        \begin{tikzpicture}
          \foreach \i [evaluate=\i] in {0,.5,1,1.5,2}{
            \draw (\i, -.3) -- (\i,2.3);
            \draw (-.3,\i) -- (2.3,\i);}
          
          \foreach \i [evaluate=\i] in {0,.5,1,1.5,2}{
            \foreach \j [evaluate=\i] in {0,.5,1,1.5,2}{
              \node[draw=black,circle,inner sep = 1pt, fill = \myorange] (v\i\j)
              at (\i,\j) {};
            }
          }
        \end{tikzpicture}
        
        \vspace{.2cm}
        \begin{tikzpicture}
          \pgfmathsetmacro{\r}{.5}
          \pgfmathsetmacro{\tri}{.1}  
          \pgfmathsetmacro{\long}{3}
          \pgfmathsetmacro{\haut}{2}

          \pgfmathsetmacro{\hh}{int(\haut+1)}
          \pgfmathsetmacro{\lll}{int(\long-1)}

          \foreach \j [evaluate=\j] in {1,...,\hh}
          {\foreach \i [evaluate=\i] in {1,...,\long}
            {\pgfmathsetmacro{\y}{int(2*\j-1)}
              \pgfmathsetmacro{\x}{int(3*\i-1)}
              \pgfmathsetmacro{\xx}{int(3*\i)}
              \foreach \angle in {0,1,2}
              {\node[draw=black,circle,inner sep=1pt,fill= \myorange] (v\x_\y_\angle) at ($({\x*\r},
                {\y*\r*sqrt(3)/2})+(\angle*120:\tri)$) {};
                \node[draw=black,circle,inner sep=1pt,fill= \myorange] (v\xx_\y_\angle) at ($({\xx*\r},
                {\y*\r*sqrt(3)/2})+(\angle*120+180:\tri)$) {};
              }
              \draw (v\x_\y_0) -- (v\x_\y_1) -- (v\x_\y_2) -- (v\x_\y_0);
              \draw (v\xx_\y_0) -- (v\xx_\y_1) -- (v\xx_\y_2) -- (v\xx_\y_0);
            }
          }
          \foreach \j [evaluate=\j] in {1,...,\hh}
          {\foreach \i [evaluate=\i] in {1,...,\long}
            {\pgfmathsetmacro{\y}{int(2*\j)}
              \pgfmathsetmacro{\x}{int(3*\i-2)}
              \pgfmathsetmacro{\xx}{int(3*\i)}
              \foreach \angle in {0,1,2}
              {\node[draw=black,circle,inner sep=1pt,fill= \myorange] (v\x_\y_\angle) at ($({(\x+0.5)*\r}, {\y*\r*sqrt(3)/2})+(\angle*120+180:\tri)$) {};
                \node[draw=black,circle,inner sep=1pt,fill= \myorange] (v\xx_\y_\angle) at ($({(\xx+0.5)*\r}, {\y*\r*sqrt(3)/2})+(\angle*120:\tri)$) {};
              }
              \draw (v\x_\y_0) -- (v\x_\y_1) -- (v\x_\y_2) -- (v\x_\y_0);
              \draw (v\xx_\y_0) -- (v\xx_\y_1) -- (v\xx_\y_2) -- (v\xx_\y_0);
            }
          }
          \foreach \j [evaluate=\j] in {1,...,\hh}
          {\foreach \i [evaluate=\i] in {1,...,\long}
            {\pgfmathsetmacro{\y}{int(2*\j-1)}
              \pgfmathsetmacro{\x}{int(3*\i-1)}
              \pgfmathsetmacro{\xx}{int(\x+1)}
              \pgfmathsetmacro{\xxx}{int(\x-1)}
              \pgfmathsetmacro{\yy}{int(\y+1)}
              \draw (v\x_\y_0) -- (v\xx_\y_0);
              \draw (v\x_\y_1) -- (v\xxx_\yy_1);
            }
          }
          \foreach \j [evaluate=\j] in {1,...,\haut}
          {\foreach \i [evaluate=\i] in {1,...,\long}
            {\pgfmathsetmacro{\y}{int(2*\j+1)}
              \pgfmathsetmacro{\x}{int(3*\i-1)}
              \pgfmathsetmacro{\xxx}{int(\x-1)}
              \pgfmathsetmacro{\yy}{int(\y+1)}
              \pgfmathsetmacro{\yyy}{int(\y-1)}
              \draw (v\x_\y_2) -- (v\xxx_\yyy_2);
            }
          }
          \foreach \j [evaluate=\j] in {1,...,\hh}
          {\foreach \i [evaluate=\i] in {1,...,\lll}
            {\pgfmathsetmacro{\y}{int(2*\j)}
              \pgfmathsetmacro{\x}{int(3*\i)}
              \pgfmathsetmacro{\xx}{int(\x+1)}
              \draw (v\x_\y_0) -- (v\xx_\y_0);
            }
          }
          \foreach \j [evaluate=\j] in {1,...,\hh}
          {\foreach \i [evaluate=\i] in {1,...,\long}
            {\pgfmathsetmacro{\y}{int(2*\j)}
              \pgfmathsetmacro{\x}{int(3*\i)}
              \pgfmathsetmacro{\yyy}{int(\y-1)}
              \draw (v\x_\y_2) -- (v\x_\yyy_2);
            }
          }
          \foreach \j [evaluate=\j] in {1,...,\haut}
          {\foreach \i [evaluate=\i] in {1,...,\long}
            {\pgfmathsetmacro{\y}{int(2*\j)}
              \pgfmathsetmacro{\x}{int(3*\i)}
              \pgfmathsetmacro{\yy}{int(\y+1)}
              \draw (v\x_\y_1) -- (v\x_\yy_1);
            }
          }
        \end{tikzpicture}
      \end{center}
    \end{column}
  \end{columns}

  \begin{alertblock}{Conjecture [Ballier and Stein 2018]}
    The domino problem is decidable in a group $\Gamma$ if and only if $\Gamma$
    has a Cayley graph $G$ of bounded treewidth.
  \end{alertblock}

\end{frame}

\begin{frame}{Crash course on treewidth}
  \begin{exampleblock}{Tree decomposition}
    $T$ is a tree decomposition of $G$ is a tree whose nodes are bags $X_i
    \subset V(G)$ such that
    \begin{itemize}
    \item $\bigcup_{i} X_i = V(G)$
    \item For all $u \in V(G)$ the subgraph of nodes containing $u$ is connected
    \item For all $uv \in E(G)$, $\exists X_i, \{u,v\} \subset X_i$
    \end{itemize}
  \end{exampleblock}

  \begin{exampleblock}{Treewidth}
    A graph has treewidth at most $k$ if it admits a tree decomposition with
    bags of size at most $k+1$
  \end{exampleblock}

\end{frame}

\begin{frame}{A few examples}
  \begin{columns}
    \begin{column}{.3\textwidth}
      \centering
      \begin{tikzpicture}
        \clip (-2.3,-2.3) rectangle (2.3,2.3);
        \foreach \i [evaluate=\i] in {0,1,2}{
          \node[draw=black,fill=\myorange,inner sep = 1pt,circle] (v00\i) at ({\i*120}:.2cm) {};
        }
        \draw (v000) -- (v001) -- (v002) -- (v000);


        \foreach \t [evaluate=\t] in {0,1,2}{
          \foreach \i [evaluate=\i] in {0,1,2}{
            \node[draw=black,fill=\myorange,inner sep = 1pt,circle] (v1\t\i) at
            ($({60+\i*120}:.2cm)+({\t*120}:1cm)$) {};
          }
          \draw (v1\t0) -- (v1\t1) -- (v1\t2) -- (v1\t0);
        }

        \foreach \t [evaluate=\t] in {0,1,2,3,4,5}{
          \foreach \i [evaluate=\i] in {0,1,2}{
            \node[draw=black,fill=\myorange,inner sep = 1pt,circle] (v2\t\i) at
            ($({\i*120}:.2cm)+({30+\t*60}:1.8cm)$) {};
          }
          \draw (v2\t0) -- (v2\t1) -- (v2\t2) -- (v2\t0);
        }

        \foreach \i [evaluate=\i] in {0,1,2}{
          \foreach \j [evaluate=\j] in {0,1,2}{
            \foreach \k [evaluate=\j] in {0,1,2}{
              \draw[gray!200] (v1\i\j) -- (v00\k);
              \pgfmathsetmacro{\x}{int(2*\i)}
              \draw[gray!200] (v1\i\j) -- (v2\x\k);
              \pgfmathsetmacro{\y}{int(Mod(2*\i-1,6))}
              \draw[gray!200] (v1\i\j) -- (v2\y\k);
            }
          }
        }
        
      \end{tikzpicture}
      
      Treewidth 3
    \end{column}
    \hfill
    \begin{column}{.3\linewidth}
      \centering
      \begin{tikzpicture}
        \clip (-1.2,-1.3) rectangle (3.2,3.3);
        \foreach \i [evaluate=\i] in {0,.5,1,1.5,2}{
          \draw (\i, -.3) -- (\i,2.3);
          \draw (-.3,\i) -- (2.3,\i);}
        
        \foreach \i [evaluate=\i] in {0,.5,1,1.5,2}{
          \foreach \j [evaluate=\i] in {0,.5,1,1.5,2}{
            \node[draw=black,circle,inner sep = 1pt, fill = \myorange] (v\i\j)
            at (\i,\j) {};
          }
        }
      \end{tikzpicture}
      
      Unbounded treewidth
    \end{column}
    \hfill
    \begin{column}{.3\linewidth}
      \vspace{.2cm}
      \centering
      \begin{tikzpicture}
        \clip (1.25,0.6) rectangle (5,5);
        \pgfmathsetmacro{\r}{.5}
        \pgfmathsetmacro{\tri}{.1}  
        \pgfmathsetmacro{\long}{4}
        \pgfmathsetmacro{\haut}{5}

        \pgfmathsetmacro{\hh}{int(\haut+1)}
        \pgfmathsetmacro{\lll}{int(\long-1)}

        \foreach \j [evaluate=\j] in {1,...,\hh}
        {\foreach \i [evaluate=\i] in {1,...,\long}
          {\pgfmathsetmacro{\y}{int(2*\j-1)}
            \pgfmathsetmacro{\x}{int(3*\i-1)}
            \pgfmathsetmacro{\xx}{int(3*\i)}
            \foreach \angle in {0,1,2}
            {\node[draw=black,circle,inner sep=1pt,fill= \myorange] (v\x_\y_\angle) at ($({\x*\r},
              {\y*\r*sqrt(3)/2})+(\angle*120:\tri)$) {};
              \node[draw=black,circle,inner sep=1pt,fill= \myorange] (v\xx_\y_\angle) at ($({\xx*\r},
              {\y*\r*sqrt(3)/2})+(\angle*120+180:\tri)$) {};
            }
            \draw (v\x_\y_0) -- (v\x_\y_1) -- (v\x_\y_2) -- (v\x_\y_0);
            \draw (v\xx_\y_0) -- (v\xx_\y_1) -- (v\xx_\y_2) -- (v\xx_\y_0);
          }
        }
        \foreach \j [evaluate=\j] in {1,...,\hh}
        {\foreach \i [evaluate=\i] in {1,...,\long}
          {\pgfmathsetmacro{\y}{int(2*\j)}
            \pgfmathsetmacro{\x}{int(3*\i-2)}
            \pgfmathsetmacro{\xx}{int(3*\i)}
            \foreach \angle in {0,1,2}
            {\node[draw=black,circle,inner sep=1pt,fill= \myorange] (v\x_\y_\angle) at ($({(\x+0.5)*\r}, {\y*\r*sqrt(3)/2})+(\angle*120+180:\tri)$) {};
              \node[draw=black,circle,inner sep=1pt,fill= \myorange] (v\xx_\y_\angle) at ($({(\xx+0.5)*\r}, {\y*\r*sqrt(3)/2})+(\angle*120:\tri)$) {};
            }
            \draw (v\x_\y_0) -- (v\x_\y_1) -- (v\x_\y_2) -- (v\x_\y_0);
            \draw (v\xx_\y_0) -- (v\xx_\y_1) -- (v\xx_\y_2) -- (v\xx_\y_0);
          }
        }
        \foreach \j [evaluate=\j] in {1,...,\hh}
        {\foreach \i [evaluate=\i] in {1,...,\long}
          {\pgfmathsetmacro{\y}{int(2*\j-1)}
            \pgfmathsetmacro{\x}{int(3*\i-1)}
            \pgfmathsetmacro{\xx}{int(\x+1)}
            \pgfmathsetmacro{\xxx}{int(\x-1)}
            \pgfmathsetmacro{\yy}{int(\y+1)}
            \draw (v\x_\y_0) -- (v\xx_\y_0);
            \draw (v\x_\y_1) -- (v\xxx_\yy_1);
          }
        }
        \foreach \j [evaluate=\j] in {1,...,\haut}
        {\foreach \i [evaluate=\i] in {1,...,\long}
          {\pgfmathsetmacro{\y}{int(2*\j+1)}
            \pgfmathsetmacro{\x}{int(3*\i-1)}
            \pgfmathsetmacro{\xxx}{int(\x-1)}
            \pgfmathsetmacro{\yy}{int(\y+1)}
            \pgfmathsetmacro{\yyy}{int(\y-1)}
            \draw (v\x_\y_2) -- (v\xxx_\yyy_2);
          }
        }
        \foreach \j [evaluate=\j] in {1,...,\hh}
        {\foreach \i [evaluate=\i] in {1,...,\lll}
          {\pgfmathsetmacro{\y}{int(2*\j)}
            \pgfmathsetmacro{\x}{int(3*\i)}
            \pgfmathsetmacro{\xx}{int(\x+1)}
            \draw (v\x_\y_0) -- (v\xx_\y_0);
          }
        }
        \foreach \j [evaluate=\j] in {1,...,\hh}
        {\foreach \i [evaluate=\i] in {1,...,\long}
          {\pgfmathsetmacro{\y}{int(2*\j)}
            \pgfmathsetmacro{\x}{int(3*\i)}
            \pgfmathsetmacro{\yyy}{int(\y-1)}
            \draw (v\x_\y_2) -- (v\x_\yyy_2);
          }
        }
        \foreach \j [evaluate=\j] in {1,...,\haut}
        {\foreach \i [evaluate=\i] in {1,...,\long}
          {\pgfmathsetmacro{\y}{int(2*\j)}
            \pgfmathsetmacro{\x}{int(3*\i)}
            \pgfmathsetmacro{\yy}{int(\y+1)}
            \draw (v\x_\y_1) -- (v\x_\yy_1);
          }
        }
      \end{tikzpicture}
      
      Unbounded treewidth
    \end{column}
  \end{columns}
  \pause
  \vspace{.5cm}
  \begin{tabular}{lc}
  Intuition behind the conjecture: &\\
  Bounded treewidth means tree like structure with periodic
    colorings&$\checkmark$\\
    \pause
  Unbounded treewidth means infinite ``grid-like'' workspace & ?
  \end{tabular}
\end{frame}

\begin{frame}{Torsos}
    \begin{exampleblock}{Torso}
    Given a tree decomposition $T$ of $G$, the torso of a bag $X_i$ is the graph
    $G\llbracket X_i \rrbracket$ such that
    \begin{itemize}
    \item $V(G\llbracket X_i \rrbracket) = X_i$,
    \item for all $u,v \in X_i$, if $uv \in E(G)$, then $uv\in E(G\llbracket X_i \rrbracket)$,
    \item for all $u,v \in X_i$, if there is a path between $u$ and $v$ in $G$
      that avoids $X_i$, then $uv \in E(G\llbracket X_i \rrbracket)$.
    \end{itemize}
  \end{exampleblock}

\end{frame}

\section*{A structure theorem on quasi-transitive graphs avoiding a minor}

\begin{frame}{Crash course on minors}
  \begin{exampleblock}{Definition}
    A graph $H$ is a minor of $G$ if $H$ can be obtained from $G$ by contracting
    edges and by deleting vertices and edges.
  \end{exampleblock}

\end{frame}

\begin{frame}{Robertson and Seymour Graph minor structure theorem}
  \begin{alertblock}{Why are $H$-minor free graphs $H$-minor free ?}
    Let $G$ be a $H$-minor-free graph. Then 
    \begin{itemize}
    \item $G$ is to thin to contain $H$
    \item $G$ is piecewise almost embeddable on surfaces too simple to contain
      $H$ as a minor.
    \end{itemize}
  \end{alertblock}

  \begin{block}{Precise statement}
    Let $H$ be a fixed graph. There exist a $k$, such that for all $G$ that does
    not contain $H$ as a minor, $G$ admits a tree-decomposition, such that :
    \begin{itemize}
    \item The adhesions have size at most $k$,
    \item Each torso is ``almost'' embeddable in a surface in which $H$ does not
      embed (too low genus)
    \end{itemize}
  \end{block}

  \begin{block}{Diestel and Thomas 1999}
    The same holds for infinite graphs $G$ that exclude some finite minor.
  \end{block}
\end{frame}

\begin{frame}{What about graphs with many symmetries ?}
  \begin{exampleblock}{Quasi-transitive graphs}
    A graph $G$ is quasi-transitive if it can be colored with finitely many
    colors, such that for all $u$,$v$ colored identically, there is $\phi \in
    Aut(G)$ such that $\phi(u) = \phi(v)$.
  \end{exampleblock}

  \begin{center}
    \begin{tikzpicture}
      \foreach \i [evaluate=\i] in {0,.5,1,1.5,2}{
        \draw (\i, -.3) -- (\i,2.3);
        \draw (-.3,\i) -- (2.3,\i);
        \only<2->{
        \draw (-.3,\i-.3) -- (2.3-\i, 2.3);
        \draw (\i+.2,-.3) -- (2.3, 1.8-\i);
        \draw (\i-.3, 2.3)--(2.3,\i-.3);
        \draw (-.3,\i-.2) -- (\i-.2, -.3);
        }
      }

      \only<2->{
      \draw (-.3,2.2)--(-.2,2.3);
      \draw (2.2,2.3)--(2.3,2.2);}
      
      \foreach \i [evaluate=\i] in {0,.5,1,1.5,2}{
        \foreach \j [evaluate=\i] in {0,.5,1,1.5,2}{
          \node[draw=black,circle,inner sep = 1pt, fill = \myorange] (v\i\j)
          at (\i,\j) {};
        }
      }

      \only<2->{
      \foreach \i [evaluate=\i] in {0,.5,1,1.5,2,2.5}{
        \foreach \j [evaluate=\i] in {0,.5,1,1.5,2,2.5}{
          \node[draw=black,circle,inner sep = 1pt, fill = \myblue] (v\i\j)
          at (\i-.25,\j-.25) {};
        }
      }}
    
    \only<3>{
      \foreach \i [evaluate=\i] in {0,.5,1,1.5,2,2.5}{
        \foreach \j [evaluate=\i] in {0,.5,1,1.5,2}{
          \node[draw=black,circle,inner sep = 1pt, fill = \myred] (v\i\j)
          at (\i-.25,\j) {};
        }
      }}
    

    \end{tikzpicture}
  \end{center}
  
\end{frame}
\begin{frame}{Preserve the symmetric structure in $T$}
  \begin{exampleblock}{Canonical tree decompositions}
    A tree decomposition is said to be canonical if for all $\phi \in Aut(G)$,
    $\phi$ maps bags on other bags ($Aut(G)$ induces an action on $T$ such that
    $\forall \phi, \forall i$, $\phi(X_i) = X_{i\cdot\phi}$).
  \end{exampleblock}
  
  \pause
  \begin{block}{Esperet, Giocanti, L. 2023}
    Let $G$ be a quasi-transitive locally finite graph $G$ avoiding the
    countable clique as a minor. Then $G$ admits a {\bf canonical} tree
    decomposition whose torsos are finite or planar.
  \end{block}

  \pause
  \begin{block}{Esperet, Giocanti, L. 2023}
    Let $G$ be a quasi-transitive locally finite graph $G$ avoiding the
    countable clique as a minor. Then $G$ admits a {\bf canonical} tree
    decomposition with adhesions of size at most 3 such that:
    \begin{itemize}
    \item each torso is a minor of $G$,
    \item torso are planar or have bounded treewidth
    \end{itemize}
  \end{block}
\end{frame}

\begin{frame}{Application 1: Hadwiger number}
  \begin{exampleblock}{Hadwiger number}
    The Hadwiger number of a graph is the supremum of the sizes of its complete minors.
  \end{exampleblock}

  \begin{block}{Thomassen 1992}
    Every locally finite quasi-transitive 4-connected graph attains its Hadwiger number.
  \end{block}

  \begin{block}{Esperet, Giocanti, L. 2023}
    Every locally finite quasi-transitive graph attains its Hadwiger number.
  \end{block}

  \centering
  ``$K_\infty$ minor free $\Rightarrow$ $K_t$ minor free for some $t$''
\end{frame}

\end{document}
