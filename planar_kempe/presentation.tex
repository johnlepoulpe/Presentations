\documentclass[11pt,xcolor=dvipsnames,presentation,aspectratio=169]{beamer}

\usepackage{BeamerColor}
\usepackage{etoolbox}

\usepackage[T1]{fontenc}%
\usepackage[utf8]{inputenc}%
\usepackage[main= english,francais]{babel}%
\usepackage{mathtools}
\usepackage{amssymb}

\usepackage{graphicx}%
\usepackage{courier}%
\usepackage{algorithm2e}

\usepackage{url}%
\urlstyle{sf}%

\usepackage{subcaption}%
\usepackage{caption}
\captionsetup[figure]{labelformat=empty}% redefines the caption setup of the figures environment in the beamer class.
\usepackage{tikz}

\usepackage{listings}%

\usepackage{faktor}%
\usepackage{cancel}%
\usepackage{stmaryrd}%

\usepackage[style=verbose,backend=biber]{biblatex}
\addbibresource{biblio.bib}

\newcommand\blfootnote[1]{%
  \begingroup
  \renewcommand\thefootnote{}\footnote{#1}%
  \addtocounter{footnote}{-1}%
  \endgroup
}

%%%%%%%%% Colors of Beamer Layout %%%%%%%%%%%%%
\newcommand{\myorange}{Orange}
\newcommand{\mygreen}{DarkGreen}
\newcommand{\mycyan}{LightSeaGreen}
\newcommand{\myred}{Fuchsia}
\newcommand{\myyellow}{Goldenrod}
\newcommand{\myblue}{RoyalBlue}
\newcommand{\mydarkblue}{MidnightBlue}

%%%%%%%%% Begin of Beamer Layout %%%%%%%%%%%%%
\ProcessOptionsBeamer
\usecolortheme{whale}
\usecolortheme[named=\myblue]{structure}
\useinnertheme{rounded}
\useoutertheme{infolines}


\setbeamertemplate{footline}[frame number]
\setbeamertemplate{headline}[default]
\setbeamertemplate{navigation symbols}{}
\setbeamertemplate{itemize items}[circle]
\defbeamertemplate*{headline}{info theme}{}
\defbeamertemplate*{footline}{info theme}{\leavevmode%


  \hbox{%
\begin{beamercolorbox}[wd=.2\paperwidth,ht=2.25ex,dp=1ex,center]{author in head/foot}%
\usebeamerfont{author in head/foot}\insertshortauthor
\end{beamercolorbox}%
\begin{beamercolorbox}[wd=.71\paperwidth,ht=2.25ex,dp=1ex,center]{title in head/foot}%
\usebeamerfont{title in head/foot}\insertsectionhead
\end{beamercolorbox}%
\begin{beamercolorbox}[wd=.09\paperwidth,ht=2.25ex,dp=1ex,right]{section in head/foot}%
\usebeamerfont{section in head/foot}\insertframenumber{}~/~\inserttotalframenumber\hspace*{2ex}
\end{beamercolorbox}
}\vskip0pt}
\setbeamercolor*{block title example}{fg=\myorange}
\AtBeginEnvironment{exampleblock}{\setbeamercolor{itemize item}{fg=\myorange}}
\setbeamercolor*{block title alerted}{fg=\myred}
\AtBeginEnvironment{alertblock}{\setbeamercolor{itemize item}{fg=\myred}}
\setbeamertemplate{footline}[info theme]
\defbeamertemplate{description item}{align left}{\insertdescriptionitem\hfill}


\lstset{%
	basicstyle=\sffamily,%
	columns=fullflexible,%
	language=C++,% % À ajuster dans chaque projet!
	frame=lb,%
	frameround=fftf,%
}%

\sloppy

%\setbeamertemplate{headline}{}%

\setbeamertemplate{navigation symbols}{%
	\insertslidenavigationsymbol%
	% \insertframenavigationsymbol%
	% \insertsubsectionnavigationsymbol%
	% \insertsectionnavigationsymbol%
	% \insertdocnavigationsymbol%
	% \insertbackfindforwardnavigationsymbol%
}

\newcommand{\backupbegin}{
   \newcounter{finalframe}
   \setcounter{finalframe}{\value{framenumber}}
}
\newcommand{\backupend}{
   \setcounter{framenumber}{\value{finalframe}}
}

\DeclareMathOperator{\diam}{diam}
\DeclareMathOperator{\mad}{mad}
\DeclareMathOperator{\tw}{tw}

\begin{document}

\title{Kempe recoloring of planar graphs}
\author[Clément Legrand]{Clément Legrand-Duchesne}
\institute{LaBRI, Bordeaux}
%\date{\today}
\date{\today}

\begin{frame}
  \titlepage
  \begin{center}
  Joint work with Quentin Deschamps, Carl Feghali,\\ František Kardoš and Théo Pierron
  \end{center}

\end{frame}

\section*{Some context}
\begin{frame}{Recoloring with Kempe changes}

  \begin{columns}
    \begin{column}{.55\linewidth}
      \begin{exampleblock}{Kempe chain (1879)}
        Maximal bichromatic connected component in $G$
      \end{exampleblock}
    \end{column}
    \begin{column}{.2\linewidth}
      \centering
      \begin{tikzpicture}
        \clip (-1,1) rectangle (1,-1);
        \node[draw=black,fill=\myorange,circle,inner sep=2pt]   (a) at (90:.8cm) {};
        \node[draw=black,fill=\myyellow,circle,inner sep=2pt]  (b) at (162:.8cm) {};
        \node[draw=black,fill=\myred,circle,inner sep=2pt] (c) at (234:.8cm) {};
        \node[draw=black,fill=\myblue,circle,inner sep=2pt] (d) at (306:.8cm) {};
        \node[draw=black,fill=\myred,circle,inner sep=2pt] (e) at (18:.8cm) {};

        \draw (a) -- (b) -- (c) -- (d) -- (e) -- (a);
      \end{tikzpicture}
      
      %\includegraphics[width=.8\textwidth]{../figures/pdf/knot_example.pdf}
    \end{column}

    \begin{column}{.2\linewidth}
      \centering
      \begin{tikzpicture}
        \only<1>{
          \clip (-1,1) rectangle (1,-1);
          \node[draw=black,fill=\myorange,circle,inner sep=2pt]   (a) at (90:.8cm) {};
          \node[draw=black,fill=\myyellow,circle,inner sep=2pt]  (b) at (162:.8cm) {};
          \node[draw=black,very thick,fill=\myblue,circle,inner sep=2pt] (c) at (234:.8cm) {};
          \node[draw=black,very thick,fill=\myred,circle,inner sep=2pt] (d) at (306:.8cm) {};
          \node[draw=black,very thick,fill=\myblue,circle,inner sep=2pt] (e) at (18:.8cm) {};

          \draw (e) -- (a) -- (b) -- (c);
          \draw[very thick] (c) -- (d) -- (e);
        }

        \only<2>{
          \clip (-1,1) rectangle (1,-1);
          \node[draw=black,fill=\myorange,circle,inner sep=2pt]   (a) at (90:.8cm) {};
          \node[draw=black,very thick,fill=\myred,circle,inner sep=2pt]  (b) at (162:.8cm) {};
          \node[draw=black,very thick,fill=\myyellow,circle,inner sep=2pt] (c) at (234:.8cm) {};
          \node[draw=black,fill=\myblue,circle,inner sep=2pt] (d) at (306:.8cm) {};
          \node[draw=black,fill=\myred,circle,inner sep=2pt] (e) at (18:.8cm) {};
          
          \draw[very thick] (b) -- (c);
          \draw (a) -- (b);
          \draw (c) -- (d) -- (e) -- (a);
        }

        \only<3->{
          \clip (-1,1) rectangle (1,-1);
          \node[draw=black,fill=\myorange,circle,inner sep=2pt]   (a) at (90:.8cm) {};
          \node[draw=black,fill=\myyellow,circle,inner sep=2pt]  (b) at (162:.8cm) {};
          \node[draw=black,fill=\myred,circle,inner sep=2pt] (c) at (234:.8cm) {};
          \node[draw=black,very thick,fill=\myyellow,circle,inner sep=2pt] (d) at (306:.8cm) {};
          \node[draw=black,fill=\myred,circle,inner sep=2pt] (e) at (18:.8cm) {};
          
          \draw (a) -- (b) -- (c) -- (d) -- (e) -- (a);
        }
      \end{tikzpicture}
    \end{column}
  \end{columns}
  
  \uncover<4->{
  \begin{exampleblock}{Reconfiguration graph}
    \begin{itemize}
    \item $V(R^k(G)) =$ $k$-colorings of $G$.
    \item $\alpha$ and $\beta$ adjacent  if $\alpha
      \xleftrightarrow[\text{Kempe}]{} \beta$
    \end{itemize}
  \end{exampleblock}}
  \vspace{-.3cm}
  \uncover<5->{
  \begin{alertblock}{Usual questions}
    \begin{itemize}
    \item Given two $k$-colorings $\alpha$ and $\beta$, $\alpha
      \leftrightsquigarrow \beta$ ?
    \item Is $R^k(G)$ connected ?
    \item What is $\diam(R^k(G))$ ?
    \item Does the corresponding Markov chain mix well ?
    \end{itemize}
  \end{alertblock}}

\end{frame}

\begin{frame}{Fundamental lemma}
  \begin{block}{Las Vergnas, Meyniel 1981}
   All $k$-colorings of a $d$-degenerate graph are equivalent, for $k > d$
  \end{block}
  
  \begin{columns}
    \begin{column}{.6\textwidth}
      By induction: Let $v$ with $\deg(v) \le d$, consider $G' = G\setminus\{v\}$
      $$\alpha_{|G'} \mathop{\leadsto}_{K}
      \beta_{|G'}$$
      Lift sequence to $G$ then recolor $v$
    \end{column}
    \hfill
    \begin{column}{.3\textwidth}
      \centering
        \begin{tikzpicture}
          \only<1>{
          \clip (-.5,1) rectangle (3,-2);
          \node[draw,fill=\myred, circle,inner sep=2pt] (A) at (0,0) {};
          \node[above] (A') at (0,0) {$v$};
          \node[draw,fill=\myblue, circle, inner sep=2pt] (a) at (20:1.5cm) {};
          \node[draw,fill=\myyellow, circle, inner sep=2pt] (b) at (0:1.5cm) {};
          \node[draw,very thick,fill=\myyellow, circle, inner sep=2pt] (c) at (320:1.5cm) {};
          \node[draw,fill=\myorange, circle, inner sep=2pt] (d) at (340:1.5cm) {};
          
          \node (e) at (335:2.2cm) {};
          \node (f) at (345:2.2cm) {};

          \node (h) at (315:2.2cm) {};
          \node (g) at (320:2.22cm) {};
          \node (i) at (325:2.18cm) {};

          \node (j) at (0:2.2cm) {};

          \node (k) at (15:2.18cm) {};
          \node (l) at (20:2.2cm) {};
          \node (m) at (25:2.18cm) {};

          \node[above] (G) at (2.6, -1.2) {$G'$};
          \begin{scope}[shift={(350:2.8cm)}]
            \draw[domain=135:205] plot ({2*cos(\x)}, {2*sin(\x)});
          \end{scope}
          
          \draw (j) -- (b) -- (A) -- (c);
          \draw[very thick] (h) -- (c) -- (g);
          \draw (a) -- (A) -- (d);
          \draw (f) -- (d) -- (e);
          \draw (i) -- (c);
          \draw (m) -- (a) -- (l)  (a) -- (k);
          }

          \only<2>{
          \clip (-.5,1) rectangle (3,-2);
          \node[draw,very thick,fill=\myred, circle,inner sep=2pt] (A) at (0,0) {};
          \node[above] (A') at (0,0) {$v$};
          \node[draw,fill=\myblue, circle, inner sep=2pt] (a) at (20:1.5cm) {};
          \node[draw,very thick,fill=\myyellow, circle, inner sep=2pt] (b) at (0:1.5cm) {};
          \node[draw,very thick,fill=\myyellow, circle, inner sep=2pt] (c) at (320:1.5cm) {};
          \node[draw,fill=\myorange, circle, inner sep=2pt] (d) at (340:1.5cm) {};
          
          \node (e) at (335:2.2cm) {};
          \node (f) at (345:2.2cm) {};

          \node (h) at (315:2.2cm) {};
          \node (g) at (320:2.22cm) {};
          \node (i) at (325:2.18cm) {};

          \node (j) at (0:2.2cm) {};

          \node (k) at (15:2.18cm) {};
          \node (l) at (20:2.2cm) {};
          \node (m) at (25:2.18cm) {};

          \node[above] (G) at (2.6, -1.2) {$G'$};
          \begin{scope}[shift={(350:2.8cm)}]
            \draw[domain=135:205] plot ({2*cos(\x)}, {2*sin(\x)});
          \end{scope}

          \draw[very thick] (j) -- (b) -- (A) -- (c);
          \draw[very thick] (h) -- (c) -- (g);
          \draw (a) -- (A) -- (d);
          \draw (f) -- (d) -- (e);
          \draw (i) -- (c);
          \draw (m) -- (a) -- (l)  (a) -- (k);
          }

          \only<3->{
          \clip (-.5,1) rectangle (3,-2);
          \node[draw,fill=\mydarkblue, circle,inner sep=2pt] (A) at (0,0) {};
          \node[above] (A') at (0,0) {$v$};
          \node[draw,fill=\myblue, circle, inner sep=2pt] (a) at (20:1.5cm) {};
          \node[draw,fill=\myyellow, circle, inner sep=2pt] (b) at (0:1.5cm) {};
          \node[draw,very thick,fill=\myyellow, circle, inner sep=2pt] (c) at (320:1.5cm) {};
          \node[draw,fill=\myorange, circle, inner sep=2pt] (d) at (340:1.5cm) {};
          
          \node (e) at (335:2.2cm) {};
          \node (f) at (345:2.2cm) {};

          \node (h) at (315:2.2cm) {};
          \node (g) at (320:2.22cm) {};
          \node (i) at (325:2.18cm) {};

          \node (j) at (0:2.2cm) {};

          \node (k) at (15:2.18cm) {};
          \node (l) at (20:2.2cm) {};
          \node (m) at (25:2.18cm) {};

          \node[above] (G) at (2.6, -1.2) {$G'$};
          \begin{scope}[shift={(350:2.8cm)}]
            \draw[domain=135:205] plot ({2*cos(\x)}, {2*sin(\x)});
          \end{scope}

          \draw (j) -- (b) -- (A) -- (c);
          \draw[very thick] (h) -- (c) -- (g);
          \draw (a) -- (A) -- (d);
          \draw (f) -- (d) -- (e);
          \draw (i) -- (c);
          \draw (m) -- (a) -- (l)  (a) -- (k);
          }            
      \end{tikzpicture}
    \end{column}
  \end{columns}

  %% \uncover<4>{\begin{block}{Bounding the reconfiguration diameter of $d$-degenerate graphs}
  %%   \begin{itemize}
  %%   \item total number of Kempe changes : $f(n) \le 2f(n-1) +1$ 
  %%   \item number of recoloring per vertex : $g(n) \le dg(n-1) +1$ 
  %%   \end{itemize}
  %% \end{block}}

\end{frame}

\begin{frame}{Recoloring planar graphs}  
  \begin{block}{Meyniel 1978}
    \begin{itemize}
    \item All 5-colorings of a planar graph are Kempe equivalent 
    \item False for 4-colorings:
    \end{itemize}
  \end{block}
  \begin{columns}
    \begin{column}{.45\textwidth}
    \centering
      \begin{tikzpicture}
        \node[draw=black,fill=\myorange,circle,inner sep=2pt]   (a) at (45:1cm) {};
        \node[draw=black,fill=\myblue,circle,inner sep=2pt]  (b) at (135:1cm) {};
        \node[draw=black,fill=\myyellow,circle,inner sep=2pt] (c) at (225:1cm) {};
        \node[draw=black,fill=\myred,circle,inner sep=2pt] (d) at (315:1cm) {};
        \node[draw=black,fill=\myred,circle,inner sep=2pt] (B) at (135:.5cm) {};
        \node[draw=black,fill=\myblue,circle,inner sep=2pt] (D) at (315:.5cm) {};
        \node[draw=black,fill=\myyellow,circle,inner sep=2pt] (A) at (45:1.8cm) {};
        \node[draw=black,fill=\myorange,circle,inner sep=2pt] (C) at (225:1.8cm) {};
        \node (D') at (315:2cm) {};
        
        \draw (a) -- (b) -- (c) -- (d) -- (a) -- (A) -- (b) -- (B) -- (D) -- (d)
        -- (A);
        \draw (c) -- (B) -- (a) -- (D) -- (c) -- (C)  (b) -- (C) -- (d) ;
        \draw (A) .. controls (D') ..  (C);
      \end{tikzpicture}
    \end{column}
    \hfill
    
    \begin{column}{.45\textwidth}
    \centering
      \begin{tikzpicture}
        \node[draw=black,fill=\myorange,circle,inner sep=2pt]   (a) at (45:1cm) {};
        \node[draw=black,fill=\myblue,circle,inner sep=2pt]  (b) at (135:1cm) {};
        \node[draw=black,fill=\myorange,circle,inner sep=2pt] (c) at (225:1cm) {};
        \node[draw=black,fill=\myblue,circle,inner sep=2pt] (d) at (315:1cm) {};
        \node[draw=black,fill=\myred,circle,inner sep=2pt] (B) at (135:.5cm) {};
        \node[draw=black,fill=\myyellow,circle,inner sep=2pt] (D) at (315:.5cm) {};
        \node[draw=black,fill=\myyellow,circle,inner sep=2pt] (A) at (45:1.8cm) {};
        \node[draw=black,fill=\myred,circle,inner sep=2pt] (C) at (225:1.8cm) {};
        \node (D') at (315:2cm) {};
        
        \draw (a) -- (b) -- (c) -- (d) -- (a) -- (A) -- (b) -- (B) -- (D) -- (d)
        -- (A);
        \draw (c) -- (B) -- (a) -- (D) -- (c) -- (C)  (b) -- (C) -- (d) ;
        \draw (A) .. controls (D') ..  (C);
      \end{tikzpicture}
    \end{column}
  \end{columns}
\end{frame}

\begin{frame}{Bounding the reconfiguration diameter}
  \begin{alertblock}{Conjecture of Bonamy, Bousquet, Feghali, Johnson 19}  
    $R^k(G)$ has diameter $O(n^2)$, if $G$ is $d$-degenerate and $k > d$
  \end{alertblock}
  
  \begin{block}{Bonamy, Delecroix, L. 22+}
    $R^k(G)$ has polynomial diameter for: 
    \begin{itemize}
    \item $k \ge \Delta$ (unless, $k=3$ and $G = K_2 \square K_3$),
    \item $k \ge \mad(G) + \varepsilon$
    \item $k \ge \tw(G) +1$
    \end{itemize}
  \end{block}

  \begin{alertblock}{Journées GrR 2021}
    What about planar graphs with $k = 5$ or $k = 6$ ?
  \end{alertblock}

  \uncover<2>{
  \begin{block}{Deschamps, Feghali, Kardoš, L., Pierron 22+}
    $R^k(G)$ has diameter $O(n^{195})$ if $G$ is planar and $k \ge 5$
  \end{block}}
\end{frame}

\begin{frame}{Recoloring 3-colorable planar graphs}
  \begin{block}{Mohar 2007}
    $R_4(G)$ is connected if $G$ is a 3-colorable planar graph 
  \end{block}

  \begin{block}{Fisk 1977}
    $R_4(G)$ has polynomial diameter if $G$ is a triangulation of the plane
  \end{block}

  \begin{columns}
    \begin{column}{.2\textwidth}
      \centering
      \begin{tikzpicture}
        \node[draw=black,fill=\myblue,circle,inner sep=2pt]   (a) at (45:.5cm) {};
        \node[draw=black,fill=\myorange,circle,inner sep=2pt]  (b) at (135:.5cm) {};
        \node[draw=black,fill=\myblue,circle,inner sep=2pt] (c) at (225:.5cm) {};
        \node[draw=black,fill=\myred,circle,inner sep=2pt] (d) at (315:.5cm) {};

        \draw[very thick] (d) -- (b);
        \draw (d) -- (a) -- (b) -- (c) -- (d);
      \end{tikzpicture}
      
      singular
    \end{column}
    
    \hfill
    \begin{column}{.2\textwidth}
      \centering
      \begin{tikzpicture}
        \node[draw=black,fill=\myblue,circle,inner sep=2pt]   (a) at (45:.5cm) {};
        \node[draw=black,fill=\myorange,circle,inner sep=2pt]  (b) at (135:.5cm) {};
        \node[draw=black,fill=\myyellow,circle,inner sep=2pt] (c) at (225:.5cm) {};
        \node[draw=black,fill=\myred,circle,inner sep=2pt] (d) at (315:.5cm) {};

        \draw[very thick, dashed] (d) -- (b);
        \draw (d) -- (a) -- (b) -- (c) -- (d);
      \end{tikzpicture}

      non-singular
    \end{column}
    \hfill
    \begin{column}{.6\textwidth}
      \centering
      \begin{tikzpicture}
        \node[draw=black,fill=\myorange,circle,inner sep=2pt]   (a) at (0:1.5cm) {};
        \node[draw=black,fill=\myred,circle,inner sep=2pt]  (b) at (60:1cm) {};
        \node[draw=black,fill=\myorange,circle,inner sep=2pt] (c) at (120:2cm) {};
        \node[draw=black,fill=\myred,circle,inner sep=2pt] (d) at (180:.5cm) {};
        \node[draw=black,fill=\myorange,circle,inner sep=2pt] (e) at (240:1.5cm) {};
        \node[draw=black,fill=\myred,circle,inner sep=2pt] (f) at (300:1cm) {};

        \draw[very thick, dashed] (a) -- (b) -- (c) -- (d) -- (e) -- (f) --  (a);

        \begin{scope}[shift = {(-2.5,0)}]
          \node[draw=black,fill=\myorange,circle,inner sep=2pt]   (A) at (0:.5cm) {};
          \node[draw=black,fill=\myred,circle,inner sep=2pt]  (B) at (60:1cm) {};
          \node[draw=black,fill=\myorange,circle,inner sep=2pt] (C) at (180:.8cm) {};
          \node[draw=black,fill=\myred,circle,inner sep=2pt] (D) at (270:.5cm) {};

          \draw[very thick, dashed] (A) -- (B) -- (C) -- (D) -- (A);
        \end{scope}
      \end{tikzpicture}
    \end{column}
  \end{columns}
\end{frame}

\section*{Recoloring planar graphs}
\begin{frame}{Sketch of proof}
  \begin{block}{Theorem}
    For every 5-coloring $\alpha$ and every 4-coloring $\beta$, there is a
    sequence of Kempe changes going from $\alpha$ to $\beta$ recoloring each
    vertex $P(n)$ times
  \end{block}

  \begin{block}{Idea of proof}
    \vspace{-.5cm}
    \begin{columns}
      \hfill
      \begin{column}{.35\textwidth}
        Assumption: there exist $I$,
        \begin{itemize}
        \item $I$ linear size independent set
        \item monochrome in $\alpha$ and in $\beta$
        \item $\forall v \in I, \deg(v) \le 4$
        \end{itemize}
        \vspace{1.5cm}
      \end{column}
      \hfill
      \begin{column}{.65\textwidth}
        $$\begin{array}{lllll}
          G && F = G \setminus I && H = G \setminus B\\
          \alpha &\quad& \alpha_{|F}&\quad& \\
          && \rotatebox[origin=c]{-90}{$\longrightarrow$} \text{ Induction} && \\
          \gamma' && \gamma \text{ avoiding } \alpha(I) &&\\
          \rotatebox[origin=c]{-90}{$\longrightarrow$} \text{ trivial changes}&& && \\
          \delta \text{ s. t.} \delta(B) = 1 && && \delta_{|B} \text{ 4-col}\\
          && && \rotatebox[origin=c]{-90}{$\longrightarrow$} \text{ Fisk} \\
          \beta && && \beta_{|B} \text{ 3-col}\\
        \end{array}
        $$        
      \end{column}
      \hfill
    \end{columns}
  \end{block}
\end{frame}

\begin{frame}{Collapsing}
  By induction:
  \begin{columns}
    \hfill
    \begin{column}{.6\linewidth}
      \begin{itemize}
        \uncover<2->{\item Find a linear-sized independent set $I$, monochrome in $\alpha$ and
          $\beta$ with $\forall v \in I, \deg(v) \le 6$.}
        \uncover<3->{
        \item Collapse $I$ to obtain $H$ s. t.
          \begin{itemize}
          \item $F = H \setminus I$ is planar
          \item $\forall v \in I, \deg_H(v) \le 4$
        \end{itemize}
        Apply induction to recolor $F$ to a 4-coloring avoiding $\alpha(I)$.} 
        \uncover<6->{
        \item Recolor the color class $B$ of $\beta$ that contains $I$, remove $B$
          $\leadsto$ 3-colorable planar graph}
      \end{itemize}
    \end{column}

    \hfill
    \begin{column}{.35\linewidth}
        \begin{tikzpicture}
          \clip (-3,-1.8) rectangle (3,1.8);
          \only<2-5>{\node[draw=black,very thick,fill=\myorange,circle,inner sep=2pt]   (a) at (0,0) {};}
          \only<6->{
            \draw (.35,.35) -- (0,0) -- (0,.5);
            \draw (.35,-.35) -- (0,0) -- (0,-.5) (0,0) -- (.5,0);
            \node[draw=black,fill=\myred,circle,inner sep=2pt]   (a) at (0,0) {};}
          
          \only<2->{\node[draw=black,fill=\myblue,circle,inner sep=2pt]   (b) at (10:1.5cm) {};}
          \only<2-5>{
            \begin{scope}[shift = (70:1.5cm)]
              \draw (.35,.35) -- (0,0) -- (0,.5);
            \end{scope}
            \node[draw=black,fill=\myred,circle,inner sep=2pt]   (c) at
            (70:1.5cm) {};
    }
          \only<2->{\node[draw=black,fill=\myyellow,circle,inner sep=2pt]   (d) at (130:1.5cm) {};}
          \only<2-4>{\node[draw=black,fill=\myblue,circle,inner sep=2pt]   (e) at (190:1.5cm) {};}
          \only<5>{\node[draw=black,fill=\myred,circle,inner sep=2pt]   (e) at (190:1.5cm) {};}
          \only<3>{\node[draw=black,fill=\myyellow,circle,inner sep=2pt]   (f) at
            (250:1.5cm) {};}
          \only<2,4->{\node[draw=black,fill=\mydarkblue,circle,inner sep=2pt]   (f) at
            (250:1.5cm) {};}
          \only<2-5>{
            \begin{scope}[shift = (310:1.5cm)]
              \draw (.35,-.35) -- (0,0) -- (0,-.5) (0,0) -- (.5,0);
            \end{scope}
            \node[draw=black,fill=\myred,circle,inner sep=2pt]   (g) at (310:1.5cm) {};}
          
          \node (ct1) at (170:3cm) {};
          \node (ct2) at (210:3cm) {};
          
          \only<2-5>{\draw (c) -- (d) -- (e) -- (a) -- (f) (g) -- (a) -- (b) -- (c) -- (a) --
            (d);}
          
          \only<6->{\draw (b) -- (a) -- (d);}
          
          
          \only<4,5>{\draw[dashed] (f) .. controls (ct2) and (ct1) .. (d);}
        \end{tikzpicture}
    \end{column}
    \hfill
  \end{columns}
\end{frame}

\begin{frame}{Open questions}
  \begin{alertblock}{Question 1}
    Let $G$ be a $d$-degenerate graph. For $k > d$, do we have
    $\diam(R^k(G)) = O(n^2)$ ?
  \end{alertblock}

  \begin{alertblock}{Question 2}
    If $R^k(G)$ is connected, can we have a non-linear lower bound on
    $\diam(R^k(G))$ ?
  \end{alertblock}

  \begin{alertblock}{Question 3}
    Are there other arguments than frozen colorings to prove that $R^k(G)$ is
    disconnected ?
  \end{alertblock}
  \vspace{1cm}
  \centering
  Thank you !
\end{frame}

\backupbegin
\backupend

\end{document}
