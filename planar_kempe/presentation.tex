\documentclass[11pt,xcolor=dvipsnames,presentation]{beamer}

\usepackage{BeamerColor}
\usepackage{etoolbox}

\usepackage[T1]{fontenc}%
\usepackage[utf8]{inputenc}%
\usepackage[main= english,francais]{babel}%
\usepackage{mathtools}

\usepackage{graphicx}%
\usepackage{courier}%
\usepackage{algorithm2e}

\usepackage{url}%
\urlstyle{sf}%

\usepackage{subcaption}%
\usepackage{caption}
\captionsetup[figure]{labelformat=empty}% redefines the caption setup of the figures environment in the beamer class.
\usepackage{tikz}

\usepackage{listings}%

\usepackage{faktor}%

\usepackage{stmaryrd}%

\usepackage[style=verbose,backend=biber]{biblatex}
\addbibresource{biblio.bib}

\newcommand\blfootnote[1]{%
  \begingroup
  \renewcommand\thefootnote{}\footnote{#1}%
  \addtocounter{footnote}{-1}%
  \endgroup
}

%%%%%%%%% Begin of Beamer Layout %%%%%%%%%%%%%
\ProcessOptionsBeamer
\usecolortheme{whale}
\usecolortheme[named=DarkGreen]{structure}
\useinnertheme{rounded}
\useoutertheme{infolines}
\setbeamertemplate{footline}[frame number]
\setbeamertemplate{headline}[default]
\setbeamertemplate{navigation symbols}{}
\setbeamertemplate{itemize items}[circle]
\defbeamertemplate*{headline}{info theme}{}
\defbeamertemplate*{footline}{info theme}{\leavevmode%
\hbox{%
\begin{beamercolorbox}[wd=.2\paperwidth,ht=2.25ex,dp=1ex,center]{author in head/foot}%
\usebeamerfont{author in head/foot}\insertshortauthor
\end{beamercolorbox}%
\begin{beamercolorbox}[wd=.71\paperwidth,ht=2.25ex,dp=1ex,center]{title in head/foot}%
\usebeamerfont{title in head/foot}\insertsectionhead
\end{beamercolorbox}%
\begin{beamercolorbox}[wd=.09\paperwidth,ht=2.25ex,dp=1ex,right]{section in head/foot}%
\usebeamerfont{section in head/foot}\insertframenumber{}~/~\inserttotalframenumber\hspace*{2ex}
\end{beamercolorbox}
}\vskip0pt}
\setbeamercolor*{block title example}{fg=LightSeaGreen}
\AtBeginEnvironment{exampleblock}{\setbeamercolor{itemize item}{fg=LightSeaGreen}}
\setbeamercolor*{block title alerted}{fg=DarkOrange2}
\AtBeginEnvironment{alertblock}{\setbeamercolor{itemize item}{fg=DarkOrange2}}
\setbeamertemplate{footline}[info theme]
\defbeamertemplate{description item}{align left}{\insertdescriptionitem\hfill}
%%%%%%%%% End of Beamer Layout %%%%%%%%%%%%%
\newcommand{\myorange}{DarkOrange2}
\newcommand{\mygreen}{DarkGreen}
\newcommand{\mycyan}{LightSeaGreen}


\lstset{%
	basicstyle=\sffamily,%
	columns=fullflexible,%
	language=C++,% % À ajuster dans chaque projet!
	frame=lb,%
	frameround=fftf,%
}%

\sloppy

%\setbeamertemplate{headline}{}%

\setbeamertemplate{navigation symbols}{%
	\insertslidenavigationsymbol%
	% \insertframenavigationsymbol%
	% \insertsubsectionnavigationsymbol%
	% \insertsectionnavigationsymbol%
	% \insertdocnavigationsymbol%
	% \insertbackfindforwardnavigationsymbol%
}

\newcommand{\backupbegin}{
   \newcounter{finalframe}
   \setcounter{finalframe}{\value{framenumber}}
}
\newcommand{\backupend}{
   \setcounter{framenumber}{\value{finalframe}}
}

\DeclareMathOperator{\diam}{diam}
\DeclareMathOperator{\mad}{mad}
\DeclareMathOperator{\tw}{tw}

\begin{document}

\title{Kempe recoloring of planar graphs}
\author[Clément Legrand]{Clément Legrand-Duchesne}
\institute{LaBRI, Bordeaux}
%\date{\today}
\date{\today}

\begin{frame}
  \titlepage

  Joint work with Quentin Deschamps, Carl Feghali, František Kardoš and Théo Pierron
\end{frame}

\section*{Some context}
\begin{frame}{Recoloring with Kempe changes}
  \uncover<1->{
  \begin{columns}
    \begin{column}{.66\textwidth}
      \begin{exampleblock}{Kempe chain (1879)}
        Maximal bichromatic connected component in $G$
      \end{exampleblock}
    \end{column}
    \begin{column}{.33\textwidth}
      \centering
      \begin{tikzpicture}
        \node[draw=black,fill=\myorange,circle,inner sep=2pt]   (a) at (90:.8cm) {};
        \node[draw=black,fill=\mycyan,circle,inner sep=2pt]  (b) at (162:.8cm) {};
        \node[draw=black,fill=magenta,circle,inner sep=2pt] (c) at (234:.8cm) {};
        \node[draw=black,fill=blue,circle,inner sep=2pt] (d) at (306:.8cm) {};
        \node[draw=black,fill=magenta,circle,inner sep=2pt] (e) at (18:.8cm) {};

        \draw (a) -- (b) -- (c) -- (d) -- (e) -- (a);
      \end{tikzpicture}
      
      %\includegraphics[width=.8\textwidth]{../figures/pdf/knot_example.pdf}
    \end{column}
  \end{columns}
  }

  \vspace{1.5cm}
  
  \begin{columns}
    \uncover<2->{
      \begin{column}{.33\textwidth}
        \centering
        \begin{tikzpicture}
          \node[draw=black,fill=\myorange,circle,inner sep=2pt]   (a) at (90:.8cm) {};
          \node[draw=black,fill=\mycyan,circle,inner sep=2pt]  (b) at (162:.8cm) {};
          \node[draw=black,very thick,fill=blue,circle,inner sep=2pt] (c) at (234:.8cm) {};
          \node[draw=black,very thick,fill=magenta,circle,inner sep=2pt] (d) at (306:.8cm) {};
          \node[draw=black,very thick,fill=blue,circle,inner sep=2pt] (e) at (18:.8cm) {};

          \draw (e) -- (a) -- (b) -- (c);
          \draw[very thick] (c) -- (d) -- (e);
        \end{tikzpicture}      
    \end{column}
    }
    \uncover<3->{
    \begin{column}{.33\textwidth}
       \centering
      \begin{tikzpicture}
        \node[draw=black,fill=\myorange,circle,inner sep=2pt]   (a) at (90:.8cm) {};
        \node[draw=black,very thick,fill=magenta,circle,inner sep=2pt]  (b) at (162:.8cm) {};
        \node[draw=black,very thick,fill=\mycyan,circle,inner sep=2pt] (c) at (234:.8cm) {};
        \node[draw=black,fill=blue,circle,inner sep=2pt] (d) at (306:.8cm) {};
        \node[draw=black,fill=magenta,circle,inner sep=2pt] (e) at (18:.8cm) {};

        \draw[very thick] (b) -- (c);
        \draw (a) -- (b);
        \draw (c) -- (d) -- (e) -- (a);
      \end{tikzpicture}
    \end{column}
    }
    \uncover<4->{
    \begin{column}{.33\textwidth}
       \centering
      \begin{tikzpicture}
        \node[draw=black,fill=\myorange,circle,inner sep=2pt]   (a) at (90:.8cm) {};
        \node[draw=black,fill=\mycyan,circle,inner sep=2pt]  (b) at (162:.8cm) {};
        \node[draw=black,fill=magenta,circle,inner sep=2pt] (c) at (234:.8cm) {};
        \node[draw=black,very thick,fill=\mycyan,circle,inner sep=2pt] (d) at (306:.8cm) {};
        \node[draw=black,fill=magenta,circle,inner sep=2pt] (e) at (18:.8cm) {};

        \draw (a) -- (b) -- (c) -- (d) -- (e) -- (a);
      \end{tikzpicture}
    \end{column}
    }
  \end{columns}
  \uncover<5->{
  \begin{exampleblock}{Reconfiguration graph}
    \begin{itemize}
    \item $V(R^k(G)) =$ $k$-colorings of $G$.
    \item $\alpha$ and $\beta$ adjacent  if $\alpha
      \xleftrightarrow[\text{Kempe}]{} \beta$
    \end{itemize}
  \end{exampleblock}}

\end{frame}

\begin{frame}{Recoloring planar graphs}  
  \begin{block}{Meyniel 1978}
    \begin{itemize}
    \item All 5-colorings of a planar graph are Kempe equivalent 
    \item False for 4-colorings:
    \end{itemize}
  \end{block}
  \begin{columns}
    \begin{column}{.45\textwidth}
    \centering
      \begin{tikzpicture}
        \node[draw=black,fill=\myorange,circle,inner sep=2pt]   (a) at (45:1cm) {};
        \node[draw=black,fill=\mycyan,circle,inner sep=2pt]  (b) at (135:1cm) {};
        \node[draw=black,fill=\mygreen,circle,inner sep=2pt] (c) at (225:1cm) {};
        \node[draw=black,fill=magenta,circle,inner sep=2pt] (d) at (315:1cm) {};
        \node[draw=black,fill=magenta,circle,inner sep=2pt] (B) at (135:.5cm) {};
        \node[draw=black,fill=\mycyan,circle,inner sep=2pt] (D) at (315:.5cm) {};
        \node[draw=black,fill=\mygreen,circle,inner sep=2pt] (A) at (45:1.8cm) {};
        \node[draw=black,fill=\myorange,circle,inner sep=2pt] (C) at (225:1.8cm) {};
        \node (D') at (315:2cm) {};
        
        \draw (a) -- (b) -- (c) -- (d) -- (a) -- (A) -- (b) -- (B) -- (D) -- (d)
        -- (A);
        \draw (c) -- (B) -- (a) -- (D) -- (c) -- (C)  (b) -- (C) -- (d) ;
        \draw (A) .. controls (D') ..  (C);
      \end{tikzpicture}
    \end{column}
    \hfill
    
    \begin{column}{.45\textwidth}
    \centering
      \begin{tikzpicture}
        \node[draw=black,fill=\myorange,circle,inner sep=2pt]   (a) at (45:1cm) {};
        \node[draw=black,fill=\mycyan,circle,inner sep=2pt]  (b) at (135:1cm) {};
        \node[draw=black,fill=\myorange,circle,inner sep=2pt] (c) at (225:1cm) {};
        \node[draw=black,fill=\mycyan,circle,inner sep=2pt] (d) at (315:1cm) {};
        \node[draw=black,fill=magenta,circle,inner sep=2pt] (B) at (135:.5cm) {};
        \node[draw=black,fill=\mygreen,circle,inner sep=2pt] (D) at (315:.5cm) {};
        \node[draw=black,fill=\mygreen,circle,inner sep=2pt] (A) at (45:1.8cm) {};
        \node[draw=black,fill=magenta,circle,inner sep=2pt] (C) at (225:1.8cm) {};
        \node (D') at (315:2cm) {};
        
        \draw (a) -- (b) -- (c) -- (d) -- (a) -- (A) -- (b) -- (B) -- (D) -- (d)
        -- (A);
        \draw (c) -- (B) -- (a) -- (D) -- (c) -- (C)  (b) -- (C) -- (d) ;
        \draw (A) .. controls (D') ..  (C);
      \end{tikzpicture}
    \end{column}
  \end{columns}

  \begin{block}{Las Vergnas and Meyniel 1981}
    All the $k$ colorings of a $d$-degenerate graph are Kempe equivalent, for $k>d$
  \end{block}
\end{frame}

\begin{frame}{Bounded diameter}
  \begin{alertblock}{Conjecture of Bonamy, Bousquet, Feghali, Johnson 19}  
    $R_k(G)$ has diameter $O(n^2)$, if $G$ is $d$-degenerate and $k > d$
  \end{alertblock}
  
  \begin{block}{Bonamy, Delecroix, L. 22+}
    $R_k(G)$ has polynomial diameter for: 
    \begin{itemize}
    \item $k \ge \Delta$ (unless, $k=3$ and $G = K_2 \square K_3$),
    \item $k \ge \mad(G) + \varepsilon$
    \item $k \ge \tw(G)$
    \end{itemize}
  \end{block}

  \begin{alertblock}{Journées GrR 2021}
    What about planar graphs with $k = 5$ or $k = 6$ ?
  \end{alertblock}

  \begin{block}{Deschamps, Feghali, Kardoš, L., Pierron 22+}
    $R_k(G)$ has polynomial diameter if $G$ is planar and $k \ge 5$
  \end{block}
\end{frame}

\section{Warm-up}
\begin{frame}{3-colorable planar graphs}
  \begin{block}{Mohar 2007}
    $R_4(G)$ is connected if $G$ is a 3-colorable planar graph 
  \end{block}

  \begin{block}{Fisk 1977}
    $R_4(G)$ has polynomial diameter if $G$ is a triangulation of the plane
  \end{block}

  \begin{columns}
    \begin{column}{.2\textwidth}
      \centering
      \begin{tikzpicture}
        \node[draw=black,fill=\mycyan,circle,inner sep=2pt]   (a) at (45:.5cm) {};
        \node[draw=black,fill=\myorange,circle,inner sep=2pt]  (b) at (135:.5cm) {};
        \node[draw=black,fill=\mycyan,circle,inner sep=2pt] (c) at (225:.5cm) {};
        \node[draw=black,fill=magenta,circle,inner sep=2pt] (d) at (315:.5cm) {};

        \draw[very thick] (d) -- (b);
        \draw (d) -- (a) -- (b) -- (c) -- (d);
      \end{tikzpicture}
      
      singular
    \end{column}
    
    \hfill
    \begin{column}{.2\textwidth}
      \centering
      \begin{tikzpicture}
        \node[draw=black,fill=\mycyan,circle,inner sep=2pt]   (a) at (45:.5cm) {};
        \node[draw=black,fill=\myorange,circle,inner sep=2pt]  (b) at (135:.5cm) {};
        \node[draw=black,fill=\mygreen,circle,inner sep=2pt] (c) at (225:.5cm) {};
        \node[draw=black,fill=magenta,circle,inner sep=2pt] (d) at (315:.5cm) {};

        \draw[very thick, dashed] (d) -- (b);
        \draw (d) -- (a) -- (b) -- (c) -- (d);
      \end{tikzpicture}

      non-singular
    \end{column}
    \hfill
    \begin{column}{.6\textwidth}
      \centering
      \begin{tikzpicture}
        \node[draw=black,fill=\myorange,circle,inner sep=2pt]   (a) at (0:1.5cm) {};
        \node[draw=black,fill=magenta,circle,inner sep=2pt]  (b) at (60:1cm) {};
        \node[draw=black,fill=\myorange,circle,inner sep=2pt] (c) at (120:2cm) {};
        \node[draw=black,fill=magenta,circle,inner sep=2pt] (d) at (180:.5cm) {};
        \node[draw=black,fill=\myorange,circle,inner sep=2pt] (e) at (240:1.5cm) {};
        \node[draw=black,fill=magenta,circle,inner sep=2pt] (f) at (300:1cm) {};

        \draw[very thick, dashed] (a) -- (b) -- (c) -- (d) -- (e) -- (f) --  (a);

        \begin{scope}[shift = {(-2.5,0)}]
          \node[draw=black,fill=\myorange,circle,inner sep=2pt]   (A) at (0:.5cm) {};
          \node[draw=black,fill=magenta,circle,inner sep=2pt]  (B) at (60:1cm) {};
          \node[draw=black,fill=\myorange,circle,inner sep=2pt] (C) at (180:.8cm) {};
          \node[draw=black,fill=magenta,circle,inner sep=2pt] (D) at (270:.5cm) {};

          \draw[very thick, dashed] (A) -- (B) -- (C) -- (D) -- (A);
        \end{scope}
      \end{tikzpicture}
    \end{column}
  \end{columns}
\end{frame}

\section{General case}
\begin{frame}{Sketch of proof}
  \begin{block}{Theorem}
    Each 5-coloring is equivalent to a 4-coloring up to a polynomial number of
    Kempe changes
  \end{block}

  
\end{frame}
\end{document}
